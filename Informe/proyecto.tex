\documentclass[a4paper,11pt]{book}
%\documentclass[a4paper,twoside,11pt,titlepage]{book}
\usepackage{listings}
\usepackage[utf8]{inputenc}
\usepackage[spanish,es-tabla]{babel}

\usepackage{textcomp}
\usepackage{multirow}
\usepackage[table,xcdraw]{xcolor}

% \usepackage[style=list, number=none]{glossary} %
%\usepackage{titlesec}
%\usepackage{pailatino}

\decimalpoint
\usepackage{dcolumn}
\newcolumntype{.}{D{.}{\esperiod}{-1}}
\makeatletter
\addto\shorthandsspanish{\let\esperiod\es@period@code}
\makeatother


%\usepackage[chapter]{algorithm}
\RequirePackage{verbatim}
%\RequirePackage[Glenn]{fncychap}
\usepackage{fancyhdr}
\usepackage{graphicx}
\usepackage{afterpage}

\usepackage{longtable}

%\usepackage[pdfborder={000}]{hyperref} %referencia
\usepackage[colorlinks=true,urlcolor=blue,linkcolor=red]{hyperref}

% ********************************************************************
% Re-usable information
% ********************************************************************
\newcommand{\myTitle}{Diseño de Metaheurística para problemas combinatorios costosos\xspace}
\newcommand{\myDegree}{Doble Grado en Ingeniería Informática y Matemáticas\xspace}
\newcommand{\myName}{Irene Trigueros Lorca\xspace}
\newcommand{\myProf}{Daniel Molina Cabrera\xspace}
\newcommand{\myOtherProf}{Francisco Herrera Triguero\xspace}
%\newcommand{\mySupervisor}{Put name here\xspace}
\newcommand{\myFaculty}{Escuela Técnica Superior de Ingenierías Informática y de
Telecomunicación\\Facultad de Ciencias\xspace}
\newcommand{\myFacultyShort}{E.T.S. de Ingenierías Informática y de
Telecomunicación\\Facultad de Ciencias\xspace}
\newcommand{\myDepartment}{Departamento de ...\xspace}
\newcommand{\myUni}{\protect{Universidad de Granada}\xspace}
\newcommand{\myLocation}{Granada\xspace}
\newcommand{\myTime}{\today\xspace}
\newcommand{\myVersion}{Version 0.1\xspace}


\hypersetup{
pdfauthor = {\myName irenetrigueros@correo.ugr.es},
pdftitle = {\myTitle},
pdfsubject = {},
pdfkeywords = {palabra_clave1, palabra_clave2, palabra_clave3, ...},
pdfcreator = {LaTeX con el paquete ....},
pdfproducer = {pdflatex}
}

%\hyphenation{}


%\usepackage{doxygen/doxygen}
%\usepackage{pdfpages}
\usepackage{url}
\usepackage{colortbl,longtable}
\usepackage[stable]{footmisc}
%\usepackage{index}

%\makeindex
%\usepackage[style=long, cols=2,border=plain,toc=true,number=none]{glossary}
% \makeglossary

% Definición de comandos que me son tiles:
%\renewcommand{\indexname}{Índice alfabético}
%\renewcommand{\glossaryname}{Glosario}

\pagestyle{fancy}
\fancyhf{}
\fancyhead[LO]{\leftmark}
\fancyhead[RE]{\rightmark}
\fancyhead[RO,LE]{\textbf{\thepage}}
\renewcommand{\chaptermark}[1]{\markboth{\textbf{#1}}{}}
\renewcommand{\sectionmark}[1]{\markright{\textbf{\thesection. #1}}}

%\setlength{\headheight}{1.5\headheight}

\newcommand{\HRule}{\rule{\linewidth}{0.5mm}}
%Definimos los tipos teorema, ejemplo y definición podremos usar estos tipos
%simplemente poniendo \begin{teorema} \end{teorema} ...
\newtheorem{teorema}{Teorema}[chapter]
\newtheorem{ejemplo}{Ejemplo}[chapter]
\newtheorem{definicion}{Definición}[chapter]
\newtheorem{proposicion}{Proposición}[chapter]

\definecolor{gray97}{gray}{.97}
\definecolor{gray75}{gray}{.75}
\definecolor{gray45}{gray}{.45}
\definecolor{gray30}{gray}{.94}

\lstset{ frame=Ltb,
     framerule=0.5pt,
     aboveskip=0.5cm,
     framextopmargin=3pt,
     framexbottommargin=3pt,
     framexleftmargin=0.1cm,
     framesep=0pt,
     rulesep=.4pt,
     backgroundcolor=\color{gray97},
     rulesepcolor=\color{black},
     %
     stringstyle=\ttfamily,
     showstringspaces = false,
     basicstyle=\scriptsize\ttfamily,
     commentstyle=\color{gray45},
     keywordstyle=\bfseries,
     %
     numbers=left,
     numbersep=6pt,
     numberstyle=\tiny,
     numberfirstline = false,
     breaklines=true,
   }
 
% minimizar fragmentado de listados
\lstnewenvironment{listing}[1][]
   {\lstset{#1}\pagebreak[0]}{\pagebreak[0]}

\lstdefinestyle{CodigoC}
   {
	basicstyle=\scriptsize,
	frame=single,
	language=C,
	numbers=left
   }
\lstdefinestyle{CodigoC++}
   {
	basicstyle=\small,
	frame=single,
	backgroundcolor=\color{gray30},
	language=C++,
	numbers=left
   }

 
\lstdefinestyle{Consola}
   {basicstyle=\scriptsize\bf\ttfamily,
    backgroundcolor=\color{gray30},
    frame=single,
    numbers=none
   }


\newcommand{\bigrule}{\titlerule[0.5mm]}

%Para que el número de capítulo y el título estén en la misma línea
\usepackage{titlesec}
\titleformat{\chapter}[hang] 
{\normalfont\huge\bfseries}{\chaptertitlename\ \thechapter:}{1em}{} 

%Para conseguir que en las páginas en blanco no ponga cabecerass
\makeatletter
\def\clearpage{%
  \ifvmode
    \ifnum \@dbltopnum =\m@ne
      \ifdim \pagetotal <\topskip
        \hbox{}
      \fi
    \fi
  \fi
  \newpage
  \thispagestyle{empty}
  \write\m@ne{}
  \vbox{}
  \penalty -\@Mi
}
\makeatother

\usepackage{lmodern}
\usepackage[T1]{fontenc}
\usepackage{mathtools}
\usepackage{vmargin}
\usepackage{caption}
\usepackage{subcaption}
\usepackage{float}
\usepackage{amsmath}
\usepackage{amsfonts}
\usepackage{amssymb}
\usepackage{enumerate}
\usepackage{schemata}
\usepackage{hyperref}
\usepackage{enumitem}
\usepackage[all]{hypcap}    %for going to the top of an image when a figure reference is clicked
\newcommand\diagram[2]{\schema{\schemabox{#1}}{\schemabox{#2}}}
\usepackage{xspace}
\usepackage{algorithm}
\usepackage{algpseudocode}
\usepackage{cleveref}
\usepackage{biblatex}
\addbibresource{TFG.bib}

\usepackage{float}
\floatstyle{plaintop}
\restylefloat{table}
\usepackage[paper=portrait,pagesize]{typearea}

\algnewcommand\algorithmicforeach{\textbf{for each}}
\algdef{S}[FOR]{ForEach}[1]{\algorithmicforeach\ #1\ \algorithmicdo}


\setlength{\parskip}{0.5em}

\setpapersize{A4}
\setmargins{2.25cm}       % margen izquierdo
{1.5cm}                        % margen superior
{16.5cm}                      % anchura del texto
{23.42cm}                    % altura del texto
{10pt}                           % altura de los encabezados
{1cm}                           % espacio entre el texto y los encabezados
{0pt}                             % altura del pie de página
{2cm}                           % espacio entre el texto y el pie de página
\usepackage{pdfpages}
\begin{document}
\begin{titlepage}
 
 
\newlength{\centeroffset}
\setlength{\centeroffset}{-0.5\oddsidemargin}
\addtolength{\centeroffset}{0.5\evensidemargin}
\thispagestyle{empty}

\noindent\hspace*{\centeroffset}\begin{minipage}{\textwidth}

\centering
\includegraphics[width=0.8\textwidth]{imagenes/logo_ugr_nuevo.png}\\[1.4cm]

\textsc{ \Large TRABAJO FIN DE GRADO\\[0.2cm]}
\textsc{ DOBLE GRADO EN INGENIERÍA INFORMÁTICA Y MATEMÁTICAS}\\[1cm]
% Upper part of the page
% 
% Title
{\Huge\bfseries Diseño de Metaheurística para problemas combinatorios costosos\\
}
\noindent\rule[-1ex]{\textwidth}{3pt}\\[3.5ex]
%{\large\bfseries Subtitulo del Proyecto}
\end{minipage}

\vspace{1cm}
\noindent\hspace*{\centeroffset}\begin{minipage}{\textwidth}
\centering

\textbf{Autor}\\ {Irene Trigueros Lorca}\\[2.5ex]
\textbf{Directores}\\
{Daniel Molina Cabrera\\
Francisco Herrera Triguero}\\[1cm]

\begin{tabular}{c c}
\includegraphics[scale=1]{imagenes/etsiit_logo.png} & \includegraphics[scale=0.2]{imagenes/ciencias_logo.png}\\
\textsc{Escuela Técnica Superior de}\\\textsc{Ingenierías Informática}\\\textsc{y de Telecomunicación} & \textsc{Facultad de Ciencias} \\
\end{tabular} 

%\includegraphics[width=0.3\textwidth]{imagenes/etsiit_logo.png}\\[0.1cm]
%\textsc{Escuela Técnica Superior de Ingenierías Informática y de Telecomunicación}\\
\textsc{---}\\
Granada, Junio de 2023
\end{minipage}
%\addtolength{\textwidth}{\centeroffset}
%\vspace{\stretch{2}}
\end{titlepage}



\chapter*{}
%\thispagestyle{empty}
%\cleardoublepage

%\thispagestyle{empty}

\input{portada/portada_2}



\cleardoublepage
\thispagestyle{empty}

\begin{center}
{\large\bfseries Diseño de Metaheurística para problemas combinatorios costosos}\\
\end{center}
\begin{center}
Irene Trigueros Lorca\\
\end{center}

%\vspace{0.7cm}
\noindent{\textbf{Palabras clave}: metaheurística, combinatorio, costoso, optimización, algoritmo genético, exploración, explotación, diversidad}\\

\vspace{0.7cm}
\noindent{\textbf{Resumen}}\\

Existen problemas de mucho interés, no solo teórico, sino también que poseen aplicaciones significativas en el mundo real, cuyo coste de evaluación es excesivamente elevado, o incluso inasequibles (a este tipo de problemas los llamaremos problemas costosos o problemas \textit{expensive}), como podrían ser los problemas de optimización de redes neuronales, por lo que resulta interesante desarrollar algoritmos capaces de obtener soluciones competitivas en muy pocas evaluaciones. 
Además, en la literatura ya es posible encontrar algoritmos específicos en problemas \textit{expensives} de optimización continua; pero no ocurre lo mismo con el caso de problemas combinatorios, para los que no se han diseñado algoritmos específicamente para este tipo de problemas, pero que son importantes de tratar ya que es cada vez más común usar problemas combinatorios en problemas complejos.

Por lo tanto, en este proyecto se propone un algoritmo metaheurístico nuevo especialmente diseñado para la resolución problemas combinatorios costosos (\textit{expensive}), siguiendo un formato semejante al de un diario de desarrollo. 
El objetivo principal de este algoritmo será encontrar buenas soluciones en una reducida cantidad de tiempo. 

Inicialmente se presentará un contexto matemático necesario para entender cómo resolver problemas de minimización sin restricciones y el cómo y porqué usar tests estadísticos (y cuáles usar) para la comparación de los distintos algoritmos. 

Tras esto, se sigue la exposición de los algoritmos genéticos clásicos que se han usado como base para el diseño del algoritmo a desarrollar y las modificaciones que se le introducirán a estos conforme se avance en el desarrollo del algoritmo; ambas contendrán pseudocódigo asociado para su más fácil interpretación. 

El diseño del algoritmo objetivo se hará como se indica a continuación. 
En primer lugar, se tomará como algoritmo base un algoritmo genético y, una vez analizado su rendimiento, se implementarán una serie de modificaciones que, progresivamente, compondrán la versión final del algoritmo. 
Estas modificaciones se justificarán observando y analizando los resultados obtenidos en los intentos anteriores con el objetivo de encontrar formas de aprovechar al máximo las pocas evaluaciones que nos podemos permitir en este tipo de problemas. 
Se describen detalladamente las tareas adicionales llevadas a cabo propias del desarrollo de un algoritmo de esta clase. 
Además, se realizarán análisis experimentales que irán demostrando que las sucesivas versiones del algoritmo van mejorando las anteriores. 

Como no existen algoritmos específicos para la resolución de problemas combinatorios costosos, no podemos comparar los resultados finales obtenidos con ningún tipo de \textit{benchmark} con el fin de obtener unas conclusiones. 
Por ello, el análisis de la potencia del algoritmo desarrollado se basará en la comparación con los resultados de las anteriores versiones desarrolladas y un análisis del porcentaje de mejora con respecto algoritmo genético clásico que se ha usado como base. 
Las conclusiones que se alcanzan indican que el algoritmo presentado en este proyecto es altamente competitivo. 


\cleardoublepage


\thispagestyle{empty}


\begin{center}
{\large\bfseries Metaheuristics design for expensive combinatorial problems }\\
\end{center}
\begin{center}
Irene Trigueros Lorca\\
\end{center}

%\vspace{0.7cm}
\noindent{\textbf{Keywords}: metaheuristic, combinatory, expensive, optimization, genetic algorithm, exploration, explotation, diversity}\\

\vspace{0.7cm}
\noindent{\textbf{Abstract}}\\

There are very interesting problems, not only in the theoretical field, but they also have significants applications in real life, which their computational cost is excessively high, or even unafforadable (these kind of problems will be refered as expensive problems), such as the neural network optimization problems, therefore it is compelling to develope algorithms that are able to obtain competitive solutions in few evaluations. 
Furthermore, specific algorithms for the optimization of expensive continous problems can be found in the literature; but that is not the case for the combinatory expensive problems, for those which no algorithms able to solve them have been developed, though they are important to deal with as is increasingly common to use combinatorial problems in complex problems.

Hence, throughout this project, an original metaheuristic algorithm is going to be proposed with the purpose of solving combinatory expensive problems, following a format similar to that of a developer diary. 
The main objective of this algorithm will be to find good solutions in a greatly reduced amount of  time. 

Initially, a mathematical frame necessary to understand how to solve unconstrained minimization problems and how and why to use statistical tests (and which ones to use) for the comparison of the different algorithms will be presented. 

This will be followed by an exposition fo the classical genetic algorithms that have been used as a basis for the design of the algorithm to be developed and the modifications that will be introduced to it as the algorithm development progresses will be presented; both of them will contain asssociated pseudocode for an easier interpretation. 

The design of the target algorithm will be carried out as follows. 
First, a genetic algorithm is going to be chosen as the base algorithm and, once its performance is analyzed, a series of modifications are going  to be implemented, which, at the end, will compose the final version of the algorithm. 
These modifications are going to be justified by observing and analyzing the results obtained through the previous attempts with the objective of finding ways of making the most of the few iterations available in these kind of problems. 
Additional tasks regarding the development of this type of algorithms are described in detail. 
%Furthermore, experimental analysis is included, where both base and the following versions algorithms are compared. 
Furthermore, experimental analyses will be performed to show that successive version of the algorithm improve on the previous ones. 

Since there are no specific algorithms for solving expensive combinatory problems, we are not able to compare the final results obtained with any kind of benchmark in order to obtain conclusions. 
Therefore, the analysis of the performance of the developed algorithm will be based on the comparison with the results of the previous developed versions and an analysis of the percentage of improvement regarding the classical genetic algorithm that has been used as a basis. 
The conclusions reached indicate that the algorithm presented within this project is highly performant.

\chapter*{}
\thispagestyle{empty}

\noindent\rule[-1ex]{\textwidth}{2pt}\\[4.5ex]

Yo, \textbf{Irene Trigueros Lorca}, alumna de la titulación Doble Grado en Ingeniería Informática y Matemáticas de la \textbf{Escuela Técnica Superior de Ingenierías Informática y de Telecomunicación de la Universidad de Granada} y de la \textbf{Facultad de Ciencias}, con DNI 77385991F, autorizo la
ubicación de la siguiente copia de mi Trabajo Fin de Grado en la biblioteca de ambos centro para que pueda ser consultada por las personas que lo deseen.

\vspace{6cm}

\noindent Fdo: Irene Trigueros Lorca

\vspace{2cm}

\begin{flushright}
Granada, a \today.
\end{flushright}


\chapter*{}
\thispagestyle{empty}

\noindent\rule[-1ex]{\textwidth}{2pt}\\[4.5ex]

D. \textbf{Daniel Molina Cabrera}, Profesor del Departamento Ciencias de la Computación e Inteligencia Artificial de la Universidad de Granada.

\vspace{0.5cm}

D. \textbf{Francisco Herrera Triguero}, Profesor del Departamento Ciencias de la Computación e Inteligencia Artificial de la Universidad de Granada.


\vspace{0.5cm}

\textbf{Informan:}

\vspace{0.5cm}

Que el presente trabajo, titulado \textit{\textbf{Diseño de Metaheurística para problemas combinatorios costosos}},
ha sido realizado bajo su supervisión por \textbf{Irene Trigueros Lorca}, y autorizamos la defensa de dicho trabajo ante el tribunal
que corresponda.

\vspace{0.5cm}

Y para que conste, expiden y firman el presente informe en Granada a \today.

\vspace{1cm}

\textbf{Los directores:}

\vspace{5cm}

\noindent \textbf{Daniel Molina Cabrera \hspace{5cm} Francisco Herrera Triguero}

\chapter*{Agradecimientos}
\thispagestyle{empty}

       \vspace{1cm}


En primer lugar, quiero agradecer y dedicar este proyecto a mis padres porque gracias a ellos he conseguido llegar a este punto de mi vida, por apoyarme en todo lo que pueden, por educarme para ser una persona que ponga todos sus esfuerzos en conseguir lo que se propone y por las incontables horas de llamadas a lo largo de estos cinco años aunque fuese solo para preguntarme si me iba bien o tenía algún problema. 

También a mi hermana que siempre sabe como animarme y me ha inspirado todos estos años a mejorar como persona para poder ser la mejor influencia para ella. 

A todos los amigos que he hecho en la universidad, por todas las horas de estudio, por todo el apoyo y la ayuda que nos hemos aportado a lo largo de los años, por todas las conversaciones absurdas solo para distraernos de los problemas... 
Empezaron como desconocidos y se convirtieron en una segunda familia. 

Finalmente, agradecer a mis tutores, ya que sin ellos este trabajo no habría sido posible. 
Pero sobre todo agradecer a Daniel Molina Cabrera, por haberme guiado y aconsejado a lo largo del desarrollo de este trabajo, así como por haber estado disponible para ayudarme siempre que lo he necesitado, aún sin disponer de tiempo para ello. 
\frontmatter
{ \hypersetup{hidelinks} \tableofcontents }
\listoffigures
%\listoftables
%
\mainmatter
\setlength{\parskip}{5pt}

\chapter{Introducción}

%<párrafo indicando lo útil que son las metaheurísticas, tanto para problemas continuos como combinatorios>
Cuando se debe afrontar un problema, la variedad de aproximaciones a seguir resulta ser muy amplia y su elección depende en gran medida de dos factores fundamentales: qué clase de solución se desea extraer del problema, y de qué recursos se dispone para ello. 
Por ello, la rama más clásica de la computación siempre ha tratado de resolver los problemas presentados de forma exacta. 
Es decir, ha tratado cada problema como si solo existiera una sola solución al mismo, la óptima. 
Esta forma de pensamiento se basa en la confianza que se tiene en que los problemas que tradicionalmente eran irresolubles para los humanos, serían más accesibles para los computadores, gracias a su capacidad superior de cómputo pesado.

Esto último resulta cierto en muchos escenarios, fundamentalmente en lo referente a los problemas más puramente matemáticos: operaciones que un humano podría tardar años en resolver a mano, un ordenador podría resolverlas en cuestión de minutos. 
Sin embargo, la mayoría de problemas que nos encontramos en el mundo real son complejos y difíciles de resolver, lo que implica que no se pueda dar con la solución óptima en un tiempo razonable. 

%Resumir esta parte porque no es un problema expensive
Uno de los problemas más conocidos capaz de ilustrar este hecho es el \textbf{Problema del Viajante de Comercio}. 
Este problema consiste en que, dado un conjunto de ciudades por las que el comercial debe pasar, se debe encontrar el orden en que visita las ciudades de forma que el comercial recorra la menor distancia posible. 
Aunque sea un problema sencillo de formular, la cantidad de posibles caminos incrementa en gran medida a la vez que el número de ciudades a recorrer aumenta. 
Por tanto, al aumentar el tamaño del problema, toda técnica conocida para extraer la solución exacta requeriría de un tiempo de ejecución que deja de ser asequible incluso para los ordenadores. 
%Es decir, tenemos métodos que nos llevan a la solución, pero no existen formas de ejecutarlos. 
A este tipo de problemas se les denominan problemas NP (\textit{Nondeterministic Polynomial Time}), por lo que no pueden ser resueltos de forma realista exacta en tiempo polinomial. 
%Llegados a este punto, si bien tenemos que dejar de plantearnos la pregunta ``¿cuál es el camino más corto?'', podemos empezar a formular una pregunta similar: ``¿qué camino es lo suficientemente corto?''; o sea, a veces tenemos que abandonar la idea de obtener la solución óptima y conformarnos con una solución aproximada de una calidad similar a la óptima.

Por ello, tenemos que encontrar alternativas más viables para la resolución de los problemas, es decir, tenemos que usar algoritmos aproximados, que proporcionan buenas soluciones (no necesariamente la óptima) en un tiempo razonable. 
Esto es, haremos uso de estrategias de diseño generales para procedimientos heurísticos de resolución de problemas: las metaheurísticas.
%Los algoritmos aproximados se pueden dividir en heurísticas y metaheurísticas, estas últimas son en las que vamos a estar más interesados. 
%Las metaheurísticas suelen ser procedimientos iterativos que guían una heurística subordinada de búsqueda, combinando de forma inteligente distintos conceptos para explorar y explotar adecuadamente el espacio de búsqueda.
%Así, las metaheurísticas son una familia de algoritmos aproximados más generales que las heurísticas (que son dependientes del problema a tratar) y aplicables a una gran variedad de problemas de optimización, tanto continuos como combinatorios. 

%<Existen problemas expensives, lo que es y algún ejemplo, y en continuo ya existe algoritmos específicos>
Los problemas de optimización costosa (EOP) o \textbf{problemas \textit{expensives}} se refieren a los problemas que requieren costes elevados, o incluso inasequibles, con el fin de evaluar los candidatos a soluciones. 
Este tipo de problemas existen en una gran cantidad, y cada vez con más frecuencia, de aplicaciones significativas del mundo real. 
Un tipo de problemas \textit{expensive} podrían ser los problemas de optimización de redes neuronales profundas utilizando metaheurísticas. 
Un ejemplo esto se podría encontrar en el artículo \parencite{buiMetaheuristicAlgorithmsOptimizing2019}; en ese estudio se proponen y se comparan tres métodos híbridos en combinación con el popular clasificador con redes neuronales para el modelado de incendios forestales.

También, cabe destacar que ``coste elevado'' es más un concepto relativo que uno absoluto en la mayoría de problemas del mundo real. 
Por ejemplo, en situaciones dinámicas, como podría ser recalcular una ruta porque se haya cortado una calle o porque se ha producido un atasco, o incluso situaciones de emergencia como epidemias o desastres naturales, transporte y envío de materiales para operaciones diarias importantes para salvar vidas, etc., el coste de optimización en situaciones normales se convierte en un coste demasiado elevado. 
Para el caso de los problemas de parámetros reales nos encontramos con que se están planteando cada vez más algoritmos especialmente diseñados para problemas \textit{expensive} (por ejemplo, que la función de evaluación dependa de una simulación).

%<El problema también se da en combinatorio, por nuevos problemas.>
Ahora bien, cada vez es más común usar problemas combinatorios en problemas complejos, lo que implica un mayor coste de evaluación. 
Un ejemplo de esto lo podemos encontrar en el artículo \parencite{demoraesDiversityPreservationMethod2022}, donde en su introducción se detalla el por qué el \textit{Well Placement Optimization Problem} (WPOP) en el desarrollo y gestión de campos petroleros se considera un problema \textit{expensive} (se debe al cálculo tan complejo de ecuaciones diferenciales para predecir la influencia de la estrategia de producción  en las propiedades geológicas y petrofísicas de la reserva). 
%Un ejemplo de esto, como ya hemos comentado anteriormente, podría ser el uso de este tipo de algoritmos para optimizar redes neuronales. 
Si bien hemos comentado que en el caso de parámetros reales (caso continuo) se han propuesto algoritmos específicos, esto no ha sido el caso para el ámbito de los problemas combinatorios \textit{expensive}, para los que no se han encontrado ninguna referencia. 
Por lo que es de gran interés crear un algoritmo que resulte útil para este tipo de situaciones, con el fin de reducir los costes todo lo posible. 

%<En este trabajo vamos a ...>
En este Trabajo de Fin de Grado vamos a diseñar, implementar y proponer un algoritmo especialmente diseñado para problemas combinatorios \textit{expensive}. 
El objetivo principal de este algoritmo será encontrar buenas soluciones en una cantidad de tiempo bastante reducida, lo que vamos a traducir en evaluar un número muy reducido de soluciones. 
Para ello, procederemos a tomar un problema sobre el que trabajar y extraer conclusiones y un algoritmo de referencia sobre el que realizaremos un proceso recursivo: introduciremos alguna modificación y compararemos los resultados obtenidos con los del algoritmo de referencia, en caso de mejorarlos, este algoritmo con modificación se convertirá en el nuevo algoritmo de referencia. 
De esta forma se puede garantizar que los resultados que finalmente alcanzaremos son competitivos. 

\section{Objetivos}

Los objetivos principales se centrarán en el desarrollo de una metaheurística útil para aplicarse a problemas combinatorios costosos:
\begin{enumerate}
	\item Parte matemática: Revisión teórica en la que se basa el desarrollo de algoritmos para problemas de minimización. 
	Dentro del mismo, encontramos los siguientes objetivos parciales:
	\begin{itemize}
		\item Repaso de distintos algoritmos de la literatura para resolver problemas de minimización sin restricciones. 
		\item Exposición de algunos resultados interesantes a considerar sobre convergencia. 
		\item Exposición de diversos \textit{tests} estadísticos para comparar algoritmos. 
	\end{itemize}
	
	\item Parte informática: Propio diseño experimental del algoritmo en cuestión. 
	Dentro del mismo, encontramos los siguientes objetivos parciales: 
	\begin{itemize}
		\item Exposición de los algoritmos clásicos que se han usado como base.
		\item Diseño de varias modificaciones y mejoras a introducir a los algoritmos base para hacerlos más competitivos en nuestro problema. 
		\item Evaluación experimental de los distintos algoritmos ejecutándolos sobre distintos conjuntos de instancias del problema concreto con el que trabajaremos, analizando los resultados obtenidos y comparándolos entre sí.
	\end{itemize}
	
\end{enumerate}

\section{Estructura de la memoria}

Este trabajo se divide en dos partes, y estas a su vez en varios capítulos. 
Además, se incluyen tres capítulos al comienzo. 
Este ha sido el Capítulo 1, y a continuación encontraremos el Capítulo 2, dedicado a la planificación y presupuesto del proyecto; y el Capítulo 3, dedicado a dar un repaso bibliográfico de información necesaria para introducir el trabajo.

La Parte I, más vinculada con la parte matemática del trabajo, está dedicada a analizar la base teórica matemática sobre la que se construye el desarrollo de algoritmos cuyo objetivo es alcanzar una solución óptima. 
Tras la introducción del problema general de optimización a tratar y algunas definiciones importantes en el Capítulo 4, en el Capítulo 5 se presentan varios algoritmos que traten de resolver un problema de minimización sin restricciones. 
Además, en el Capítulo 6 se presentan y explican una serie de test estadístico, algunos de los cuales utilizaremos posteriormente para analizar si se producen mejoras significativas entre las distintas versiones del algoritmo que se desarrollará. 
Esta parte finaliza en el Capítulo 7, donde se desarrolla el problema concreto con el que se trabajará, aportando una definición matemática e información sobre los datos de dicho problema que se utilizarán. 

La Parte II, más vinculada con la parte informática del trabajo, está dedicada a explicar y mostrar mediante pseudocódigo los distintos algoritmos y diversas modificaciones introducidas que se han usado durante el diseño de nuestra metaheurística junto con un análisis de los resultados obtenidos. 
En el Capítulo 8 se presentan los dos algoritmos de referencia que se han usado de base para el desarrollo de un algoritmo competitivo para problemas combinatorios \textit{expensive}, mientras que en el Capítulo 9 se presentan los distintos componentes que se han ido desarrollando para aplicarlos como modificaciones sobre los algoritmos anteriores. 
Por último, en el Capítulo 10 se proporcionarán los distintos parámetros utilizados y se realizará un análisis completo del desarrollo de nuestro algoritmo justificando por qué cada modificación ha sido introducida.

Finalmente, en el Capítulo 12 se encuentran las conclusiones que se han obtenido tras realizar todo el trabajo. 
Adicionalmente, se incluirá la bibliografía y un apéndice donde se encontrarán todas las tablas de resultados de los distintos algoritmos que se han probado en la parte experimental de este proyecto.
%
\chapter{Estimación y Presupuesto}

En este capítulo se detalla cómo se ha organizado el trabajo y el tiempo dedicado a cada una de ellas. 
El orden en el que se han realizado dichas tareas, queda representado en un Diagrama de Gantt () para su mejor comprensión.
Además, se realiza una estimación del presupuesto necesario para desarrollar el proyecto.

\section{Planificación}

La planificación previa de este proyecto se ha realizado siguiendo una metodología ágil. 
Es decir, la planificación se va adaptando dependiendo cómo hayan transcurrido las tareas anteriores. 
Es el modelo de planificación que más se ajusta a este tipo de trabajo, ya que, $\textit{a priori}$, se desconoce la dificultad de las tareas a realizar. 

Sin embargo, es cierto que antes de empezar el trabajo se estableció una planificación base bastante amplia para poder asegurar que se iba a finalizar el proyecto a tiempo. 

\subsection{Planificación Base}

La planificación inicial que se estableció antes de iniciar el proyecto se puede expresar mediante la información representada en la tabla \ref{Planificación Base}:

\begin{table}[h]
\begin{tabular}{|l|l|l|}
\hline
\rowcolor[HTML]{F7EAC7} 
\textbf{Resumen}                                                                                             & \textbf{Pasos}                                                                                                                             & \textbf{Duración}                                                        \\ \hline
\rowcolor[HTML]{ECF4FF} 
\cellcolor[HTML]{ECF4FF}                                                                                     & Obtener instancias del problema                                                                                                            & \multicolumn{1}{c|}{\cellcolor[HTML]{ECF4FF}}                            \\ \cline{2-2}
\rowcolor[HTML]{ECF4FF} 
\cellcolor[HTML]{ECF4FF}                                                                                     & Buscar algoritmos que lo resuelvan                                                                                                         & \multicolumn{1}{c|}{\cellcolor[HTML]{ECF4FF}}                            \\ \cline{2-2}
\rowcolor[HTML]{ECF4FF} 
\multirow{-3}{*}{\cellcolor[HTML]{ECF4FF}\begin{tabular}[c]{@{}l@{}}Investigar el\\ problema\end{tabular}}   & Buscar posibles implementaciones                                                                                                           & \multicolumn{1}{c|}{\multirow{-3}{*}{\cellcolor[HTML]{ECF4FF}Noviembre}} \\ \hline
\rowcolor[HTML]{DDFDFF} 
\cellcolor[HTML]{DDFDFF}                                                                                     & \begin{tabular}[c]{@{}l@{}}Elegir un algoritmo como referencia\\ y estudiarlo\end{tabular}                                                 & \cellcolor[HTML]{DDFDFF}                                                 \\ \cline{2-2}
\rowcolor[HTML]{DDFDFF} 
\multirow{-2}{*}{\cellcolor[HTML]{DDFDFF}\begin{tabular}[c]{@{}l@{}}Algoritmos\\ de referencia\end{tabular}} & \begin{tabular}[c]{@{}l@{}}Reducir el problema y meter \\ equilibrio\end{tabular}                                                          & \multirow{-2}{*}{\cellcolor[HTML]{DDFDFF}Enero-Febrero}                  \\ \hline
\rowcolor[HTML]{ECF4FF} 
Experimentación                                                                                              & \begin{tabular}[c]{@{}l@{}}Formas de inicialización no\\ aleatorias $\xrightarrow{}{}$ Diseño experimental\end{tabular} & 2 semanas                                                                \\ \hline
\rowcolor[HTML]{DDFDFF} 
Histórico                                                                                                    & Usar el histórico                                                                                                                          & Marzo                                                                    \\ \hline
\rowcolor[HTML]{ECF4FF} 
Memoria                                                                                                      & Escribir el informe                                                                                                                        & Mayo                                                                     \\ \hline
\end{tabular}
\caption{\label{Planificación Base}Planificación Base}
\end{table}

\subsection{Planificación Final}


\subsubsection{Tareas realizadas}

Si bien es cierto que se han realizado una gran cantidad de tareas (sobre todo distintas modificaciones sobre algoritmos), con el fin de mantener la simplicidad, se han agrupado algunas tareas que tenían funciones similares. 
Esta agrupación también es de ayuda para simplificar la estimación del tiempo que se le ha dedicado a cada una de las tareas. 
Así, las tareas realizadas se resumen en la siguiente lista:

\begin{enumerate}
	\item \textbf{Planteamiento y comprensión del problema}: Revisión del trabajo a realizar y reuniones con los tutores para proponer modificaciones y comprender mejor y aclarar todos los matices del Trabajo de Fin de Grado.
	
	\item \textbf{Búsqueda de información y lecturas}: Búsqueda y lectura comprensiva de todos los artículos y documentos necesarios para la realización del proyecto.
	
	\item \textbf{Planificación del proyecto}: Planificación de algunos aspectos que usar como base, así como las tareas que eran necesarias inicialmente. 
También hace referencia a partes de reuniones con los tutores para modificar las planificaciones (añadiendo o eliminando tareas) dependiendo del progreso alcanzado y los resultados obtenidos.
	
	\item \textbf{Implementación de la propuesta inicial}: Implementación del código de los algoritmos base.
	
	\item \textbf{Adaptación de los algoritmos base}: Modificación del código de los algoritmos base para adaptarlos al problema en cuestión.
	
	\item \textbf{Modificación de los algoritmos}: Sucesivas modificaciones sobre el algoritmo base y los algoritmos que mejores resultados proporcionaban con el fin de mejorarlos aún más.
	
	\item \textbf{Obtención de resultados}: Ejecución del código para obtener todos los resultados y cambiarlos de formato para su posterior análisis.
	
	\item \textbf{Análisis de los resultados}: Interpretación de los resultados obtenidos.
	
	\item \textbf{Revisión de la parte experimental}: Una vez dada por finalizada la parte experimental, se ha hecho una revisión exhaustiva de los códigos y de los resultados obtenidos.
	
	\item \textbf{Elaboración de la memoria}: Desarrollo del informe.
	
	\item \textbf{Revisión de la memoria}: Una vez terminado el trabajo, se ha hecho una revisión exhaustiva de la memoria.
\end{enumerate}

Téngase en cuenta hay tareas que se han realizado casi simultáneamente, como serían la ``Modificación de los algoritmos'', ``Obtención de los resultados'' y ``Análisis de los resultados''. 
Esto se debe a la necesidad de saber cómo han influido las modificaciones para empezar a estudiar qué otra modificación podría ser beneficiosa. 
Por ejemplo, si se converge rápidamente a una solución, hay que estudiar por qué ha pasado y, una vez hecha la hipótesis, estudiar qué se podría modificar para que no suceda.

Una estimación del tiempo (en horas) dedicado a cada tarea se puede encontrar en la tabla \ref{Tiempo_Dedicado}.

\subsubsection{Tiempo dedicado}

\begin{table}[H]
\begin{tabular}{|l|c|}
\hline
\rowcolor[HTML]{F7EAC7} 
\textbf{Actividad}                       & \multicolumn{1}{l|}{\cellcolor[HTML]{F7EAC7}\textbf{Duración (horas)}} \\ \hline
\rowcolor[HTML]{ECF4FF} 
Planteamiento y comprensión del problema & 20                                                                     \\ \hline
\rowcolor[HTML]{DDFDFF} 
Búsqueda de información y lecturas       & 15                                                                     \\ \hline
\rowcolor[HTML]{ECF4FF} 
Planificación del proyecto               & 5                                                                      \\ \hline
\rowcolor[HTML]{DDFDFF} 
Implementación de la propuesta inicial   & 10                                                                     \\ \hline
\rowcolor[HTML]{ECF4FF} 
Adaptación de los algoritmos bases       & 5                                                                      \\ \hline
\rowcolor[HTML]{DDFDFF} 
Modificación de los algoritmos           & 200                                                                    \\ \hline
\rowcolor[HTML]{ECF4FF} 
Obtención de resultados                  & 50                                                                     \\ \hline
\rowcolor[HTML]{DDFDFF} 
Análisis de los resultados               & 10                                                                     \\ \hline
\rowcolor[HTML]{ECF4FF} 
Revisión de la parte experimental        & 5                                                                     \\ \hline
\rowcolor[HTML]{DDFDFF} 
Elaboración de la memoria                & 170                                                                    \\ \hline
\rowcolor[HTML]{ECF4FF} 
Revisión de la memoria                   & 10                                                                     \\ \hline
\rowcolor[HTML]{F7EAC7} 
\textbf{Total}                           & \cellcolor[HTML]{FCE6AB}500                                            \\ \hline
\end{tabular}
\caption{\label{Tiempo_Dedicado}Tiempo dedicado}
\end{table}

\section{Presupuesto}

Si quisiéramos valorar económicamente el proyecto, tenemos que tener en cuenta dos aspectos fundamentales: el precio de la mano de obra y el de cómputo como si tuviésemos que pagarlo. 
Sin embargo

El precio de la mano de obra son 25\texteuro\xspace la hora. 
El ordenador portátil usado para la realización de las ejecuciones ha sido un Asus Tuf Gaming A15 FA506IU-HN278 con un procesador AMD\textregistered\xspace Ryzen\texttrademark\xspace 7 4800H APU
%https://www.pccomponentes.com/asus-tuf-gaming-a15-fa506iu-hn278-amd-ryzen-7-4800h-apu-16gb-1tb-ssd-gtx-1660ti-156
y 16GB (8GB$\times$2) de RAM.

Por lo tanto, el \textbf{presupuesto del proyecto queda fijado en 12500\texteuro\xspace}, a razón de 25\texteuro\xspace por cada hora de trabajo dedicada al proyecto.
%
\chapter{Repaso Bibliográfico}

En esta sección se lleva a cabo una revisión bibliográfica sobre el tema en el que hemos centrado el proyecto. 
Buscamos con ello, llevar a cabo un pequeño recordatorio para poder entender los distintos ámbitos que más se han tratado y centrado los estudios en cuanto al diseño de metaheurísticas, las cuales se usan para resolver problemas cada vez más complejos.  
Posteriormente utilizaremos estos conocimientos para nuestro propio diseño de una metaheurística útil para problemas combinatorios \textit{expensive}.

% Cada frase se desarrolla en un párrafo, con una o varias referencias

\section{Contexto Bibliográfico}

% Párrafo sobre las metaheurísticas, con referencias [evocomp, EABook]

La computación evolutiva (\textit{evolutionary computation}, EC) es un área de la ciencia de la computación que usa ideas de la evolución biológica para resolver problemas computacionales. 
La evolución es, en efecto, un método de búsqueda entre un número enorme de posibilidades de ``soluciones'' que permitan a los organismos sobrevivir y reproducirse en sus ambientes. 
También se puede ver la evolución como un método de adaptación a un entorno cambiante. 
En \parencite{backEvolutionaryComputationOverview1996} nos podemos encontrar con un resumen del desarrollo de la computación evolutiva en el tiempo junto con aplicaciones reales de algoritmos evolutivos (tanto comerciales como científicas), como podría ser el uso de programación genética para mejorar estrategias óptimas de recolección. 
Si consideramos la computación evolutiva como un medio para encontrar buenas soluciones (aunque no sean las óptimas) dado un problema de optimización, es natural considerar que su hibridación con métodos de optimización existentes resultará en mejorar su rendimiento al explotar sus ventajas. 
Tales métodos de optimización se refieren desde algoritmos exactos estudiados en programación matemática \parencite{islamMATHEMATICALPROGRAMMING2020} hasta algoritmos heurísticos hechos a medida para unos problemas dados. 
Los llamados algoritmos metaheurísticos también tienen un objetivo similar, y pueden ser combinados con EC, incluso si ambos enfoques suelen competir entre si. 
En el libro \parencite{michalewiczHandbookEvolutionaryComputation1997} se presentan varias posibilidades de combinar ECs con métodos de optimización, poniendo énfasis en la optimización de problemas combinatorios, tales como métodos \textit{greedy}, construcciones heurísticas de soluciones factibles, programación dinámica, etc. 


% Introducir los AGs con su referencia. Dentro del AG indicar los operadores comunes con su referencia (Nam)

Una de las versiones de algoritmos evolutivos más utilizadas son los algoritmos genéticos (AG), que será en el que nos centremos ya que supondrá ser un algoritmo base para el desarrollo de este proyecto. 
Los algoritmos genéticos son algoritmos basados en poblaciones que se pueden describir como la combinación de dos procesos: la generación de elementos del espacio de búsqueda (recombinación o mutación de la población) y la actualización (selección y redimensionamiento) para producir nuevas soluciones basadas en el conjunto de puntos que se crean junto con los de la anterior población \parencite{smithOperatorParameterAdaptation1997} \parencite{tuRobustStochasticGenetic2004}. 
En \parencite{backEvolutionaryComputationOverview1996} se presenta una lista de componentes para la versión más simple de un AG, que se puede resumir en:
\begin{itemize}
	\item Una población de soluciones candidatas a un problema dado, cada una codificada de acuerdo a un esquema de representación elegido. 
	\item Una función \textit{fitness} que asigna un valor numérico a cada cromosoma (solución) de una población para medir su calidad como candidato a solución del problema.
	\item Un conjunto de operadores que se aplicarán a los cromosomas para crear una nueva población. 
Estos suelen incluir:
	\begin{itemize}
		\item Operador de Cruce: Dos cromosomas padres recombinan sus genes para producir una o más soluciones hijas. 
		\item Mutación: Uno o varios genes de una solución se modifican de forma aleatoria. 
	\end{itemize}
\end{itemize}
Una explicación más completa y detallada de lo relativo a AGs se puede encontrar en \parencite{reevesGeneticAlgorithms2010}. 
En dicho capítulo también presentan algunos artículos y libros donde encontrar aplicaciones de AGs exitosas para la optimización de problemas combinatorios, entre ellas se destaca \parencite{reevesFeatureArticleGenetic1997}, la cual lista algunas de las referencias más útiles y accesibles que podrían ser de interés para gente experimentando con metaheurísticas, como podrían ser el problema del viajante de comercio, problemas relacionados con grafos, problema de la mochila en forma binaria, etc. 

% Introducir el CHC con su referencia, y algún artículo de aplicación.

% Párrafo sobre problemas expensives, incluyendo referencia de repaso bibliográfico que te envié

% Indicar que hay problemas combinatorios que no están cubiertos, citar trabajo de neuroevolución, y concreto mío, que hace muy pocas evaluaciones y aún así tarda demasiado tiempo

% Referencias sobre el problema, ya tenías alguna, busca más y sobre su interés.

%Describir posibles opciones a considerar
%  - Uso de parámetros adaptativos y auto-adaptativos [referencia], guiándose no solo por mejor actual, si no por histórico.
%  - Modificación del operador de cruce [referencia]
%  - Describir que usarás la guía de buenas prácticas [referencia]
%
\chapter{Contexto matemático}

Los algoritmos usan operadores matemáticos con el objetivo de alcanzar una solución a los problemas a los que son aplicados y también es necesario obtener el óptimo. 
La prueba de este hecho puede ser realizada gracias a la teoría de optimización del Análisis Numérico. 
Nos centraremos en esta teoría, aportando resultados que nos permitan saber si un punto es el óptimo de una función y su relación con los algoritmos básicos de optimización global. 

La programación no lineal es un área de las matemáticas aplicadas que involucra problemas de optimización cuando las funciones son no lineales. 
Nuestro objetivo es introducir este problema y revisar las condiciones generales de optimalidad, que son la base de muchos algoritmos por sus soluciones. 
Tras esto, aportaremos numerosas nociones relativas al rendimiento de los algoritmos en términos de convergencia, orden de convergencia y comportamiento numérico.

Cabe mencionar que aunque el problema abordado en este trabajo implica la maximización del valor de la solución, se utilizará la notación convencional y usual para tratar este tipo de problemas, es decir, la minimización del valor de la solución. 
Esto se debe a que, al fin y al cabo, son problemas equivalentes. 
Una forma sencilla de transformar un problema de maximización en uno de minimización es cambiar el símbolo de los valores. 
Por ello, en tanto que ambos problemas son equivalentes, se mantendrá la notación de minimizar la función objetivo.

\section{Definición del problema}

En primer lugar, daremos una definición a nuestro problema.

\begin{definicion}
Consideramos el problema de calcular el valor de un vector de variables de decisión $x\in\mathbb{R}^n$ que minimiza la función objetivo $f:\mathbb{R}^n\xrightarrow{}{}\mathbb{R}$ donde $x$ pertenece a un conjunto factible de soluciones $\mathcal{F}\in\mathbb{R}^n$. 
Consideramos el siguiente problema:
\begin{equation}
\min_{x\in\mathcal{F}}f(x)
\label{eq:4.1}
\end{equation}
\end{definicion}

\textbf{Nota}: Llamaremos \textbf{conjunto factible} al espacio de soluciones, es decir, al conjunto de todos los puntos posibles de un problema de optimización que satisface las restricciones del problema. 


Ahora, presentamos dos casos de ello:
\begin{itemize}
	\item El conjunto factible de soluciones $\mathcal{F}$ es todo el espacio $\mathbb{R}^n$. 
En este caso, el problema es el siguiente:
\begin{equation}
\min_{x\in\mathbb{R}^n}f(x)
\label{eq:4.2}
\end{equation}
Diremos que el problema \ref{eq:4.1} es no restringido. 
De forma general, el problema \ref{eq:4.1} es no restringido si $\mathcal{F}$ es un conjunto abierto.

	\item El conjunto factible de soluciones está descrito por restricciones de desigualdad y/o igualdad en las variables de decisión:
\begin{equation}
\mathcal{F} = \{x\in\mathbb{R}^n : g_i(x)\leq 0, i = 1,...,p; h_j(x)=0,j=1,...,m\}
\label{eq:4.3}
\end{equation}
Entonces, el problema \ref{eq:4.1} se convierte en:
\begin{equation}
 \begin{matrix}
  min_{x\in\mathbb{R}^n}f(x)\\
  g(x) \leq 0\\
  h(x) = 0
 \end{matrix}
\label{eq:4.4}
\end{equation}
donde $g:\mathbb{R}^n\xrightarrow{}{}\mathbb{R}^p$ y $h:\mathbb{R}^n\xrightarrow{}{}\mathbb{R}^m$. 
En este caso, diremos que el problema es restringido. 

El problema \ref{eq:4.1} es no lineal cuando al menos una de las funciones del problema es no lineal, es decir, $f$,$g_i$, $i=1,...,p$, $h_j$, $j=1,...,m$ es no lineal en $x$.

\end{itemize}

Normalmente, asumimos que en un  problema del tipo \ref{eq:4.4} el número de condiciones de igualdad, $m$, es menor que el número de variables, $n$; en otro caso, el conjunto factible de soluciones será el vacío, a no ser que haya dependencia en las restricciones. 
Si solo hay condiciones de igualdad, el problema se llamará ``\textbf{problema no lineal con restricciones de igualdad}''. 
Equivalentemente, se tiene el caso de que solo aparezcan condiciones de desigualdad. 

En lo siguiente asumiremos que nuestras funciones $f$,$g$,$h$ son diferenciables y continuas en $\mathbb{R}^n$. 
Además, cuando $f$ sea una función convexa y el conjunto $\mathcal{F}$ sea también convexo, al problema se le llamará ``\textbf{problema convexo no lineal}''. 
Particularmente, $\mathcal{F}$ es convexo si las funciones que nos aportan las restricciones de desigualdad son convexas y las funciones de las restricciones de igualdad son afines.

La convexidad nos permite añadir estructura a estos problemas y poder explotarlo desde un punto de vista teórico y computacional, esto se debe a que si $f$ es una función convexa cuadrática y $g$,$h$ son afines, entonces tenemos que tratar con un problema de programación cuadrática. 
Sin embargo, nos centraremos en problemas no lineales generales, sin asumir convexidad.

\begin{definicion}
Un punto un $x^*\in\mathcal{F}$ es un \textbf{solución global} de \ref{eq:4.1} si $f(x^*)\leq f(x)$, $\forall x\in\mathcal{F}$. El punto es una \textbf{solución global estricta} si $f(x^*) < f(x)$, $\forall x\in\mathcal{F}$, $x\neq x^*$.
\end{definicion}

La existencia de soluciones globales se debe a la compacidad de $\mathcal{F}$, en relación con el teorema de Weierstrass:

\begin{teorema}[Teorema de Weierstrass]
Sea $a,b\in\mathbb{R}$ con $a<b$ y sea $f:[a,b]\xrightarrow{}{}\mathbb{R}$ una función continua. 
Entonces, el intervalo $f([a,b])$ es cerrado y acotado.
\end{teorema}

Una consecuencia directa para los problemas sin restricciones es que una solución global existe si el conjunto $\mathcal{L}^\alpha = \{x\in\mathbb{R}^n\leq\alpha\}$ es compacto para un $\alpha$ finito. 

\begin{definicion}
Un punto $x^*\in\mathcal{F}$ es una \textbf{solución local} del problema del problema \ref{eq:4.1} si existe $x^*$ en un vecindario abierto $\mathcal{B}_{x^*}$ de $x^*$ tal que $f(x^*)\leq f(x)$, $\forall x\in \mathcal{F}\bigcap\mathcal{B}_{x^*}$. 
Además, es una \textbf{solución local estricta} si $f(x^*)<f(x)$, $\forall x\in\mathcal{F}\bigcap\mathcal{B}_{x^*}$, $x\neq x^*$.
\end{definicion}

Determinar una solución global a un problema de este tipo es normalmente una tarea complicada. 
Los algoritmos que resuelven este tipo de problemas se usan para alcanzar óptimos locales. 
Incluso en aplicaciones prácticas obtener este tipo de soluciones puede ser también algo bueno. 

Ahora introduciremos \textbf{notación}:
\begin{itemize}
	\item Dado un vector $y\in\mathbb{R}^n$, definimos su \textbf{traspuesta} por $y'$. 
	Esta definición puede ser extendida a las matrices $A\in\mathbb{R}^n\times\mathbb{R}^n$. 
	\item Dada una función $h:\mathbb{R}^n\xrightarrow{}{}\mathbb{R}$, denotamos el vector \textbf{gradiente} por $\nabla h(x) = \left( \dfrac{\partial f(x)}{\partial x_1}, ... , \dfrac{\partial f(x)}{\partial x_n} \right)$ y la matriz \textbf{Hessiana} se denota por
	\begin{equation*}
	\nabla^2h(x) =
 \begin{pmatrix}
  \dfrac{\partial^2f(x)}{\partial x_1^2} & \dfrac{\partial^2f(x)}{\partial x_1 \partial x_2} & \cdots & \dfrac{\partial^2f(x)}{\partial x_1 \partial x_n} \\
  \dfrac{\partial^2f(x)}{\partial x_2 \partial x_1} & \dfrac{\partial^2f(x)}{\partial x_2^2} & \cdots & \dfrac{\partial^2f(x)}{\partial x_2 \partial x_n} \\
  \vdots & \vdots & \ddots & \vdots \\
  \dfrac{\partial^2f(x)}{\partial x_n \partial x_1} & \dfrac{\partial^2f(x)}{\partial x_n \partial x_2} & \cdots & \dfrac{\partial^2f(x)}{\partial x_n^2}
 \end{pmatrix}
\label{eq:4.4}
\end{equation*}
	\item Dada una función vector $w:\mathbb{R}^n\xrightarrow{}{}\mathbb{R}^q$, denotamos la matriz de $n\times q$ cuyas columnas son $\nabla w_j(x)$, $j=1,...,q$ con $\nabla w(x)$.
	
	\item Sea $y$ un vector, $y \in \mathbb{R}^q$, denotamos la \textbf{norma Euclídea} por $||y||$. 
	Supongamos que sus componentes son $y_i$, $i=1,...,q$ y sea $A$ una matriz cuyas columnas sean $a_j$, $j=1,...,q$. 
	Sea $\mathcal{K} \subset \{1,...,q\}$ un subconjunto de índices. 
	Denotamos por $y_{\mathcal{K}}$ al subvector de $y$ con componentes $y_i$ tales que $i\in\mathcal{K}$ y por $A_{\mathcal{K}}$ a la submatriz de $A$ compuesta por las columnas $a_j$ con $j\in\mathcal{K}$.
\end{itemize}

\section{Condición de Optimalidad}

Nuestras soluciones locales deben satisfacer algunas condiciones de optimalidad necesarias. 
Si nos referimos a problemas como \ref{eq:4.2}, tenemos uno de los resultados más conocidos del Análisis Numérico clásico:

\begin{proposicion}
Sea $x^*$ una solución local al problema \ref{eq:4.2}, entonces
\begin{equation}
\nabla f(x^*) = 0
\label{eq:4.5}
\end{equation}
Además, si $f$ es continuamente 2-diferenciable, entonces
\begin{equation}
y'\nabla^2f(x^*)y \geq 0, \forall y \in \mathbb{R}^n
\label{eq:4.6}
\end{equation}
\end{proposicion}

\textbf{Nota}: Diremos que una función es continuamente $i$-diferenciable si es diferenciable $i$ veces y dichas diferenciales son continuas.

Si tenemos problemas como \ref{eq:4.4}, la mayoría de las condiciones necesarias de optimalidad usadas en el desarrollo de los algoritmos asumen que una solución local debe satisfacer algunas de estas condiciones para evitar alcanzar casos degenerados. 
Estas condiciones se suelen llamar ``\textbf{calificaciones de restricción}'' y, entre ellas, la más simple y más comúnmente usada es el requisito de restricción de independencia lineal:

\begin{definicion}
Sea $\hat{x} \in\mathcal{F}$. 
Decimos que la restricción de desigualdad $g_i$ está \textbf{activa} en $\hat{x}$ si $g_i(\hat{x}) = 0$. 
Denotamos por $\mathcal{F}_a(\hat{x})$ al conjunto de índices de las restricciones de desigualdad activas en $\hat{x}$:
\begin{equation}
\mathcal{F}_a(\hat{x}) = \{i \in\{1,...,p\} : g_i(\hat{x})=0\}
\label{eq:4.7}
\end{equation}
\end{definicion}

Nótese que en la definición anterior podemos concluir que las restricciones de igualdad $h_j$ son activas en $\hat{x}$. 
La restricción de independencia lineal se satisface si el gradiente de las restricciones activas son linealmente independientes.

Bajo las condiciones de independencia lineal asumidas, los problemas de restricciones como \ref{eq:4.4} pueden resolverse usando la función Lagrangiana generalizada:
\begin{equation}
L(x,\lambda,\mu) = f(x) +\lambda'g(x) + \mu'h(x)
\label{eq:4.8}
\end{equation}
donde $\lambda\in\mathbb{R}^p$, $\mu\in\mathbb{R}^m$ son multiplicadores de Lagrange. 
El siguiente resultado nos dan información sobre la existencia de estos multiplicadores. 

\begin{proposicion}
Sea $x^*$ una solución local de \ref{eq:4.4} y supongamos que la independencia lineal se satisface en $x^*$. 
Entonces, los multiplicadores de Lagrange $\lambda^*\geq 0$, $\mu^*$ existe de tal forma que:
\begin{equation}
\nabla_xL(x^*,\lambda^*,\mu^*) = 0
\label{eq:4.9}
\end{equation}
y
\begin{equation*}
\lambda'^*g(x^*) = 0
\end{equation*}
Además, si $f,g,h$ son continuamente 2-diferenciables, entonces:
\begin{equation*}
y'\nabla_x^2L(x^*,\lambda^*,\mu^*)y \geq 0, \forall y\in\mathcal{N}(x^*)
\end{equation*}
donde
\begin{equation*}
\mathcal{N}(x^*) = \{y\in\mathbb{R}^n : \nabla g'_{\mathcal{F}_a}y=0; \nabla h(x^*)'y = 0\}
\end{equation*}
\end{proposicion}

Si $x\in\mathcal{F}$ satisface una condición de optimalidad suficiente, entonces dicho punto es una solución local al problema \ref{eq:4.1}. 
Para problemas generales, las condiciones de optimalidad suficientes pueden ser establecidas bajo la asunción de que las funciones problemas son continuamente 2-diferenciables, así que tenemos condiciones de suficiencia de segundo orden. 
Estas condiciones cambian si el problema a tratar es \ref{eq:4.2} o \ref{eq:4.4}. 
Por lo tanto, los siguientes resultados mostrarán cuándo $x^*$ son óptimos locales en ambos casos.

\begin{proposicion}
Asumimos que $x^*$ satisface la condición \ref{eq:4.5}. 
También asumiremos que 
\begin{equation*}
y'\nabla^2f(x^*)y > 0, \forall y\in\mathbb{R}^n, y \neq 0
\end{equation*}
es decir, se asume que $\nabla^2f(x^*)$ es definida positiva. 
Entonces, $x^*$ es una solución local estricta del problema \ref{eq:4.2}.
\end{proposicion}

\begin{proposicion}
Asumimos que $x^*\in\mathcal{F}$ y $\lambda^*,\mu^*$ satisfacen las condiciones de \ref{eq:4.9}. 
Asumimos también que 
\begin{equation}
y'\nabla^2_xL(x^*,\lambda^*,\mu^*)y > 0, \forall y\in\mathcal{P}(x^*), y\neq 0
\label{eq:4.10}
\end{equation}
donde
\begin{equation*}
\mathcal{P}(x^*) = \{y\in\mathbb{R}^n : \nabla g'_{\mathcal{F}_a} y\leq 0, \nabla h(x^*)'y = 0; \nabla g_i(x^*)'y = 0, i \in\mathcal{F}_a(x^*), \lambda_i^* \} 
\end{equation*}
Entonces, $x^*$ es una solución local estricta de \ref{eq:4.4}.
\end{proposicion}

Nótese también que $\mathcal{N}(x^*) \subseteq\mathcal{P}(x^*)$  y la igualdad se da si $\lambda_i^* >0, \forall i\in\mathcal{F}_a(x^*)$.

Una característica importante y principal de los problemas convexos es que una solución global (estricta) del problema es también una solución local (estricta). 
Además, cuando $f$ es (estrictamente) convexa, las funciones $g_i$ son también convexas y las $h_j$ son afines, entonces las condiciones de optimalidad necesarias dadas en términos de primeras derivadas parciales son suficientes para que un punto $x^*$ sea una solución global (estricta).

Las condiciones de optimalidad son esenciales para problemas no lineales. 
Si conocemos la existencia de un óptimo global, entonces el método más común de obtenerlo es el siguiente:
\begin{enumerate}
	\item Encontrar todos los puntos que satisfacen las condiciones necesarias de primer orden.
	\item Tomas el óptimo global como el punto con el valor más bajo dado por la función objetivo
	\item Si la función problema es 2-diferenciable, entonces comprueba la condición necesaria de segundo orden y elimina los puntos que no la satisfagan.
	\item Para el resto de puntos comprobaremos la condición de suficiencia de segundo orden para encontrar el mínimo local.
\end{enumerate}

Tenemos que destacar que este método no funciona en casos prácticos excepto por casos simples, esto se debe a que tenemos que calcular la solución de un sistema de ecuaciones dado por $\nabla f(x)=0$ y este sistema es normalmente no trivial. 
Entonces, ¿dónde son importantes estas condiciones? 
Las condiciones de optimalidad son importantes en el desarrollo y análisis de algoritmos.

Un algoritmo que intenta resolver un problema dado por \ref{eq:4.1} genera una secuencia de soluciones factibles $x^k, k=0,1,...$ y normalmente finaliza cuando se satisface un criterio de parada. 
Este criterio se suele basar en la satisfacción de condiciones de optimalidad necesarias dentro de una tolerancia prefijada. 
Adicionalmente, estas condiciones normalmente sugieren cómo mejorar la solución actual, con lo que la siguiente debería encontrarse más cercana al óptimo.

Así, las condiciones de optimalidad necesarias nos dan la base para el análisis de convergencia de algoritmos. 
Por lo tanto, las condiciones de suficiencia juegan un papel importante en el análisis del orden de convergencia.

\section{Rendimiento de los algoritmos}

\subsection{Convergencia y Orden de Convergencia}

\begin{definicion}
Sea $\Omega\subset\mathcal{F}$ el subconjunto de puntos que satisfacen las condiciones necesarias de optimalidad del problema \ref{eq:4.1}. 
Un algoritmo se detiene cuando un punto $x^*\in\Omega$ es calculado. 
Por lo tanto, a este conjunto se le llamará \textbf{conjunto objetivo}.
\end{definicion}

Un ejemplo de este conjunto para el problema sin restricciones podría ser $\Omega = \{x \in\mathbb{R}^n: \nabla f(x) = 0\}$; mientras que para el problema restringido este conjunto será el conjunto de puntos que satisfacen \ref{eq:4.9}.

\begin{definicion}
Sea $x^k, k=0,1,...$ la secuencia de puntos generados por un algoritmo. 
Entonces, el algoritmo es \textbf{globalmente convergente} si un punto límite $x^*$ de $x^k$ existe tal que $x^k\in\Omega$ para cualquier punto de inicio $x^0\in\mathbb{R}^n$.


Diremos que el algoritmo es \textbf{localmente convergente} si la existencia de $x^*\in\Omega$ solo puede ser establecida si el punto inicial, $x^0$, pertenece a un vecindario de $\Omega$.
\end{definicion}

La definición de convergencia establecida anteriormente es la más débil que asegura que un punto $x^k$ arbitrariamente cercano a $\Omega$ puede ser obtenido con un $k$ suficientemente grande.

En el caso de no tener restricciones, esto implica
\begin{equation*}
\lim_{k\xrightarrow{}{}\infty} ||\nabla f(x^k)|| = 0
\end{equation*}

El requerimiento más fuerte de convergencia nos muestra que la secuencia $x^k$ converge a un punto $x^*\in\Omega$.

Ahora mostraremos cómo se puede definir el orden de convergencia de un algoritmo. 
Así, podemos asumir por simplicidad que los algoritmos generan una secuencia $x^k$ que converge a un punto $x^*\in\Omega$. 
El concepto más empleado en términos de convergencia es el Q-orden de convergencia, el cual considera el cociente entre dos iteraciones sucesivas dado por
\begin{equation*}
\dfrac{||x^{k+1} - x^*||}{||x^k - x^*||}
\end{equation*}

Entonces, podemos definir el siguiente tipo u orden de convergencia.

\begin{definicion}
El orden de convergencia es Q-lineal si existe una constante $r\in (0,1)$ tal que
\begin{equation*}
\dfrac{||x^{k+1}-x^*||}{||x^k-x^*||} \leq r,
\end{equation*}
para cualquier $k$ suficientemente grande.
\end{definicion}

\begin{definicion}
El orden de convergencia es Q-superlineal si
\begin{equation*}
\lim_{k\xrightarrow{}{}\infty}\dfrac{||x^{k+1}-x^*||}{||x^k-x^*||} = 0
\end{equation*}
\end{definicion}

\begin{definicion}
El orden de convergencia es Q-cuadrático si
\begin{equation*}
\dfrac{||x^{k+1}-x^*||}{||x^k-x^*||^2} \leq R
\end{equation*}
para cualquier k suficientemente grande y donde $R>0$ es una constante.
\end{definicion}

\subsection{Comportamiento numérico}

A pesar del rendimiento teórico que puedan tener los algoritmos, otro aspecto importante es el comportamiento práctico. 
En efecto, si tenemos una gran cantidad de operaciones algebraicas por operación, es posible superar una tasa de convergencia rápida. 
Tenemos numerosas medidas para evaluar el comportamiento numérico. 
Sin embargo, si la carga computacional (operaciones algebraicas por iteración) no es despreciable, entonces podemos usar algunas medidas dadas por el número de iteraciones, número de evaluaciones de funciones objetivo, etc.

Medir el rendimiento de los algoritmos es importante para problemas no lineales de gran escala. 
El término ``gran escala'' depende de la máquina que se encargue de los datos, pero ese tipo de problema son normalmente problemas sin restricciones que verifican que $n\geq 1000$, donde $n$ es el número de variables. 
Sin embargo, un problema con restricciones se considerará ser un problema de gran escala cuando el número de variables es $n\geq 100$ y cuando la suma de las condiciones es 100 o mayor. 
Uno de los problemas más importantes es el trasladar algoritmos eficientes en problemas a pequeña escala a problemas de gran escala.

\section{Programación no lineal sin restricciones}

En esta sección consideraremos algoritmos que traten de resolver el siguiente problema sin restricciones
\begin{equation}
\min_{x\in\mathbb{R}^n}f(x) 
\label{eq:4.11}
\end{equation}
donde $x\in\mathbb{R}^n$ es el vector de variables de decisión y $f:\mathbb{R}^n\xrightarrow{}{}\mathbb{R}$ es la función objetivo. 
Es lógico pensar que si somos capaces de resolver este problema, el procedimiento para ello puede ser ajustado a problemas con restricciones porque solo necesitaremos un conjunto abierto factible de soluciones $\mathcal{F}$, y un punto inicial $x^0\in\mathcal{F}$.

Asumimos por simplicidad que nuestra función $f$ es continuamente 2-diferenciable en $\mathbb{R}^n$. 
Además, asumiremos lo siguiente para asegurarnos de la existencia de una solución de la ecuación \ref{eq:4.11}: $\mathcal{L}^0 = \{x\in\mathbb{R}:f(x)\leq f(x^0)\}$ es compacto para algún $x^0\in\mathbb{R}^n$.

\subsection{Algoritmos de Optimización sin restricciones}

Ahora presentaremos algunos modelos de algoritmos que serán usados para resolver los problemas previos. 
Estos tipos de algoritmos están caracterizados por generar una secuencia de puntos, $\{x^k\}$, empezando por un punto inicial $x^0$, usando la siguiente iteración
\begin{equation}
x^{k+1} = x^k + \alpha^kd^k
\label{eq:4.12}
\end{equation}
donde $d^k$ es la dirección de búsqueda y $\alpha^k$ es el tamaño de paso junto con $d^k$. 
En este método tenemos dos parámetros a modificar: la dirección de búsqueda y el tamaño del paso, por lo tanto, dependiendo de como variamos estos datos, obtendremos diferentes métodos y esto afectará a las propiedades de convergencia. 
El tamaño de paso afecta a la convergencia global, mientras que la dirección de búsqueda afecta a la convergencia local. 

El siguiente resultado nos da una relación entre convergencia y dirección de búsqueda:

\begin{proposicion}
Sea $\{x^k\}$ la secuencia generada por \ref{eq:4.12}. 
También asumimos:
\begin{enumerate}
	\item $d^k \neq 0$ si $\nabla f(x^k)\neq 0$.
	\item $\forall k$ tenemos $f(x^{k+1}) \leq f(x^k)$.
	\item \begin{equation}
	\lim_{k\xrightarrow{}{}\infty}\dfrac{\nabla f(x^k)'d^k}{||d^k||} = 0
	\label{eq:4.13}
	\end{equation}
	\item $\forall k$ con $d^k\neq 0$, tenemos $\dfrac{|\nabla f(x^k)'d^k|}{||d^k||} \geq c||\nabla f(x^k)||$ con $c>0$.
\end{enumerate}
Entonces, tenemos que, o bien, existe $\hat{k}$ tal que $x^{\hat{k}}\in\mathcal{L}^0$ y $\nabla f(x^{\hat{k}}) = 0$, o bien, se genera una secuencia infinita tal que:
\begin{enumerate}
	\item $\{x^k\in\mathcal{L}^0\}$.
	\item $\{f(x^k)\}$ converge.
	\item \begin{equation}
	\lim_{k\xrightarrow{}{}\infty} ||\nabla f(x^k))|| = 0
	\label{eq:4.14}
	\end{equation}
\end{enumerate}
\end{proposicion}

La tercera condición implica que solo necesitamos una subsecuencia que tenga un punto límite en $\Omega$. 
Este resultado nos da información sobre cómo es la convergencia en términos de la dirección de búsqueda.

Procedemos a describir dos métodos conocidos como algoritmos de Búsqueda Lineal y métodos basados en el gradiente.

\subsection{Algoritmos de Búsqueda Lineal}

Estos algoritmos están caracterizados por determinar el tamaño de paso $\alpha^k$ junto con la dirección de búsqueda $d^k$. 
El objetivo es elegir un tamaño de paso que asegure la convergencia de \ref{eq:4.12}. 
Una elección puede ser elegir dicho tamaño de paso tal y como se describe en la siguiente ecuación:
\begin{equation*}
\alpha^k = arg \min_\alpha f(x^k+\alpha d^k)
\end{equation*}

La ecuación anterior puede ser resumida en la siguiente idea: 
el tamaño de paso es el valor que minimiza la función objetivo junto con una dirección dada. 
Sin embargo, en esta situación es posible que el mínimo no pueda ser alcanzado porque la función necesita tener propiedades que nos permitan calcular dicho mínimo, por lo tanto, una búsqueda lineal no tiene por qué suponer la mejor solución computacionalmente hablando.

Por ello, tenemos que usar métodos de aproximación. 
Uno de ellos es calcular el gradiente de la función. 
En estos casos podemos asumir que
\begin{equation}
\nabla f(x^k)'d^k < 0, \forall k
\label{eq:4.15}
\end{equation}

Los algoritmos de búsqueda lineal más simples están proporcionados por Armijo.

\begin{algorithm}
\caption{Algoritmo de Búsqueda Lineal de Armijo}\label{alg:LocalSearch}
\begin{algorithmic}[1]
\Procedure \textsc{Busqueda\_Lineal}($\delta\in (0,1), \gamma\in (0,1/2), c\in (0,1)$)
\State Elegir $\nabla^k$ tal que\begin{equation*}
\nabla^k \geq c \dfrac{|\nabla f(x^k)'d^k|}{||d^k||^2}
\end{equation*}
\State $\alpha = \nabla^k$
\State $N \gets n$
\If{$f(x^k + \alpha d^k)\leq f(x^k) + \gamma\alpha\nabla f(x^k)'d^k$}
    \State $\alpha^k = \alpha$ y parar
\Else
    \State Establecer $\alpha = \delta\alpha$ y volver al paso anterior.
\EndIf
\EndProcedure
\end{algorithmic}
\end{algorithm}

La elección inicial de $\nabla^k$ se debe a la dirección $d^k$ porque garantizamos que, en un número finito de pasos, $\alpha^k$ tiene un valor tal que $f(x^{k+1})<f(x^k)$, por lo que las condiciones de convergencia de la proposición se encuentran satisfechas.

Esta búsqueda lineal encuentra un tamaño de paso que satisface la condición de decrecimiento suficiente de la función objetivo y, consecuentemente, el desplazamiento suficiente de la secuencia actual.

En algunos casos, necesitamos considerar que los algoritmos de búsqueda lineal no necesitan información sobre las derivadas. 
En estos casos, la condición \ref{eq:4.15} no puede ser verificada. 
Por lo tanto, la dirección $d^k$ no tiene por qué ser descendiente. 

Otra posible modificación al algoritmo propuesto podría ser sustituir la condición de parada por la siguiente, que no tiene información sobre la derivada:

\begin{equation}
f(x^k +\alpha d^k) \leq f(x^k)-\gamma\alpha^2||d^k||^2
\label{eq:4.16}
\end{equation}

\subsection{Métodos de Gradiente}

El método de gradiente o método del descenso más rápido (\textit{steepest-descent}) se considera un método básico de entre todos los algoritmos de optimización no restringidos. 
Tiene la regla básica de establecer $d^k = -\nabla f(x^k)$ en \ref{eq:4.12}. 
Solo necesita información sobre la primera derivada y es importante debido a que el coste computacional y el almacenamiento disponibles son limitados. 

Este método es un ejemplo de convergencia global, porque si usamos un algoritmo de búsqueda local adecuado, podemos establecer un resultado de convergencia global. 
El problema es su orden de convergencia, el cual es bajo y esto es la razón por la que este algoritmo no se suele usar solo. 
El esquema de este algoritmo es el siguiente:

\begin{algorithm}
\caption{Método de Gradiente}\label{alg:GradientMethod}
\begin{algorithmic}[1]
\Procedure \textsc{Gradiente}
\State Establecer $x^0\in\mathbb{R}^n$ y $k=0$
\While{$\nabla f(x^k)!=0$}
	\State Establecer $d^k=-\nabla f(x^k)$ y encuentra un tamaño de paso $\alpha^k$ usando \ref{alg:LocalSearch}
	\State Establecer $x^{k+1} = x^k - \alpha^k\nabla f(x^k)$ y $k=k+1$.
\EndWhile
\EndProcedure
\end{algorithmic}
\end{algorithm}

La elección inicial del tamaño de paso como la dada por Armijo es importante, ya que puede afectar al comportamiento del algoritmo. 
En términos de convergencia, el siguiente resultado nos asegura la convergencia sin necesitar asumir la compacidad del conjunto $\mathcal{L}^0$.

\textbf{Nota}: Sea $f:\mathbb{R}^n\xrightarrow{}{}\mathbb{R}^m$. $f$ se dice \textbf{Lipschitziana} si existe $K>0$ tal que
\begin{equation*}
||f(x)-f(y)|| \leq K||z-y||, \forall x,y\in\mathbb{R}^n
\end{equation*}

\begin{proposicion}
Si $\nabla f$ es Lipschitziana, continua y $f$ está acotada inferiormente , entonces la secuencia generada por \ref{alg:GradientMethod} satisface la tesis de \ref{eq:4.5}.
\end{proposicion}

Como conclusión, con este resultados podemos comprobar que el gradiente es bueno en términos de convergencia global. 
Sin embargo, solo podemos probar la convergencia lineal. 
Desde un punto de vista práctico, el orden de convergencia es muy pobre y depende del número de condiciones de la matriz Hessiana.

\subsection{Métodos de Gradiente Conjugado}

Este algoritmo es muy popular debido a su simplicidad y sus bajos requerimientos computacionales. 
En efecto, solo necesita saber sobre las derivadas de primer orden. 
La idea principal es que la minimización de funciones cuadráticas estrictamente convexas en $\mathbb{R}^n$ como la siguiente
\begin{equation}
f(x) = \dfrac{1}{2}x'Qx+\alpha'x
\label{eq:4.17}
\end{equation}
donde $Q$ es una matriz simétrica definida positiva, puede ser dividida en $n$ minimizaciones sobre $\mathbb{R}$. 
Esto se puede hacer utilizando $n$ direcciones, $d^0,...,d^{n-1}$ conjugadas con respecto de la matriz Hessiana $Q$. 
Junto con cada dirección, se realiza una búsqueda lineal.

El siguiente algoritmo (\ref{alg:ConjugateDirection}) muestra todo lo mencionado anteriormente.

Se debe notar que el valor de $\alpha^k$ que se utilizará en el tercer paso es el que minimiza la función $f(x^k + \alpha d^k)$. 
Además, el cociente está bien definido porque para dos direcciones distintas tenemos $(d^j)Qd^i=0$, con $i,j$ tales que $i\neq j$.

\begin{algorithm}[H]
\caption{Algoritmo de direcciones conjugadas para funciones cuadráticas}\label{alg:ConjugateDirection}
\begin{algorithmic}[1]
\Procedure \textsc{Direcciones\_Conjugadas} (direcciones Q-conjugadas $d^0,...,d^{n-1}$)
\State Establecer $x^0\in\mathbb{R}^n$ y $k=0$
\While{$\nabla f(x^k)!=0$}
	\State Establecer $\alpha^k = \dfrac{\nabla f(x^k)'d^k}{(d^k)'Qd^k}$
	\State Establecer $x^{k+1} = x^k - \alpha^k\nabla f(x^k)$ y $k=k+1$.
\EndWhile
\EndProcedure
\end{algorithmic}
\end{algorithm}

En este algoritmo, las direcciones provienen de los datos, mientras que en el algoritmo del gradiente conjugado se calculan de forma iterativa usando la siguiente regla:

\begin{equation*}
d^k= \left\{ \begin{array}{lcc}
             -\nabla f(x^k) &   si  & k = 0 \\
             \\ -\nabla f(x^k)+\beta^{k-1}d^{k-1} &  si & k\geq 1 \\
             \end{array}
   \right.
\end{equation*}

El escalar $\beta^k$ se elige para reforzar la conjugación entre las direcciones. 
La opción más común es la propuesta de Fletcher-Reeves:

\begin{equation}
\beta_{FR} = \dfrac{||\nabla f(x^{k+1})||^2}{||\nabla f(x^k)||^2}
\label{eq:4.18}
\end{equation}

Sin embargo, la fórmula de Polak-Ribiére también puede usarse:

\begin{equation}
\beta_{PR} = \dfrac{\nabla f(x^{k+1})'(\nabla f(x^{k+1})-\nabla f(x^k))}{||\nabla f(x^k)||^2}
\label{eq:4.19}
\end{equation}

Ambas fórmulas son iguales en el caso cuadrático y diferentes cuando se trate de cualquier otro caso.

El esquema del gradiente conjugado se presenta como sigue:

\begin{algorithm}[H]
\caption{Algoritmo del gradiente conjugado}\label{alg:ConjugateGradiente}
\begin{algorithmic}[1]
\Procedure \textsc{Gradiente\_Conjugado}
\State Establecer $x^0\in\mathbb{R}^n$ y $k=0$
\While{$\nabla f(x^k)!=0$}
	\State Calcular $\beta^{k-1}$ usando \ref{eq:4.18} o \ref{eq:4.19} y establecer la dirección usando \begin{equation*}
d^k= \left\{ \begin{array}{lcc}
             -\nabla f(x^k) &   si  & k = 0 \\
             \\ -\nabla f(x^k)+\beta^{k-1}d^{k-1} &  si & k\geq 1 \\
             \end{array}
   \right.
\end{equation*}
	\State Encontrar $\alpha^k$ usando un algoritmo de búsqueda lineal que satisfaga las condiciones de Wolfe.
	\State Establecer $x^{k+1} = x^k + \alpha^kd^k$ y $k=k+1$.
\EndWhile
\EndProcedure
\end{algorithmic}
\end{algorithm}

Las condiciones de Wolfe mencionadas en \ref{alg:ConjugateGradiente} son las siguientes:
\begin{equation}
f(x^k + \alpha d^k) \leq f(x^k) + \gamma\alpha\nabla f(x^k)'d^k
\label{eq:4.20}
\end{equation}
que es la misma que se usó en la Búsqueda Lineal de Armijo, con la siguiente condición siendo más fuerte
\begin{equation}
|\nabla f(x^k+\alpha d^k)'d^k| \leq \beta |\nabla f(x^k)'d^k|
\label{eq:4.21}
\end{equation}
donde $\beta\in (\gamma,1)$ y $\gamma$ se encuentra en el mismo intervalo que antes.

La forma más simple de asegurar las propiedades de convergencia global en este método es aplicando reinicios periódicos junto con dirección de descenso más rápido. 
Aún así, el reinicio puede también suceder si alguno de los términos cuadráticos se pierden y pueden causar, o bien, que el método sea ineficiente, o bien, elecciones de caminos sin sentido.

El mecanismo de reinicio se realiza cada $n$ iteraciones o si se da la siguiente condición:
\begin{equation*}
|\nabla f(x^k)'\nabla f(x^{k+1})| > \delta ||\nabla f(x^{k-1})||^2
\end{equation*}
con $0 < \delta < 1$.


\subsection{Métodos de Newton}

Este método se considera uno de los algoritmos más potentes para resolver problemas de optimización sin restricciones. 
La aproximación cuadrática de la función objetivo en un vecindario de la solución actual, $x^k$, considerada es la siguiente

\begin{equation*}
q^k(s) = \dfrac{1}{2}s'\nabla^2f(x^k)s+\nabla f(x^k)'s+f(x^k) 
\end{equation*}
donde necesitamos saber las derivadas de primer y segundo orden de la función objetivo en la iteración número $k$. 
Este algoritmo también necesita calcular una dirección, $d_N$, y en ese caso, se obtiene como un punto estacionario de la aproximación anterior y es la solución del sistema dado por
\begin{equation}
\nabla^2 f(x^k)d_N = -\nabla f(x^k)
\label{eq:4.22}
\end{equation}

Por lo tanto, la matriz $\nabla^2f(x^k)d_N$ es no singular, tiene una inversa y la dirección es $d_N = -(\nabla^2 f(x^k))^{-1}\nabla f(x^k)$.

El esquema básico algorítmico está definido por la iteración

\begin{equation}
x^{k+1} = x^k - (\nabla^2f(x^k))^{-1}\nabla f(x^k)
\label{eq:4.23}
\end{equation}

La calidad de este método se debe al hecho de que si el punto de partida $x^0$ está próximo a la solución $x^*$, entonces la secuencia de puntos generado por la ecuación anterior converge a $x^*$ de forma superlineal o cuadrática (si la Hessiana es continuamente Lipschitziana en un vecindario de la solución).

Sin embargo, este método tiene algunas desventajas. 
Una de ellas es la singularidad de la matriz $\nabla^2f$ porque en el caso de ser singular, el método no puede definirse. 
Otra desventaja está relacionada con el punto inicial, $x^0$. Puede ser tal que la secuencia generada por \ref{eq:4.23} no converge, pero puede ocurrir la convergencia a un punto máximo.

Debido a estos hechos, el método de Newton necesita algunos cambios para asegurar la convergencia global a la solución. 
Un método de Newton convergente debería generar una secuencia de puntos $\{x^k\}$ con las siguientes características:

\begin{itemize}
\item Admite un punto límite.
\item Cualquier otro punto límite pertenece a $\mathcal{L}$ y es un punto estacionario de $f$.
\item Ningún punto límite es un punto máximo de $f$.
\item Si $x^*$ es un punto límite de $\{x^k\}$ y $\nabla^2 f(x^*)$ es definida positiva, entonces la convergencia es, al menos, superlineal.
\end{itemize}

Hay dos enfoques para diseñar un método de Newton convergente globalmente: un enfoque con búsqueda lineal y un enfoque con regiones de confianza.

\subsubsection{Modificaciones de Búsqueda Lineal en el Método de Newton}

la adaptación del método a este enfoque es el control del tamaño de paso junto con $d_N$ tal que \ref{eq:4.23} se convierte en
\begin{equation}
x^{k+1} = x^k - \alpha^k[\nabla^2 f(x^k)]^{-1}\nabla f(x^k)
\label{eq:4.24}
\end{equation}
donde $\alpha^k$ se elige con una buena búsqueda local. 
Un ejemplo puede ser inicializar $\Delta^k = 1$ en \ref{alg:LocalSearch}. 
Adicionalmente a este cambio, la dirección $d_N$ puede ser perturbada para asegurar la convergencia global del algoritmo. 
La forma más fácil de realizar este cambio es usar la dirección del descenso más rápido siempre cuando $d^k_N$ no satisface alguna de las condiciones de convergencia. 

Un posible esquema de dicho algoritmo se presenta a continuación:

\begin{algorithm}[H]
\caption{Método de Newton con Búsqueda Lineal}\label{alg:LSNewton}
\begin{algorithmic}[1]
\Procedure \textsc{ABLNewton} ($c_1>0,c_2>0,p\geq 2, q\geq 3$)
\State Establecer $x^0\in\mathbb{R}^n$ y $k=0$
\While{$\nabla f(x^k)!=0$}
	\If{$\exists d_N^k$ solución de $\nabla^2 f(x^k)d^k_N = -\nabla f(x^k)$ y satisface \begin{equation*}
	\nabla f(x^k)'d^k_N\leq -c_1||\nabla f(x^k)||^q, ||d_N^k||^p\leq c_2||\nabla f(x^k)||
	\end{equation*}}
		\State Establecer la dirección $d^k = d_N^k$
	\Else
		\State $d^k = -\nabla f(x^k)$
	\EndIf
	\State Encontrar $\alpha^k$ usando \ref{alg:LocalSearch}
	\State Establecer $x^{k+1} = x^k+\alpha^kd^k$ y, después, $k=k+1$
\EndWhile
\EndProcedure
\end{algorithmic}
\end{algorithm}

En el tercer paso, se toma la dirección del descenso más rápido si $\nabla f(x^k)'d^k_N\geq 0$. 
Otra posible modificación del método de Newton podría ser aquella que toma la dirección de la curvatura negativa, es decir, $d^k = -d_N^k$. 
Esta modificación se puede hacer si las siguientes dos condiciones se cumplen:
\begin{enumerate}
	\item $|\nabla f(x^k)'d^k| \geq c_1||\nabla f(x^k)||^q$
	\item $||d^k||^p \leq c_2 ||\nabla f(x^k)||$
\end{enumerate}

Una segunda modificación es perturbar la matriz Hessiana con una matriz definida positiva $Y^k$ y ahora la solución provendría de resolver el sistema $(\nabla^2 f(x^k) + Y^k)d = -\nabla f(x^k)$.

Las modificaciones comunes de los métodos de Newton se basan en el decremento monótono de los valores de la función objetivo. 
Con estos cambios la región de convergencia del método puede aumentar más de lo esperado, pero una secuencia convergente en este conjunto puede no ser una secuencia monótonamente descendente de los valores de la función objetivo. 

La siguiente modificación está basada en las condiciones de búsqueda lineal y mantiene la propiedad de convergencia global. 
En esta modificación, usaremos una regla no monótona. 
Por lo tanto, el método obtenido es el siguiente:

\begin{algorithm}[H]
\caption{Método de Newton no monótono}\label{alg:NonMonotoneNewton}
\begin{algorithmic}[1]
\Procedure \textsc{NewtonNM} ($c_1>0,c_2>0,p\geq 2, q\geq 3$ y $M$ entero)
\State Establecer $x^0\in\mathbb{R}^n$ y $k=0$
\While{$\nabla f(x^k)!=0$}
	\If{$\exists d_N^k$ solución de $\nabla^2 f(x^k)d^k_N = -\nabla f(x^k)$ y satisface \begin{equation*}
	\nabla f(x^k)'d^k_N\leq -c_1||\nabla f(x^k)||^q, ||d_N^k||^p\leq c_2||\nabla f(x^k)||
	\end{equation*}}
		\State Establecer la dirección $d^k = d_N^k$
	\Else
		\State $d^k = -\nabla f(x^k)$
	\EndIf
	\State Encontrar $\alpha^k$ usando \ref{alg:LocalSearch}, tal que \begin{equation*}
	f(x^k+\alpha d^k) \leq \max_{0\leq j\leq J}\{f(x^{k-j})\}+\gamma\alpha\nabla f(x^k)'d^k
	\end{equation*}
	con $J=m(k)$
	\State Establecer $x^{k+1} = x^k+\alpha^kd^k$ y, después, $k=k+1$
	\State Establecer $m(k) = \min\{m(k-1)+1,M\}$
\EndWhile
\EndProcedure
\end{algorithmic}
\end{algorithm}

Los métodos de Búsqueda Lineal estudiados hasta ahora convergen a puntos satisfaciendo solo las condiciones de optimalidad de primer orden necesarias de la ecuación \ref{eq:4.11}. 
Esto se debe a que el método de Newton no explota toda la información obtenida en la segunda derivada. 
Es posible obtener una convergencia más fuerte si usamos el par de direcciones $(d^k,s^k)$ y una búsqueda curvilinear, es decir,
\begin{equation}
x^{k+1} = x^k + \alpha^kd^k+(\alpha^k)^{\frac{1}{2}}s^k
\label{eq:4.25}
\end{equation}
donde $d^k$ es una dirección del método de Newton y $s^k$ es una dirección que incluye información de curvatura negativa con respecto a $\nabla^2 f(x^k)$. 
Con esta idea y algunas modificaciones a la Búsqueda Lineal de Armijo, se pueden crear algoritmos globalmente convergente que además puedan satisfacer las condiciones de optimalidad de segundo orden necesarias para la ecuación \ref{eq:4.11}.

\subsubsection{Modificaciones de Regiones de Confianza en el Método de Newton}

Este tipo de algoritmos tienen su iteración principal como muestra la siguiente ecuación
\begin{equation*}
x^{k+1} = x^k + s^k
\end{equation*}
donde el paso $s^k$ se obtiene minimizando la forma cuadrática $q^k$ de la función objetivo en una región de confianza del espacio $\mathbb{R}^n$. 
La región de confianza se define como una norma $l_p$ del paso $s$. 
Lo más común es elegir la norma Euclidiana, con la cual, en cada iteración, $s^k$ es la solución de 
\begin{equation*}
\min_{s\in\mathbb{R}^n}\dfrac{1}{2}s'\nabla^2 f(x^k)s + \nabla f(x^k)'s
\label{eq:4.26}
\end{equation*}
donde $||s||^2 \leq (a^k)^2$ con $a$ siendo el radio de la región de confianza. 
Otra opción consistiría en realizar un cambio de escala a la condición previa de la siguiente forma: $||D^ks||^2\leq (a^l)^2$. 
Por simplicidad, asumiremos que la matriz $D^k=I$, es decir, la matriz identidad en el espacio adecuado.

Estos algoritmos se caracterizan por la siguiente idea: cuando la matriz $\nabla^2 f(x^k)$ es definida positiva, entonces el radio $a^k$ tiene que ser lo suficientemente grande que el minimizador de \ref{eq:4.26} no tenga restricciones y el paso dado por Newton sea un entero. 
Además, $a^k$ se actualiza en cada iteración y su instrucción de actualización depende de la proporción $\rho^k$ entre la reducción de los valores de la función objetivo $f(x^k)-f(x^{k+1})$ y la reducción esperada $f(x^k)-q^k(s^k).$

El siguiente algoritmo presenta las ideas anteriores y también garantiza la satisfacción de las condiciones de optimalidad necesarias para \ref{eq:4.11} en su tercer paso. 

\begin{algorithm}[H]
\caption{Método de Newton basado en Regiones de Confianza}\label{alg:TrustRegionNewton}
\begin{algorithmic}[1]
\Procedure \textsc{NewtonRegionConfianza} ($0<\gamma_1\leq\gamma_2<1, 0 \delta_1 < 1 \leq\delta_2$)
\State Establecer $x^0\in\mathbb{R}^n$, $k=0$ y $a^0=0$.
\State Encontrar $s^k = arg \min_\{||s||\leq a^k\}q^k(s)$
\If{$f(x^k)==q^k(s^k)$}
	\State Parar
\Else
	\State Calcular la proporción \begin{equation*}
	\rho^k = \dfrac{f(x^k)-f(x^k+s^k)}{f(x^k)-q^k(s^k)}
	\end{equation*}
	\If{$\rho^k\geq\gamma_1$}
		\State Actualizar $x^{k+1}=x^k + s^k$
	\Else
		\State Actualizar $x^{k+1} = x^k$
	\EndIf
	\State Actualizar el radio $a^k$
	\begin{equation*}
	a^{k+1} = \left\{ \begin{array}{lcc}
             \delta_1a^k &   si  & \rho^k < \gamma_1 \\
             \\ a^k &  si & \rho^k\in\in [\gamma_1,\gamma_2] \\
             \\ \delta_2a^k & si & \rho^k > \gamma_2
             \end{array}
   \right.
	\end{equation*}
	y establecer $k=k+1$, entonces volver al paso 3.
\EndIf
\EndProcedure
\end{algorithmic}
\end{algorithm}

Si $f$ es además continuamente 2-diferenciable, entonces la secuencia del algoritmo anterior, $\{x^k\}$, tiene un punto límite que satisface las condiciones de optimalidad necesarias de primer y segundo orden para \ref{eq:4.11}. 
Además, si $\{x^k\}$ converge a un punto en el que la matriz Hessiana $\nabla^2f$ es definida positiva, entonces el orden de convergencia es superlineal. 

El esfuerzo computacional de este algoritmo es el subproblema de las regiones de confianza. 
Debido a este hecho, se han ido desarrollando cada vez más algoritmos para resolverlo. 
Sin embargo, no necesitamos una solución exacta para esta ecuación. 
Para probar que la convergencia global del algoritmo es suficiente verificar que el valor $q^k(s^k)$ es menor que el valor en un punto de Cauchy, que es el punto que minimiza el modelo cuadrático en la región de confianza.

\subsubsection{Métodos de Newton Truncados}

Los métodos de Newton necesitan calcular la solución de un sistema lineal de ecuaciones en cada iteración. 
Si echamos un vistazo a problemas a gran escala, resolver este sistema en cada iteración puede resultar demasiada carga computacionalmente hablando. 
Además, la solución exacta cuando $x^k$ está lejos de una solución y $||\nabla f(x^k)||$ es grande no es necesaria. 
Debido a este hecho, se han propuesto numerosos métodos que calculan una solución aproximada a este sistema con un orden de convergencia bueno, es decir, si $\widetilde{d^k_N}$ es una solución aproximada del sistema, entonces la medida de precisión es dada por el residual de la ecuación de Newton, que sería 
\begin{equation*}
r^k = \nabla^2 f(x^k)\widetilde{d^k_N} + \nabla f(x^k)
\end{equation*}

Si podemos controlar este residual, entonces podemos probar la convergencia superlineal.

\begin{proposicion}
Si $\{x^k\}$ converege a una solución y si 
\begin{equation*}
\lim_{k\xrightarrow{}{}\infty} \dfrac{||r^k||}{||\nabla f(x^k)||} = 0
\end{equation*}
entonces $\{x^k\}$ converge de forma superlineal.
\end{proposicion}

Estos métodos solo requieren de operaciones matriciales-vectoriales, por lo que son adecuados para problemas a gran escala. 
Al siguiente algoritmo se le conoce como el método de Newton truncado y se necesita que la matriz Hessiana sea definida positiva. 
Este método se obtiene aplicando un esquema de gradiente conjugado para encontrar una solución óptima del sistema \ref{eq:4.22}.

Generaremos los vectores $d_N^i$, que se van a ir aproximando a la dirección $d_N^k$ y se detendrá cuando ocurra uno de los siguientes casos:
\begin{itemize}
	\item El residual $r^i$ verifica que $||r^i|| \leq \epsilon^k$, para $e^k > 0$.
	\item Si se ha encontrado una dirección de curvatura negativa, es decir, la dirección conjugada $s^i$ verifica que $(s^i)'\nabla^2f(x^k)s^i\leq 0$.
\end{itemize}

Este algoritmo tiene el mismo resultado para convergencia que el método de Newton. 
Por simplicidad, eliminaremos las dependencias en la iteración $k$ y estableceremos $H = \nabla^2 f(x^k)$ y $g=\nabla f(x^k)$. 
Este método genera las direcciones conjugadas $s^i$ y los vectores $p^i$ que aproximan la solución del sistema de Newton $\widetilde{d^k_N}$

\begin{algorithm}[H]
\caption{Algoritmo de Newton Truncado}\label{alg:NewtonTruncado}
\begin{algorithmic}[1]
\Procedure \textsc{NewtonRegionConfianza} ($k,H,g$ y $\eta>0$ escalar)
\State Establecer $i=0, p^0=0,r^0=-g,s^0=r^0$ y \begin{equation*}
	\epsilon = \eta ||g||\min\left\lbrace \dfrac{1}{k+1}, ||g|| \right\rbrace
\end{equation*}
\While{}

\EndWhile
\State \begin{equation*}
	a_N = \left\{ \begin{array}{lcc}
             \delta_1a^k &   si  & \rho^k < \gamma_1 \\
             \\ a^k &  si & \rho^k\in\in [\gamma_1,\gamma_2] \\
             \\ \delta_2a^k & si & \rho^k > \gamma_2
             \end{array}
   \right.
	\end{equation*}
\EndProcedure
\end{algorithmic}
\end{algorithm}

Si aplicamos este algoritmo para encontrar una solución aproximada al sistema en el cuarto paso de \ref{alg:LSNewton} o \ref{alg:NonMonotoneNewton}, entonces obtendremos la versión monótona truncada del método de Newton.

\subsubsection{Métodos Quasi-Newton}

Estos métodos han sido introducidos con el objetivo de diseñar algoritmos eficientes que no necesiten la evaluación de derivadas de segundo orden. 
Por tanto, han establecido $d^k$ como la solución de 
\begin{equation}
B^kd = -\nabla f(x^k)
\label{eq:4.27}
\end{equation}
donde $B^k$ es una matriz definida positiva simétrica de tamaño $n\times n$ que se ajusta de forma iterativa para que la dirección $d^k$ tienda a aproximar la dirección del método de Newton. 
A la fórmula anterior se le conoce como \textbf{fórmula quasi-Newton} y su inversa es
\begin{equation}
d^k = - H^k\nabla f(x^k)
\label{eq:4.28}
\end{equation}

Ambas matrices se modifican en cada iteración como una correción de la anterior, es decir, $B^{k+1} = B^k + \Delta B^k$, y lo mismo pasa para $H^k$. 
Definimos las siguientes dos cantidades:
\begin{equation*}
\delta^k = x^{k+1} - x^k \hspace{0.5cm} \gamma^k = \nabla f(x^{k+1}) - \nabla f(x^k)
\end{equation*}

En caso de que $f$ sea una función cuadrática, entonces la ecuación quasi-Newton sería
\begin{equation}
\nabla^2 f(x^k)\delta^k = \gamma^k
\label{eq:4.29}
\end{equation}
por lo tanto, $\Delta B^k$ (lo mismo para $\Delta H^k$) se elige de la siguiente forma
\begin{equation}
(B^k + \Delta B^k)\delta^k = \gamma^k
\label{eq:4.30}
\end{equation}

La regla de actualización de $H^k$ está dada por
\begin{equation}
\Delta H = \dfrac{\delta^k(\delta^k)'}{(\delta^k)'\gamma^k} - \dfrac{H^k\gamma^k(H^k\gamma^k)'}{(\gamma^k)'H^k\gamma^k} + c(\gamma^k)'H^k\gamma^kv^k(v^k)'
\label{eq:4.31}
\end{equation}
donde
\begin{equation*}
v^k = \dfrac{\delta^k}{(\delta^k)'\gamma^k} - \dfrac{H^k\gamma^k}{(\gamma^k)'H^k\gamma^k}
\end{equation*}
y $c>0$ es un escalar. 
El algoritmo que vamos a mostrar ahora fue creado por Broyden, Flecher, Goldfarb y Shanno. 
Es un método de búsqueda lineal en el que el tamaño de paso es $\alpha^k$ se obtiene mediante un algoritmo de búsqueda lineal. 
El esquema del algoritmo se presenta debajo:

\begin{algorithm}[H]
\caption{Algoritmo BFGS Inverso Quasi-Newton}\label{alg:BFGSQuasiNewton}
\begin{algorithmic}[1]
\Procedure \textsc{InversaBFGSQuasiNewton}
\State Establecer $x_0\in\mathbb{R}^n, H^0=I$ y $k=0$
\While{$\nabla f(x^k)!=0$}
	\State Establecer la dirección $d^k = -H^k\nabla f(x^k)$
	\State Encontrar $\alpha^k$ mediante una búsqueda lineal que satisfaga las condiciones de Wolfe \ref{eq:4.20} y \ref{eq:4.21}
	\State Actualizar \begin{equation}
	\begin{matrix}
  x^{k+1} & = & x^k+\alpha^kd^k\\
  H^{k+1} & = & H^k + \Delta H^k
 \end{matrix}
	\end{equation}
	con $\Delta H^k$ dada por \ref{eq:4.31} con $c=1$.
	\State Establecer $k = k+1$
\EndWhile
\EndProcedure
\end{algorithmic}
\end{algorithm}

Distinguimos dos casos en relación a las propiedades de convergencia: caso convexo y caso no convexo. 
Cuando no tenemos convexidad, si existe una constante $\rho$ tal que para cada $k$ se verifica la siguiente condición
\begin{equation}
\dfrac{||\gamma^k||^2}{(\gamma^k)'\delta^k} \leq \rho
\label{eq:4.32}
\end{equation}
entonces la secuencia de puntos generada por el algoritmo satisface
\begin{equation*}
\lim_{k\xrightarrow{}{}\infty} inf||\nabla f(x^k)|| = 0
\end{equation*}
que es la condición débil de \ref{eq:3.5}. 
Para el caso convexo, la desigualdad anterior se mantiene. 
El siguiente resultado nos dará información sobre el orden de convergencia de este algoritmo. 

\begin{proposicion}
Sea $\{B^k\}$ una secuencia de matrices no singulares y sea $\{x^k\}$ la secuencia dada por
\begin{equation*}
x^{k+1} = x^k - (B^k)^{-1}\nabla f(x^k)
\end{equation*}
También supondremos que $\{x^k\}$ converge a un punto $x^*$ donde $\nabla^2 f(x^*)$ es también no singular. 
Entonces, la secuencia $\{x^k\}$ converge de forma superlineal a $x^* \Leftrightarrow$
\begin{equation*}
\lim_{k\xrightarrow{}{}\infty}\dfrac{||[B^k-\nabla^2f(x^*)](x^{k+1}-x^k)||}{||x^{k+1}-x^k||} = 0
\end{equation*}
\end{proposicion}

Este algoritmo está pensado para problemas de pequeña escala, ya que para los de gran escala, el almacenar una matriz como serían $B^k$ o $H^k$ ocasionaría problemas de almacenamiento. 
Por lo tanto, para esos problemas la información se obtiene de las últimas iteraciones. 

\subsubsection{Métodos sin derivadas}

Estos métodos no calculan explícitamente las derivadas de $f$. 
Son adecuados para cuando, o bien, el gradiente de la función objetivo no puede ser calculado, o bien, cuando es computacionalmente costoso. 
Sin embargo, si queremos probar las propiedades de convergencia, necesitaremos suponer que $f$ es continuamente diferenciable. 

De entre todos estos algoritmos, los dos más importantes son los algoritmos de búsqueda de patrones (PSA, \textit{pattern-search algorithm}) y los algoritmos de búsqueda lineal sin derivadas (DFLSA, \textit{derivative-free line-search algorithm}). 
Estos algoritmos son similares, y su diferencia reside en las suposiciones que hacen sobre el conjunto de direcciones y en la regla que usan para encontrar el tamaño de paso junto con las direcciones. 
Denotamos $\mathcal{D}^k = \{d^1,...,d^r\}$ con $r\geq n+1$ como el conjunto de direcciones y suponemos que son unitarias. 

También asumimos el siguiente hecho para los PSA: las direcciones $d^j\in\mathcal{D}^k$ son la j-ésima columna de la matriz $B\Gamma^k$ con $B$ una matriz no singular con coeficientes reales de tamaño $n$ y $\Gamma^k\in\mathcal{M}\subset \mathbb{Z}^{n\times r}$, donde $\mathcal{M}$ es un conjunto finito	de matrices integrales tales que su rango coincide con el número de filas que tienen (\textit{full row-rank}).

Esta suposición nos da una idea para entender que el PSA itera sobre $x^{k+1}$ en una red (\textit{lattice}) racional centrada en $x^k$. 
Sea $\mathcal{P}^k$ el conjunto de candidatos para la siguiente iteracion, es decir, 
\begin{equation*}
\mathcal{P} = \{x^{k+1} : x^k + \alpha^kd^j, d^j\in\mathcal{D}^k\}
\end{equation*}

En este conjunto se conoce el patrón y se elige el tamaño de paso para preservar la estructura algebraica en la siguiente iteración, por lo que tenemos $f(x^k+s^k) < f(x^k)$ con $s^k = \alpha^k d^j$.

Para DLFSA haremos la siguiente suposición: las direcciones $d^j\in\mathcal{D}^k$  son la j-ésima columna de una matriz $B^k$ de tamaño $n\times r$ con rango $n$. 
En este caso, no hay suposiciones adicionales y solo tiene que verificar la reducción de la función objetivo. 

Los dos siguientes algoritmos muestran una versión del PSA y del DFLSA.

\begin{algorithm}[H]
\caption{Algoritmo de Búsqueda de Patrones}\label{alg:PSA}
\begin{algorithmic}[1]
\Procedure \textsc{PSA} ($\mathcal{D^k}$ satisfaciendo la suposición previa, $\tau\in\{1,2\}, \theta=\frac{1}{2}$)
\State Establecer $x_0\in\mathbb{R}^n, \Delta^0>0$ y $k=0$
\State Comprobar la convergencia
\If{$\exists j\in\{1,\dots,r\} :$\begin{equation*}
f(x^k + \alpha^kd^j) < f(x^k), \alpha^k=\Delta^k
\end{equation*}
}
	\State Establecer $x^{k+1} = x^k$, $\Delta k+1 = \tau\Delta^k$, $k=k+1$ e ir al paso 3.
\Else
	\State Establecer $x^{k+1} = x^k$, $\Delta k+1=\theta\Delta^k$, $k = k+1$ 
\EndIf
\EndProcedure
\end{algorithmic}
\end{algorithm}

\begin{algorithm}[H]
\caption{Algoritmo de Búsqueda Lineal sin Derivadas}\label{alg:DFLSA}
\begin{algorithmic}[1]
\Procedure \textsc{DFLSA} ($\mathcal{D^k}$ satisfaciendo la suposición previa, $\gamma >0$, $\theta\in(0,1)$)
\State Establecer $x_0\in\mathbb{R}^n, \Delta^0>0$ y $k=0$
\State Comprobar la convergencia
\If{$\exists j\in\{1,\dots,r\} :$\begin{equation*}
f(x^k + \alpha^kd^j) \leq f(x^k) - \gamma(\alpha^k)^2, \alpha^k\geq\Delta^k
\end{equation*}
}
	\State Establecer $x^{k+1} = x^k$, $\Delta k+1 = \alpha^k$, $k=k+1$ e ir al paso 3.
\Else
	\State Establecer $x^{k+1} = x^k$, $\Delta k+1=\theta\Delta^k$, $k = k+1$ 
\EndIf
\EndProcedure
\end{algorithmic}
\end{algorithm}

Ambos algoritmos generan una secuencia que verifica la condición débil de convergencia de \ref{eq:4.5}, es decir,
\begin{equation*}
\lim_{k\xrightarrow{}{}\infty} inf||\nabla f(x^k)|| = 0
\end{equation*}

En este caso, tomamos un índice, $j$, en PSA tal que 

\begin{equation*}
f(x^k+\alpha^kd^j) = \min_{i:d^i\in\mathcal{D}^k} f(x^k+\alpha^kd^i) < f(x^k)
\end{equation*}
también mantenga la premisa de \ref{eq:4.5}. 
Esto también ocurre en DFLSA, porque solo tenemos que tomar un tamaño de paso, $\alpha^k$, tal que
\begin{equation*}
f\left(x^k+\dfrac{\alpha^k}{\delta}d^k\right) \geq \max\left\lbrace f(x^k+\alpha^kd^k), f(x^k)-\gamma\left(\dfrac{\alpha^k}{\delta}\right)^2\right\rbrace, \delta\in(0,1)
\end{equation*}

Ambos esquemas encuentran un tamaño de paso que les permita comprobar la convergencia del algoritmo. 
En estos algoritmos no necesitamos tener el gradiente de $f$, por lo que tenemos que comprobar la siguiente condición:
\begin{equation}
\sqrt[]{\dfrac{\sum_{i=1}^{n+1} (f(x^i)-\overline{f})^2}{n+1}} \leq tolerancia
\label{eq:4.33}
\end{equation}
donde $\overline{f} = \dfrac{1}{n+1}\sum_{i=1}^{n+1}f(x^i)$ y $\{x^i : i=1,\dots,n+1\}$ incluye el punto actual y los $n$ puntos anteriormente generados junto con las $n$ direcciones. 

Como resultado de esta teoría, tenemos algoritmos que nos permitan generar secuencias e puntos en $\mathbb{R}^n$ que converjan a los puntos óptimos y, bajo ciertas circunstancias, pueden ser el óptimo global. 
Estos métodos son el principio de la gran cantidad de algoritmos presentados en la literatura. 
Estos algoritmos proponen formas de alcanzar un óptimo global  de funciones basadas en una idea inspirada en la naturaleza. 

Incluso si la gran cantidad de algoritmos que presentaremos en las próximas secciones, el diseño de cada uno es similar a cómo esta teoría propone el movimiento, es decir, metaheurísticas basando su movimiento en una dirección dada por un vector y el criterio de parada basado en condiciones que normalmente presentan la optimalidad del mejor individuo de la población .
%
\chapter{Problema y Diseño Experimental}

En este capítulo se presenta el problema que se aborda con el algoritmo creado, así como las necesidades que surgen por el interés de identificar el aporte de este frente al panorama actual del campo en términos de rendimiento, las soluciones que existen para satisfacer dicha necesidad y cuál de ellas es la más adecuada.

\section{Descripción del problema}

El problema que se va a abordar en este proyecto es el de la optimización de problemas de tipo combinatorio. 

\begin{definicion}
Un \textbf{problema de optimización combinatoria} se define como aquel en el que el conjunto de soluciones posibles es discreto. 
Es decir, se trata de un problema de optimización que involucra una cantidad finita o numerable de soluciones posibles.
\end{definicion}
Este tipo de problemas se diferencia de los problemas de optimización continuos, en los cuales el conjunto de soluciones posibles es infinito e incontable. 

Dentro del campo de la optimización combinatoria es común que la mayoría de los procesos de resolución de problemas no puedan garantizar la solución óptima, incluso dentro del contexto del modelo que se esté utilizando. 
Sin embargo, la aproximación al óptimo suele ser suficiente para resolver los problemas en la práctica. 

Con el fin de poder estudiar más a fondo el comportamiento de los distintos algoritmos que se han estado desarrollando era necesario elegir un problema determinado con el que trabajar. 
En nuestro caso, se ha elegido la generalización del problema comúnmente llamado ``problema de la mochila'' (\textit{Knapsack Problem} (KP)):  \textbf{\textit{Quadratic Knapsack Problem} (QKP)}. 

%Justificación de por qué hemos usado QKP
Antes de definir el problema, justificaremos por qué se ha elegido este problema. 
La primera razón, y posiblemente la más importante, es la ausencia de \textit{benchmarks} para problemas combinatorios \textit{expensive}, por lo que nos hemos visto obligados a crear el nuestro propio para un futuro. 
Ante este panorama también debemos encontrar un problema específico adecuado sobre el que trabajar desde cero y que resulte de interés. 
Como el objetivo inicial estaba relacionado con redes neuronales, buscamos un problema que puede tener representación binaria (\texttt{1} para la elección de elementos, \texttt{0} para el caso contrario). 
En este sentido, el QKP cumple con este requisito, además tiene interés añadido debido a que es un problema con restricciones y constituye una alternativa moderna de un problema clásico; además de que podemos generar instancias de este problema con distintos tamaños. 
Además, es un problema que tiene muchas aplicaciones en el mundo real en situaciones donde los recursos con distintos niveles de interacción tienen que distribuirse entre distintas tareas, por ejemplo, asignar compañeros de equipo a distintos proyectos donde las contribuciones de cada miembro se consideran de forma individual y por parejas. 
Por lo tanto, teniendo en cuenta la falta de \textit{benchmarks} y referencias, el QKP resulta ser una buena opción, ya que es un problema difícil, costoso y moderno. 


\subsection{Quadratic Knapsack Problem}
%The quadratic knapsack problem—a survey
Se procede a definir en profundidad dicho problema. 
En primer lugar, se tienen $n$ elementos donde cada elemento $j$ tiene un peso entero positivo $w_j$. 
Adicionalmente, se nos da una matriz de enteros no negativos de tamaño $n\times n$, $P = \{p_{ij}\}$, donde $p_{jj}$ es el beneficio asociado a elegir el elemento $j$ y $p_{ij}+p_{ji}$ es el beneficio que se alcanza si ambos elementos $i,j$, con $i<j$ son seleccionados. 
Consideramos que una combinación de elementos es una solución a QKP cuando peso el total (la suma del peso de todos los elementos seleccionados) no superan la capacidad máxima de la mochila dada, $c$. 
Así, el problema consiste el maximizar el beneficio total sin sobrepasar la capacidad máxima.

Por conveniencia en la notación, establecemos que $N=\{1,\dots,n\}$ denotará el conjunto de elementos. 
Representando la lista de elementos de forma binaria, $x_j$, para indicar si el elemento $j$ ha sido seleccionado (su valor será 0 si no ha sido seleccionado, 1 en caso contrario), el problema podrá ser formulado de la siguiente forma:
\begin{equation}
\begin{aligned}
  \text{maximizar} & \sum_{i\in N}\sum_{j\in N}p_{ij}x_ix_j \\
  \text{sujeto a } & \sum_{j\in N}w_jx_j \leq c\\
  & x_j\in \{0,1\}, j \in N 
\end{aligned}
\label{eq:QKP}
\end{equation}

Sin pérdida de generalidad, podemos suponer que:
\begin{itemize}
	\item $\max_{j\in N} w_j \leq c < \sum_{j\in N}w_j$
	\item La matriz de beneficios es simétrica, es decir, $p_{ij} = p_{ji}$, $\forall j > i$.
\end{itemize}

Una vez definido el problema, es fácil ver por qué es una versión generalizada del KP. 
KP se puede obtener a partir de QKP si $p_{ij} = 0$, para todo $i\neq j$. 
También se considera una versión restringida del \textit{Quadratic 0-1 Programming Problem} (QP), el cual se define como \ref{eq:QKP} sin la restricción de capacidad.

Como uno cabría esperar, debido a su generalidad, el QKP tiene un amplio espectro de aplicaciones. 
Witzgall \parencite{witzgallMathematicalMethodsSite1975} presentó un problema que surge en telecomunicación cuando un número de localizaciones para satélites tienen que ser seleccionados, tales que el tráfico global entre estas estaciones se maximice y la limitación de presupuesto se cumpla; este problema resulta ser un QKP. 

%En tanto que QKP es un problema $\mathcal{NP}-hard$, no podemos esperar encontrar una aproximación totalmente polinómica a no ser que $\mathcal{NP}=\mathcal{P}$. 
%Sin embargo, Rader y Woeginger %referencias
%desarrollaron un esquema de aproximación en tiempo completamente polinómico (FPTAS, \textit{Fully polynomial-time approximation scheme}) para el caso especial donde todos los beneficios $p_{ij}\geq 0$ y donde 

\subsection{Datos del problema}

Utilizaremos 97 archivos de datos generados aleatoriamente de %pedir referencia a página
, los cuales se pueden distribuir de forma que se indica en la Tabla \ref{DatosProblema}



Se entiende como ``densidad'' al porcentaje de beneficios combinados positivos, es decir, $p_{ij} > 0$. 
Particularmente QP tendría densidad 0\%.

Ahora bien, todos los archivos tienen el mismo formato, lo cual resulta útil para definir funciones capaces de obtener los datos más relevantes para nuestros algoritmos. 
El formato que siguen los archivos es el siguiente:
\begin{itemize}
	\item La referencia de la instancia: Su nombre.
	\item El número de variables ($n$)
	\item Los coeficientes lineales de la función objetivo $p_{jj}$
	\item Los coeficientes cuadráticos $p_{ij}$: representados en líneas
	\item Una línea en blanco
	\item 0 si la restricción es de tipo $\leq$, lo cual siempre va a ocurrir ya que estamos considerando instancias QKP.
	\item La capacidad $c$ de la mochila.
	\item Los coeficientes de capacidad/peso, $w_j$.
	\item Algunos comentarios.
\end{itemize}

\begin{table}[H]
\caption{Datos del Problema}
\label{DatosProblema}
\begin{tabular}{|c|c|c|}
\hline
\rowcolor[HTML]{F7EAC7} 
\multicolumn{1}{|l|}{\cellcolor[HTML]{F7EAC7}Número de variables} & \multicolumn{1}{l|}{\cellcolor[HTML]{F7EAC7}Densidad} & \multicolumn{1}{l|}{\cellcolor[HTML]{F7EAC7}Número de archivos} \\ \hline
\rowcolor[HTML]{DDFDFF} 
\cellcolor[HTML]{DAE8FC}                                          & 25\%                                                  & 10                                                              \\ \cline{2-3} 
\cellcolor[HTML]{DAE8FC}                                          & 50\%                                                  & 10                                                              \\ \cline{2-3} 
\rowcolor[HTML]{DDFDFF} 
\cellcolor[HTML]{DAE8FC}                                          & 75\%                                                  & 10                                                              \\ \cline{2-3} 
\multirow{-4}{*}{\cellcolor[HTML]{DAE8FC}n = 100}                 & 100\%                                                 & 9                                                              \\ \hline
\rowcolor[HTML]{DAE8FC} 
\cellcolor[HTML]{DDFDFF}                                          & 25\%                                                  & 9                                                              \\ \cline{2-3} 
\cellcolor[HTML]{DDFDFF}                                          & 50\%                                                  & 10                                                              \\ \cline{2-3} 
\rowcolor[HTML]{DAE8FC} 
\cellcolor[HTML]{DDFDFF}                                          & 75\%                                                  & 10                                                              \\ \cline{2-3} 
\multirow{-4}{*}{\cellcolor[HTML]{DDFDFF}n = 200}                 & 100\%                                                 & 10                                                              \\ \hline
\rowcolor[HTML]{DDFDFF} 
\cellcolor[HTML]{DAE8FC}                                          & 25\%                                                  & 9                                                              \\ \cline{2-3} 
\multirow{-2}{*}{\cellcolor[HTML]{DAE8FC}n = 300}                 & 50\%                                                  & 10                                                              \\ \hline
\end{tabular}
\end{table}

%Introducir algún ejemplo?

\section{Diseño Experimental}

\subsection{Criterio de Parada}

Se han elegido las instancias mencionadas anteriormente ya que también se proporcionaban algunos resultados de otros algoritmos, por lo que resultaba conveniente a la hora de comprobar si los resultados obtenidos por nuestros algoritmos bases eran competitivos. 
Ya que, en el caso de que no lo fuesen, tendríamos que buscar otros algoritmos base sobre los que trabajar.
Sin embargo, nos encontramos con un problema, esto es, el criterio de parada presentado en estos casos es el tiempo.
Adicionalmente, el tiempo de parada también depende del número de elementos, $n$, de forma que se tiene:
\begin{itemize}
\item Para $n = 100$, se tienen \textbf{5 segundos} de ejecución.
\item Para $n > 100$, en nuestro caso, $n = 200$ y $n = 300$, se tienen \textbf{30 segundos} de ejecución.
\end{itemize}

Como se ha indicado antes, utilizar el tiempo de ejecución como criterio de parada resulta un problema. 
Esto se debe a que no es un criterio de parada fiable, ya que depende de la capacidad de computación de cada ordenador. 
No es comparable la velocidad de los ordenadores actuales con la velocidad de los ordenadores de dentro de 10 años, de la misma no podemos comparar el rendimiento de un ordenador de hace 10 años con respecto a uno actual. 
E incluso dentro de los ordenadores de la misma generación, dependerá de las características propias de cada computador. 
Por lo que se llega a la conclusión de que si se quiere que los resultados obtenidos en este trabajo puedan ser usados como referencia, o incluso si se quiere recrear el trabajo, en un futuro se debe cambiar el criterio de parada a algo portable, a algo que sea independiente de cuándo se produzca el experimento. 
Por ello, se ha decidido cambiar el criterio de parada a un número de iteraciones máximo. 
Este criterio sí es portable, ya que independientemente de qué tipo de computador se utilice para obtener los resultados siempre se van a obtener los mismos resultados utilizando los mismos parámetros.

Primero debemos indicar lo que entenderemos por iteraciones. 
Denotaremos como iteraciones al número de veces que se repite el procedimiento completo un determinado algoritmo, en nuestro caso se va a traducir en lo siguiente: 
\begin{itemize}
	\item Para el algoritmo \textit{Random}, el número iteraciones será el número de veces que generemos una solución de forma aleatoria.
	\item Para los algoritmos genéticos, una iteración puede suponer un número distinto de evaluaciones. 
	En los algoritmos genéticos al número de iteraciones se llama también generaciones.
\end{itemize}

Dicho esto, para elegir el número de iteraciones máximo que se iba a utilizar se tomó como referencia el tiempo establecido anteriormente. 
En tanto se tenía elegir un número de iteraciones como criterio de parada, se decidió estudiar a qué equivalía actualmente el criterio de parada por tiempo establecido. 
Para ello, mediante una variable \texttt{contador}, se ejecutaron todos los archivos con el tiempo como criterio de parada y se mostraba por pantalla el número de iteraciones que se había alcanzado. 
Tras realizar una media de todos estos datos y redondearlo, obtenemos que el nuevo criterio de parada es \textbf{90000 iteraciones} para todos los archivos. 
El que el número de iteraciones máximo no dependa del número de elementos como lo hacía el tiempo de ejecución máximo se puede justificar en tanto que se necesita más tiempo para realizar todos los cálculos si aumenta el número de datos. 

Además, para asegurarnos que realmente ambos criterios de parada eran equivalentes, se ejecutaron todos los archivos usando el algoritmo base con criterio de parada por iteraciones y por tiempo. 
Nótese que no solo se almacenan las soluciones finales, si no que también se almacenan las soluciones intermedias llegado a ciertos porcentajes de la ejecución. 
Tras esto, se compararon ambos resultados y se pudo comprobar que, efectivamente, eran equivalentes. 

Por lo tanto, se puede decir con seguridad que los cambios que apliquemos a un criterio de parada en específico se puede traducir a un cambio en el otro criterio de parada. 
Esto cabe la pena destacarlo ya que, recordemos, el objetivo de este trabajo es crear un algoritmo útil y competitivo para tratar con problemas \textit{expensive}, por lo que queremos obtener buenos resultados en un tiempo muy reducido. 

Para simular de forma eficiente esta reducción del tiempo, se propuso el siguiente cambio con respecto al tiempo como criterio de parada:
\begin{itemize}
	\item En vez de ejecutar los archivos con $n=100$ durante 5 segundos, lo reduciremos a 25ms.
	\item En vez de ejecutar los archivos con $n>100$ durante 30 segundos, lo reduciremos a 150ms.
\end{itemize}
Haciendo cálculos, obtenemos que solo utilizaremos el 0.5\% inicial de cada ejecución, lo que podemos traducir en que nuestro nuevo criterio de parada serán \textbf{450 iteraciones}. 

Con el fin de comprobar que esta reducción había sido suficiente y necesaria, se ejecutan de nuevo todos los archivos con el algoritmo base ahora con 450 iteraciones y comparamos los resultados con los obtenidos anteriormente con 90000 iteraciones. 
Podemos ver que, efectivamente, se han obtenido resultados mucho peores. 

Finalmente, ya hemos establecido nuestro criterio de parada escalable y tenemos unos resultados base que utilizar. 
A partir de esto, nuestro objetivo será mejorar estos resultados lo máximo posible.

\subsection{Parámetros}

La ejecución del programa no requiere de ningún tipo de parámetro que se tenga que introducir manualmente. 
Todos los parámetros que se nombren a continuación vienen definido dentro del código del programa \texttt{main} como constantes globales o que dependen del propio problema. 

En primer lugar, hablemos sobre las constantes globales:
\begin{itemize}
	\item \texttt{NEVALUACIONESMAX}: Es el número de iteraciones máximas, mantendremos su valor a 450 por lo explicado en el apartado anterior. 
	\item \texttt{NTRIES}: Para que nuestros resultados no sean muy dependientes de la aleatoriedad, es necesario ejecutar cada archivo cierta cantidad de veces y obtener la media. 
Esta media será verdaderamente el resultado que guardaremos y compararemos con el resto. 
De forma más o menos arbitraria, se ha establecido su valor a 50.
	\item \texttt{INITSEED}: En relación con la constante anterior, es necesario no utilizar siempre la misma semilla de aleatoriedad, ya que eso resultaría en obtener siempre los mismos resultados, por lo que ejecutarlos varias veces sería una pérdida de tiempo. 
Así pues, tendremos que utilizar distintas semillas para cada ejecución. 
Además, por conveniencia, es recomendable que las semillas utilizadas sean comunes a todas las sucesivas ejecuciones de los distintos algoritmos (una vez más, para asegurarnos que la aleatoriedad afecta a todos los algoritmos de forma similar) y que siempre se obtengan los mismos resultados aún si ejecutamos el mismo archivo varias veces. 
Por ello, una solución es establecer un valor de semilla inicial arbitrario y utilizar este para generar el resto de semillas.
\end{itemize}

En definitiva, podemos resumir los parámetros, para lo que se usan y sus valores en la Tabla \ref{Resumen}.

\begin{table}
\caption{Resumen de parámetros utilizados}
\label{Resumen}
\begin{tabular}{|l|l|l|}
\hline
\rowcolor[HTML]{F7EAC7} 
Parámetros                                 & Resumen                                                                                                                                   & Valor                                                                                 \\ \hline
\rowcolor[HTML]{DAE8FC} 
\texttt{NEVALUACIONESMAX} & \begin{tabular}[c]{@{}l@{}}Criterio de parada: Número de\\ Iteraciones Máximas para \\ cada algoritmo\end{tabular}                       & 450                                                                                   \\ \hline
\rowcolor[HTML]{DDFDFF} 
\texttt{NTRIES}           & \begin{tabular}[c]{@{}l@{}}Número de veces que se va a \\ ejecutar el algoritmo por cada\\ archivo\end{tabular}                           & 50                                                                                    \\ \hline
\rowcolor[HTML]{DAE8FC} 
\texttt{INITSEED}         & \begin{tabular}[c]{@{}l@{}}Semilla inicial para poder \\ generar el resto de semillas que\\ utilizaremos para los algoritmos\end{tabular} & 5                                                                                     \\ \hline
\rowcolor[HTML]{DDFDFF} 
\texttt{EvaluacionMAX}    & \begin{tabular}[c]{@{}l@{}}Criterio de parada: Número de\\ Iteraciones Máximas dicho\\ algoritmo\end{tabular}                            & \texttt{NEVALUACIONESMAX}                                           \\ \hline
\rowcolor[HTML]{DAE8FC} 
\texttt{Semilla}          & \begin{tabular}[c]{@{}l@{}}Semilla de aleatoriedad para\\ determinada ejecución del \\ algoritmo\end{tabular}                             & \begin{tabular}[c]{@{}l@{}}Valor generado aleatoriamente \\ por \texttt{INITSEED}\end{tabular} \\ \hline
\rowcolor[HTML]{DDFDFF} 
%\texttt{numcro}             & \begin{tabular}[c]{@{}l@{}}Número de elementos de la\\ población de soluciones\end{tabular}                                               & 10                                                                                    \\ \hline
%\rowcolor[HTML]{DAE8FC} 
%\texttt{pmut}             & \begin{tabular}[c]{@{}l@{}}Probabilidad de mutación\\ de las soluciones de la población\\ ({[}0,1{]})\end{tabular}                        & 0.1                                                                                   \\ \hline
\end{tabular}
\end{table}

%
\chapter{Algoritmos de Referencia}

En este capítulo se presentan los dos algoritmos que se utilizaran como base para desarrollar un algoritmo competitivo para problemas \textit{expensive}, se describirán con sus referencias y se indicarán algunas características; aunque se explicarán de forma más detallada los componentes propios de cada uno en el siguiente capítulo.

\section{Algoritmo Genético Estacionario Uniforme}

En primer lugar, debemos explicar brevemente la importancia de los Algoritmos Genéticos (AG) y, posteriormente, justificar la elección de su versión Algoritmo Genético Estacionario Uniforme (AGEU). 

Los seres vivos son solucionadores de problemas de forma natural. 
Exhiben una versatilidad que ponen en evidencia hasta a los mejores programas. 
Esta observación es especialmente humillante para los informáticos, que necesitan utilizar meses e incluso años de esfuerzos intelectuales en un algoritmo, mientras que estos organismos obtienen sus habilidades a través de los aparentemente indirectos mecanismos de evolución y selección natural. 

Los investigadores más pragmáticos observan el notable poder de la evolución como algo que simular. 
La selección natural elimina uno de los mayores inconvenientes en el diseño de software: especificar de antemano todas las características de un problema y las acciones que dicho programa tendría que tomar para tratar con ellas. 
Aprovechando los mecanismos de evaluación, los investigadores pueden ser capaces de ``reproducir'' programas que resuelvan problemas incluso cuando nadie pueda comprender enteramente su estructura. 
Efectivamente, estos llamados \textbf{algoritmos genéticos} han demostrado la habilidad de hacer avances en el diseño de sistemas complejos. 

Los AGs hacen posible explorar un rango mucho más amplio de posibles soluciones a un problema que programas convencionales. 
Además, en los estudios  realizados sobre la selección natural de programas bajo condiciones controladas bien entendidas, los resultados prácticos alcanzados pueden aportar cierto conocimiento sobre los detalles de cómo la vida y la inteligencia evolucionan en el mundo natural. 

El funcionamiento de un AG viene dado por el siguiente pseudocódigo (\ref{alg:AG})

\begin{algorithm}[H]
\caption{Algoritmo Genético}\label{alg:AG}
\begin{algorithmic}[1]
\Procedure \texttt{AG}($EMax > 0, nelem > 0, pcruce \in [0,1], pmut \in [0,1]$)
\State Generar una población inicial aleatoria
\State Calcular la función \textit{fitness} de cada individuo
\State \texttt{generacion} = 0
\While{\texttt{generacion} < EMax}
	\State Calcular el número de parejas a formar para el cruce $\xrightarrow{}{} ncruce = pcruce*nelem$
	\State Seleccionar aleatoriamente con repetición $4*ncruce$ soluciones de la población
	\State Aplicar torneo de 2 en 2 soluciones del conjunto anterior y almacenar solo la mejor $\xrightarrow{}{}$ \texttt{padres}
	\For{$i\in[0,ncruce]; i+=2$}
		\State Generar 2 hijos cruzando \texttt{padres}$[i]$ y \texttt{padres}$[i+1]$
		\State Aplicar el Operador de Reparación sobre ambos hijos
		\State Calcular la función \textit{fitness} de cada hijo
		\State Almacenar dichos hijos $\xrightarrow{}{}$ \texttt{hijos}
	\EndFor	
	\State Calcular el número de soluciones a mutar $\xrightarrow{}{} nmut = pmut*nelem$
	\State Mutar $nmut$ soluciones distintas
	\State Calcular la función \textit{fitness} de las nuevas soluciones
	\State Aplicar el Operador de Selección sobre la población actual e \texttt{hijos}
	\State \texttt{generacion} = \texttt{generacion}+1
\EndWhile
\EndProcedure
\end{algorithmic}
\end{algorithm}

En el caso de la versión AGEU, su pseudocódigo viene representado en \ref{alg:AGEU}. 
Claramente sigue la misma estructura, pero presenta dos especificaciones:
\begin{itemize}
	\item Estacionario (E): En relación con el operador de selección. 
Se enfrentan las soluciones hijas con las de la población de la generación anterior y mantenemos las mejores. 
	\item Uniforme (U): En relación con el cruce de las soluciones. 
Las soluciones hijas van a tener de partida los elementos comunes a ambas soluciones padres. 
El resto de los elementos de los hijos se obtienen de forma que para algunos elementos \texttt{hijo$_i$} los obtiene de \texttt{padre$_i$} y el resto los obtiene del otro padre. 
El objetivo de esto es preservar selecciones prometedoras.	
\end{itemize}
También se estuvieron barajando sus contrapartidas estudiadas en la asignatura Metaheurísticas:
\begin{itemize}
	\item Generacional (G): En relación con el operador de selección. 
Sustituimos la antigua generación por la nueva. 
	\item Posición (P): En relación con el cruce de las soluciones. 
Las soluciones hijas van a tener de partida los elementos comunes a ambas soluciones padres. 
Sin embargo, en este caso las asignaciones restantes se toman de un padre (no importa cual) y se asignan en un orden aleatorio distinto para completar cada hijo. 
\end{itemize}

La versión generacional puede ocasionar que se pierda la mejor solución hasta el momento, lo que impediría que se pudiese seguir una búsqueda profundizando en dicha solución. 
Como tenemos pocas iteraciones, no nos podemos permitir buenas soluciones sin ningún tipo de garantía, así que no sería un buen enfoque inicial. 

Por otra parte, la versión de posición no tendría mucho sentido en nuestro problema, ya que una de sus características más importantes se pierde, esto es, no podemos garantizar que a partir de dos soluciones factibles obtengamos otra factible; por lo que seguimos necesitando el operador de reparación. 
Además, en comparación con la versión uniforme, es muy disruptiva, comparte menos información de los padres y puede ser más complicado que converja. 

A efectos prácticos, se ha usado los resultados y el análisis que realicé en el trabajo ``Problemas con técnicas basadas en poblaciones'' de la asignatura Metaheurísticas (curso 2021-2022), donde se debía comparar experimental y teóricamente los distintos algoritmos genéticos y meméticos. 
En dicho trabajo se llega a la conclusión de que la mejor opción es utilizar AGEU.

\subsection{Pseudocódigo}

A efectos prácticos, este pseudocódigo (\ref{alg:AGEU}) se diferenciará del anterior (\ref{alg:AG}) en el cruce, ya que solo cruzaremos dos padres no necesitaremos el parámetro $pcruce$, y que en el operador de selección especificaremos que es estacionario. 

Como se ha dicho al principio de este capítulo, una explicación más detallada junto con el pseudocódigo de cada una de las componentes será dada en el siguiente capítulo.

\begin{algorithm}[H]
\caption{Algoritmo Genético Estacionario Uniforme}\label{alg:AGEU}
\begin{algorithmic}[1]
\Procedure \texttt{AG}($EMax > 0, nelem > 0, pmut \in [0,1]$)
\State Generar una población inicial aleatoria
\State Calcular la función \textit{fitness} de cada individuo
\State \texttt{generacion} = 0
\While{\texttt{generacion} < EMax}
	\State Seleccionar aleatoriamente 4 soluciones de la población sin repetición 2 a 2
	\State Aplicar torneo de 2 en 2 soluciones del conjunto anterior y almacenar solo la mejor $\xrightarrow{}{}$ \texttt{padres}
	\State Generar 2 hijos cruzando \texttt{padres$_1$} y \texttt{padres$_2$}
	\State Aplicar el Operador de Reparación sobre ambos hijos
	\State Calcular la función \textit{fitness} de cada hijo
	\State Almacenar dichos hijos $\xrightarrow{}{}$ \texttt{hijos}
	\State Calcular el número de soluciones a mutar $\xrightarrow{}{} nmut = pmut*nelem$
	\State Mutar $nmut$ soluciones distintas
	\State Calcular la función \textit{fitness} de las nuevas soluciones
	\State Aplicar el Operador de Selección Estacionario sobre la población actual e \texttt{hijos}
	\State \texttt{generacion} = \texttt{generacion}+1
\EndWhile
\EndProcedure
\end{algorithmic}
\end{algorithm}

\section{CHC}

El algoritmo CHC utiliza un método de selección elitista que, combinada con un mecanismo de prevención de incesto y un método para obligar que la población diverja cada vez que converge, permite el mantenimiento de la diversidad de la población. 
Este algoritmo se ha utilizado de forma exitosa en el pasado para problemas de optimización estáticos. 

El algoritmo CHC (\textit{Cross-generational elitist selecition, Heterogeneous recombination and Cataclysmic mutation}) propuesto por Eshelman utiliza un método de selección elitista  combinado con un cruce altamente disruptivo para promover la diversidad de la población. 
La principal característica de este algoritmo es su capacidad de prevenir la convergencia de la población, algo que, como luego comprobaremos, será útil en nuestro problema. 

%En este algoritmo no se necesita de la mutación como era necesario en el AG, esto se debe precisamente a su característica principal, es capaz de forzar la diversidad de la población, por lo que la mutación deja de ser necesaria.

Originalmente, cuando la población converge se pueden tomar dos acciones:
\begin{itemize}
	\item Reiniciar la población entera de forma aleatoria con excepción de la mejor solución
	\item Reiniciar la población utilizando la mejor solución como base y generando el resto realizando modificaciones sobre esta.
\end{itemize}
Sin embargo, esto es solo útil cuando se tienen bastantes evaluaciones. 
En nuestro problema realizar esto resultaría en una pérdida de tiempo e iteraciones importantes, por lo tanto, no se tendrá en cuenta. 

\subsection{Pseudocódigo}

\begin{algorithm}[H]
\caption{Algoritmo CHC}\label{alg:CHC}
\begin{algorithmic}[1]
\Procedure \texttt{AG}($EMax > 0, nelem > 0$)
\State Generar una población inicial aleatoria
\State Calcular la función \textit{fitness} de cada individuo
\State \texttt{generacion} = 0
\State \texttt{threshold} = $n$/4
\While{\texttt{generacion} < EMax}
	\State Calcular el número de parejas a formar para el cruce $\xrightarrow{}{} ncruce = pcruce*nelem$
	\State Desordenar los elementos de la población actual y comprobar si cumplen la condición de prevención de incesto (distancia de Hamming > \texttt{threshold}) de dos en dos
	\If{\texttt{hamming} > \texttt{threshold}}
		\State Almacenar las soluciones $\xrightarrow{}{}$ \texttt{parejas}
	\EndIf
	\If{\texttt{parejas} = $\emptyset$}
		\If{\texttt{threshold} $\neq$ 0}
			\State \texttt{threshold} = \texttt{threshold}-1
		\EndIf
	\Else
		\For{i$\in [0,\texttt{parejas.size()}]$; i=i+2}
			\State Generar 2 hijos cruzando \texttt{parejas}$[i]$ y \texttt{parejas}$[i+1]$
			\State Aplicar el Operador de Reparación sobre ambos hijos
			\State Calcular la función \textit{fitness} de cada hijo
			\State Almacenar dichos hijos $\xrightarrow{}{}$ \texttt{hijos}
		\EndFor
		\State Aplicar el Operador de Selección sobre la población actual e \texttt{hijos}
	\EndIf
	\State \texttt{generacion} = \texttt{generacion}+1
\EndWhile
\EndProcedure
\end{algorithmic}
\end{algorithm}

%
\chapter{Componentes de la propuesta}

En este capítulo se presentan los componentes, es decir, las funcionalidades propias de los algoritmos que se utilizaran como base (presentados en el capítulo anterior). 
Se describirán detalladamente además de usar pseudocódigo para representarlos.

\section{Componentes comunes}

\subsection{Operador de Reparación}

Cuando se realiza el cruce de dos soluciones pueden ocurrir dos casos:
\begin{itemize}
	\item El resultado del cruce pueda seguir considerándose una solución.
	\item El resultado del cruce no constituya el espacio de soluciones.
\end{itemize}

Más adelante en este capítulo se explicarán cómo son los cruces y se entenderá por qué es posible que el resultado de un cruce no sea una solución. 
En este caso, no es lógico desechar al ``hijo'', por lo que debemos ``arreglarlo'' para que se vuelva una solución. 
Ese es el objetivo del Operador de Reparación. 

En nuestro caso lo vamos a aplicar en ambos casos (que el ``hijo'' sea solución o no). 
Si el ``hijo'' no es solución es obvio el por qué necesitamos aplicarlo. 
Si el ``hijo'' es solución lo aplicaremos para asegurarnos que no hay huecos libres, es decir, que no hay elementos adicionales que podrían introducirse adicionalmente. 
Esto último se hace ya que queremos maximizar el valor de la solución, por lo que si queremos que sea mínimamente competente para cuando intentemos introducirlas en la población. 

Entonces seguimos el siguiente proceso, ilustrado en el pseudocódigo \ref{alg:OR}:
\begin{itemize}
	\item Si no es solución (su peso supera al peso máximo): Se deberán eliminar elementos del ``hijo'' hasta que constituya una solución. 
La forma de eliminar elementos será usando Greedy, es decir, se eliminarán los elementos con menor proporción $valor\_acumulado/peso$. 
Esta lógica viene dada por un intento de eliminar el máximo peso posible sin reducir mucho el valor final cuando se vuelva una solución. 
	
	\item Si es solución (su peso no supera al peso máximo): Se buscará, utilizando Greedy, un elemento para introducir. 
En este caso, nos interesa encontrar el elemento con mayor proporción $valor\_acumulado/peso$, ya que eso nos permitiría potencialmente aumentar significativamente el valor total (evaluación de la función \textit{fitness}) de la solución. 
Esto es, al tener en cuenta el peso puede llegar a resultar en que seamos capaces de introducir más elementos, pudiendo superar finalmente el valor que se tendría si se introdujese el elemento con el mayor valor acumulado pero no permitiese introducir más elementos. 
Este proceso se repetirá hasta que no sea capaz de introducir ningún otro elemento en la solución. 
%Además, este caso también se aplica cuando se acaba el caso contrario, para asegurarnos que no se puede aumentar más el valor sin volver a superar el peso máximo. 
\end{itemize}


\begin{algorithm}[H]
\caption{Operador de Reparación}\label{alg:OR}
\begin{algorithmic}[1]
\Procedure \texttt{Operador Reparación}($hijo$)
\State Calcular el peso total de $hijo \xrightarrow{}{}$ \texttt{pesoHijo}
\If{\texttt{pesoHijo} > \texttt{c}}
	\While{\texttt{pesoHijo} > \texttt{c}}
		\State Eliminar elemento usando Greedy
	\EndWhile
\Else
	\State \texttt{anadido = true} 
	\While{\texttt{anadido}}
		\State \texttt{anadido} = Añadir elemento usando Greedy
	\EndWhile
\EndIf
\EndProcedure
\end{algorithmic}
\end{algorithm}

Téngase en cuenta que al eliminar y añadir elementos del ``hijo'', se debe recalcular su peso total.


\section{Componentes de AG}

\subsection{Cruce Uniforme}

El cruce es un operador genético usado para variar los cromosomas de una generación a otra. 
Dos soluciones obtenidas de la población con anterior se cruzarán con el objetivo de producir una descendencia superior. 

Nos encontramos con distintos tipos de cruces básicos:
\begin{itemize}
	\item \textbf{Cruce en un punto}:  Dados dos padres, se le asignan los elementos de \texttt{padre$_1$} a \texttt{hijo$_1$} y de \texttt{padre$_2$} a \texttt{hijo$_2$} hasta cierto cromosoma elegido con anterioridad. 
A partir de dicho cromosoma, cambiaremos la asignación de forma que \texttt{hijo$_1$} hereda de \texttt{padre$_2$} e \texttt{hijo$_2$} hereda de \texttt{padre$_1$}. 
Un ejemplo de este tipo de cruce podría ser:
%Insertar ejemplo
	\item \textbf{Cruce en dos puntos}: Sigue la misma lógica que el anterior, solo que elegimos dos puntos a partir de los cuales se cambian los elementos de qué padre se asignan a cada hijo. 
Un ejemplo de este tipo de cruce podría ser:
%Insertar ejemplo
	\item \textbf{Cruce uniforme}: En este caso, en cada cromosoma se elige de forma aleatoria de qué padre lo hereda, cumpliéndose que si un hijo hereda cierto cromosoma de un padre, el otro hijo deberá heredar el mismo cromosoma del otro padre. 
Un ejemplo de este tipo de cruce podría ser: 
%Insertar ejemplo
\end{itemize}

El cruce de dos soluciones buenas no tiene por qué siempre dar lugar a una solución mejor o igual de buena. 
Sin embargo, si los padres son buenas soluciones, la probabilidad de tener un hijo bueno es elevada; en el caso de que el hijo no sea una buena solución, será eliminado durante el periodo de reemplazo. 

En nuestro problema es posible que el cruce de dos soluciones no de lugar a una solución. 
Esto se debe a la elección aleatoria de qué cromosomas elegir, no estamos teniendo en cuenta el peso que se está alcanzando al asignar cada elemento; por lo que es totalmente posible que  al asignar los elementos a cada hijo se sobrepase la capacidad máxima, dejando por ello de ser una solución válida. 

Más adelante en este capítulo explicaremos otro tipo de cruce, que es el cruce HUX. 

En nuestro caso, realmente utilizamos una mezcla de cruce en un punto y cruce uniforme. 
Esto es, vamos a asignarle a cada hijo la mitad de cada uno de los padres, pero esta asignación será aleatoria: desordenamos el orden de los índices y lo partimos por la mitad. 
Esto viene representado en el pseudocódigo (\ref{alg:CU}) siguiente

\begin{algorithm}[H]
\caption{Cruce Uniforme}\label{alg:CU}
\begin{algorithmic}[1]
\Procedure \texttt{Cruce Uniforme}($padre_1, padre_2$)
\State Desordenar los índices que indican la posición de cada elemento
\For{i in 0..$n$}
	\If{i < $n/2$}
		\State \texttt{hijo$_1$[indice[i]]} = \texttt{padre$_1$[indice[i]]}
		\State \texttt{hijo$_2$[indice[i]]} = \texttt{padre$_2$[indice[i]]}
	\Else
		\State \texttt{hijo$_1$[indice[i]]} = \texttt{padre$_2$[indice[i]]}
		\State \texttt{hijo$_2$[indice[i]]} = \texttt{padre$_1$[indice[i]]}
	\EndIf
\EndFor
\EndProcedure
\end{algorithmic}
\end{algorithm}

\subsection{Mutación}

Modifica al azar una muy pequeña parte del cromosoma de los individuos, y permite alcanzar zonas del espacio de búsqueda que no estaban cubiertas por los individuos de la población actual. 
La mutación sola de por sí generalmente no permite avanzar en la búsqueda de una solución, pero nos garantiza que la población no va a evolucionar hacia una población uniforme que no sea capaz de seguir evolucionando. 




\subsection{Operador de Reemplazo Estacionario}

\section{Componentes de CHC}

\subsection{Cruce HUX}

\subsection{Enfrentamiento}
%
\chapter{Parte Experimental (In progress)}


\newpage
Adicionalmente, como se ha comentado antes, cada algoritmo requiere de sus propios parámetros. 
Aunque explicaremos en más profundidad cada algoritmo en los siguientes capítulos, es necesario que establezcamos ahora cuáles son dichos parámetros y qué valores se han utilizado. 
Para ello, primero vamos a resumir en la siguiente tabla (\ref{table:Parametros}) qué parámetros usa cada algoritmo y después explicaremos qué es cada uno y qué valor tienen asignado.

\begin{table}[H]
\label{table:Parametros}
\caption{Parámetros utilizados por cada algoritmo}
\begin{tabular}{|l|l|l|l|l|}
\hline
\rowcolor[HTML]{F7EAC7} 
Algoritmo          & EvaluacionMAX & Semilla     & Nº Elementos & Probabilidad mutación \\ \hline
\rowcolor[HTML]{DAE8FC} 
\textbf{Random}    & \textbf{Sí}   & \textbf{Sí} & No           & No                    \\ \hline
\rowcolor[HTML]{DDFDFF} 
\textbf{AGEU}      & \textbf{Sí}   & \textbf{Sí} & \textbf{Sí}  & \textbf{Sí}           \\ \hline
\rowcolor[HTML]{DAE8FC} 
\textbf{GACEP}     & \textbf{Sí}   & \textbf{Sí} & \textbf{Sí}  & \textbf{Sí}           \\ \hline
\rowcolor[HTML]{DDFDFF} 
\textbf{CHC}       & \textbf{Sí}   & \textbf{Sí} & \textbf{Sí}  & No                    \\ \hline
\rowcolor[HTML]{DAE8FC} 
\textbf{GACEPCHC}  & \textbf{Sí}   & \textbf{Sí} & \textbf{Sí}  & \textbf{Sí}           \\ \hline
\rowcolor[HTML]{DDFDFF} 
\textbf{GACEP3103} & \textbf{Sí}   & \textbf{Sí} & \textbf{Sí}  & \textbf{Sí}           \\ \hline
\end{tabular}
\end{table}

\begin{itemize}
	\item \texttt{EvaluacionMAX}: Es el número de evaluaciones máximas para dicho algoritmo. 
Su valor será el de \texttt{NEVALUACIONESMAX}.
	\item \texttt{Semilla}: Como se ha indicado anteriormente, es necesario establecer una semilla de aleatoriedad en cada ejecución del algoritmo. 
Por lo tanto, el valor de este parámetro será el de la semilla generada por \texttt{INITSEED}. 
	\item \texttt{Nº Elementos}: Número de soluciones que va a constituir una población. 
Es necesario tener una población pequeña con el fin de utilizar el menor número de evaluaciones posible, pero suficiente como para poder trabajar con cierto margen. 
Por lo tanto, estableceremos el tamaño de población a 10. 
	\item \texttt{Probabilidad mutación}: En algunos algoritmos intentaremos modificar un poco alguna solución en cada iteración. 
Este valor establece el porcentaje de soluciones que la población que mutará. 
Genéricamente este valor suele ser del 0.1, por lo que también lo utilizaremos en nuestro problema. 
Correspondería que solo mutaría una solución de la población por cada generación. 
\end{itemize}
\section{Algoritmos de Referencia experimentación}

\section{Resultados versión expensive}

\section{Incorporación del histórico}

\section{Uso de GRASP}

\section{Operador de Cruce Intensivo}

\section{Estudio de la diversidad}

\section{Incrementando la diversidad (con nuevo reemplazo)}

\section{Tablas de prueba}


% Please add the following required packages to your document preamble:
% \usepackage[table,xcdraw]{xcolor}
% If you use beamer only pass "xcolor=table" option, i.e. \documentclass[xcolor=table]{beamer}
\begin{table}[]
\begin{tabular}{r|
>{\columncolor[HTML]{D3FFB6}}r |r|}
\cline{2-3}
\multicolumn{1}{l|}{}                             & \multicolumn{1}{l|}{\cellcolor[HTML]{FFFFC7}GACEPwGRASP} & \multicolumn{1}{l|}{\cellcolor[HTML]{FFFFC7}GACEPwoGRASP} \\ \hline
\multicolumn{1}{|r|}{\cellcolor[HTML]{FCE6AB}1}   & \textbf{1.309278}                                        & 1.690722                                                  \\ \hline
\multicolumn{1}{|r|}{\cellcolor[HTML]{FCE6AB}2}   & \textbf{1.154639}                                        & 1.845361                                                  \\ \hline
\multicolumn{1}{|r|}{\cellcolor[HTML]{FCE6AB}3}   & \textbf{1.185567}                                        & 1.814433                                                  \\ \hline
\multicolumn{1}{|r|}{\cellcolor[HTML]{FCE6AB}5}   & \textbf{1.154639}                                        & 1.845361                                                  \\ \hline
\multicolumn{1}{|r|}{\cellcolor[HTML]{FCE6AB}10}  & \textbf{1.051546}                                        & 1.948454                                                  \\ \hline
\multicolumn{1}{|r|}{\cellcolor[HTML]{FCE6AB}20}  & \textbf{1.092784}                                        & 1.907216                                                  \\ \hline
\multicolumn{1}{|r|}{\cellcolor[HTML]{FCE6AB}30}  & \textbf{1.123711}                                        & 1.876289                                                  \\ \hline
\multicolumn{1}{|r|}{\cellcolor[HTML]{FCE6AB}40}  & \textbf{1.061856}                                        & 1.938144                                                  \\ \hline
\multicolumn{1}{|r|}{\cellcolor[HTML]{FCE6AB}50}  & \textbf{1.082474}                                        & 1.917526                                                  \\ \hline
\multicolumn{1}{|r|}{\cellcolor[HTML]{FCE6AB}60}  & \textbf{1.072165}                                        & 1.927835                                                  \\ \hline
\multicolumn{1}{|r|}{\cellcolor[HTML]{FCE6AB}70}  & \textbf{1.103093}                                        & 1.896907                                                  \\ \hline
\multicolumn{1}{|r|}{\cellcolor[HTML]{FCE6AB}80}  & \textbf{1.103093}                                        & 1.896907                                                  \\ \hline
\multicolumn{1}{|r|}{\cellcolor[HTML]{FCE6AB}90}  & \textbf{1.103093}                                        & 1.896907                                                  \\ \hline
\multicolumn{1}{|r|}{\cellcolor[HTML]{FCE6AB}100} & \textbf{1.103093}                                        & 1.896907                                                  \\ \hline
\end{tabular}
\caption{Solo Ranking}
\end{table}

% Please add the following required packages to your document preamble:
% \usepackage[table,xcdraw]{xcolor}
% If you use beamer only pass "xcolor=table" option, i.e. \documentclass[xcolor=table]{beamer}
\begin{table}[]
\begin{tabular}{l|r|r|}
\cline{2-3}
                                                   & \multicolumn{1}{l|}{\cellcolor[HTML]{FFFFC7}GACEPwGRASP} & \multicolumn{1}{l|}{\cellcolor[HTML]{FFFFC7}GACEPwoGRASP} \\ \hline
\multicolumn{1}{|l|}{\cellcolor[HTML]{FCE6AB}F01}  & 203870                                                   & \cellcolor[HTML]{D3FFB6}\textbf{203917}                   \\ \hline
\multicolumn{1}{|l|}{\cellcolor[HTML]{FCE6AB}F02}  & \cellcolor[HTML]{D3FFB6}\textbf{88229.5}                 & 86893.8                                                   \\ \hline
\multicolumn{1}{|l|}{\cellcolor[HTML]{FCE6AB}F03}  & \cellcolor[HTML]{D3FFB6}\textbf{202940}                  & 202056                                                    \\ \hline
\multicolumn{1}{|l|}{\cellcolor[HTML]{FCE6AB}F04}  & \cellcolor[HTML]{D3FFB6}\textbf{235610}                  & 234528                                                    \\ \hline
\multicolumn{1}{|l|}{\cellcolor[HTML]{FCE6AB}F05}  & 238439                                                   & \cellcolor[HTML]{D3FFB6}\textbf{238931}                   \\ \hline
\multicolumn{1}{|l|}{\cellcolor[HTML]{FCE6AB}F06}  & \cellcolor[HTML]{D3FFB6}\textbf{81325.7}                 & 78698.4                                                   \\ \hline
\multicolumn{1}{|l|}{\cellcolor[HTML]{FCE6AB}F07}  & \cellcolor[HTML]{D3FFB6}\textbf{10867.2}                 & 10611.2                                                   \\ \hline
\multicolumn{1}{|l|}{\cellcolor[HTML]{FCE6AB}F08}  & \cellcolor[HTML]{D3FFB6}\textbf{67669.3}                 & 67007.1                                                   \\ \hline
\multicolumn{1}{|l|}{\cellcolor[HTML]{FCE6AB}F09}  & 239982                                                   & \cellcolor[HTML]{D3FFB6}\textbf{241977}                   \\ \hline
\multicolumn{1}{|l|}{\cellcolor[HTML]{FCE6AB}F10}  & \cellcolor[HTML]{D3FFB6}\textbf{26642.1}                 & 26107.5                                                   \\ \hline
\multicolumn{1}{|l|}{\cellcolor[HTML]{FCE6AB}F11}  & \cellcolor[HTML]{D3FFB6}\textbf{20217.6}                 & 20195.7                                                   \\ \hline
\multicolumn{1}{|l|}{\cellcolor[HTML]{FCE6AB}F12}  & \cellcolor[HTML]{D3FFB6}\textbf{58173.8}                 & 58108.8                                                   \\ \hline
\multicolumn{1}{|l|}{\cellcolor[HTML]{FCE6AB}F13}  & \cellcolor[HTML]{D3FFB6}\textbf{3967.14}                 & 3841.06                                                   \\ \hline
\multicolumn{1}{|l|}{\cellcolor[HTML]{FCE6AB}F14}  & \cellcolor[HTML]{D3FFB6}\textbf{53183.3}                 & 52534.5                                                   \\ \hline
\multicolumn{1}{|l|}{\cellcolor[HTML]{FCE6AB}F15}  & 62310.9                                                  & \cellcolor[HTML]{D3FFB6}\textbf{62661.5}                  \\ \hline
\multicolumn{1}{|l|}{\cellcolor[HTML]{FCE6AB}F16}  & \cellcolor[HTML]{D3FFB6}\textbf{38963}                   & 38120.5                                                   \\ \hline
\multicolumn{1}{|l|}{\cellcolor[HTML]{FCE6AB}F17}  & \cellcolor[HTML]{D3FFB6}\textbf{15641.2}                 & 15454.7                                                   \\ \hline
\multicolumn{1}{|l|}{\cellcolor[HTML]{FCE6AB}F18}  & \cellcolor[HTML]{D3FFB6}\textbf{22181.6}                 & 21423.6                                                   \\ \hline
\multicolumn{1}{|l|}{\cellcolor[HTML]{FCE6AB}F19}  & \cellcolor[HTML]{D3FFB6}\textbf{37575.9}                 & 36782.8                                                   \\ \hline
\multicolumn{1}{|l|}{\cellcolor[HTML]{FCE6AB}F20}  & 94259.5                                                  & \cellcolor[HTML]{D3FFB6}\textbf{94302.4}                  \\ \hline
\multicolumn{1}{|l|}{\cellcolor[HTML]{FCE6AB}F21}  & \cellcolor[HTML]{D3FFB6}\textbf{89232.8}                 & 88189.9                                                   \\ \hline
\multicolumn{1}{|l|}{\cellcolor[HTML]{FCE6AB}F22}  & \cellcolor[HTML]{D3FFB6}\textbf{110642}                  & 108332                                                    \\ \hline
\multicolumn{1}{|l|}{\cellcolor[HTML]{FCE6AB}F23}  & \cellcolor[HTML]{D3FFB6}\textbf{37247.4}                 & 36961.9                                                   \\ \hline
\multicolumn{1}{|l|}{\cellcolor[HTML]{FCE6AB}F24}  & 110026                                                   & \cellcolor[HTML]{D3FFB6}\textbf{110489}                   \\ \hline
\multicolumn{1}{|l|}{\cellcolor[HTML]{FCE6AB}F25}  & \cellcolor[HTML]{D3FFB6}\textbf{60821.5}                 & 57749                                                     \\ \hline
\multicolumn{1}{|l|}{\cellcolor[HTML]{FCE6AB}F26}  & \cellcolor[HTML]{D3FFB6}\textbf{18066.4}                 & 17239.7                                                   \\ \hline
\multicolumn{1}{|l|}{\cellcolor[HTML]{FCE6AB}F27}  & \cellcolor[HTML]{D3FFB6}\textbf{55937.8}                 & 55209.6                                                   \\ \hline
\multicolumn{1}{|l|}{\cellcolor[HTML]{FCE6AB}F28}  & \cellcolor[HTML]{D3FFB6}\textbf{58848.7}                 & 57800.7                                                   \\ \hline
\multicolumn{1}{|l|}{\cellcolor[HTML]{FCE6AB}F29}  & \cellcolor[HTML]{D3FFB6}\textbf{73569.8}                 & 72183.1                                                   \\ \hline
\multicolumn{1}{|l|}{\cellcolor[HTML]{FCE6AB}F30}  & \cellcolor[HTML]{D3FFB6}\textbf{151250}                  & 151073                                                    \\ \hline
\multicolumn{1}{|l|}{\cellcolor[HTML]{FCE6AB}F31}  & 191323                                                   & \cellcolor[HTML]{D3FFB6}\textbf{191744}                   \\ \hline
\multicolumn{1}{|l|}{\cellcolor[HTML]{FCE6AB}F32}  & \cellcolor[HTML]{D3FFB6}\textbf{104160}                  & 101933                                                    \\ \hline
\multicolumn{1}{|l|}{\cellcolor[HTML]{FCE6AB}F33}  & \cellcolor[HTML]{D3FFB6}\textbf{66522.7}                 & 65749.4                                                   \\ \hline
\multicolumn{1}{|l|}{\cellcolor[HTML]{FCE6AB}F34}  & \cellcolor[HTML]{D3FFB6}\textbf{80481.7}                 & 78826.7                                                   \\ \hline
\multicolumn{1}{|l|}{\cellcolor[HTML]{FCE6AB}F35}  & \cellcolor[HTML]{D3FFB6}\textbf{31122.7}                 & 30323.9                                                   \\ \hline
\multicolumn{1}{|l|}{\cellcolor[HTML]{FCE6AB}F36}  & \cellcolor[HTML]{D3FFB6}\textbf{153157}                  & 150580                                                    \\ \hline
\multicolumn{1}{|l|}{\cellcolor[HTML]{FCE6AB}F37}  & \cellcolor[HTML]{D3FFB6}\textbf{120369}                  & 118139                                                    \\ \hline
\multicolumn{1}{|l|}{\cellcolor[HTML]{FCE6AB}F38}  & \cellcolor[HTML]{D3FFB6}\textbf{21816.6}                 & 21744.5                                                   \\ \hline
\multicolumn{1}{|l|}{\cellcolor[HTML]{FCE6AB}F39}  & \cellcolor[HTML]{D3FFB6}\textbf{112427}                  & 111748                                                    \\ \hline
\multicolumn{1}{|l|}{\cellcolor[HTML]{FCE6AB}F40}  & \cellcolor[HTML]{D3FFB6}\textbf{379570}                  & 371906                                                    \\ \hline
\multicolumn{1}{|l|}{\cellcolor[HTML]{FCE6AB}F41}  & 958934                                                   & \cellcolor[HTML]{D3FFB6}\textbf{960695}                   \\ \hline
\multicolumn{1}{|l|}{\cellcolor[HTML]{FCE6AB}F42}  & \cellcolor[HTML]{D3FFB6}\textbf{315558}                  & 310368                                                    \\ \hline
\multicolumn{1}{|l|}{\cellcolor[HTML]{FCE6AB}F43}  & \cellcolor[HTML]{D3FFB6}\textbf{32375.5}                 & 31956                                                     \\ \hline
\multicolumn{1}{|l|}{\cellcolor[HTML]{FCE6AB}F44}  & 106229                                                   & \cellcolor[HTML]{D3FFB6}\textbf{107874}                   \\ \hline
\multicolumn{1}{|l|}{\cellcolor[HTML]{FCE6AB}F45}  & \cellcolor[HTML]{D3FFB6}\textbf{813622}                  & 804278                                                    \\ \hline
\multicolumn{1}{|l|}{\cellcolor[HTML]{FCE6AB}F46}  & \cellcolor[HTML]{D3FFB6}\textbf{44144.1}                 & 43930.7                                                   \\ \hline
\multicolumn{1}{|l|}{\cellcolor[HTML]{FCE6AB}F47}  & \cellcolor[HTML]{D3FFB6}\textbf{726019}                  & 711466                                                    \\ \hline
\multicolumn{1}{|l|}{\cellcolor[HTML]{FCE6AB}F48}  & \cellcolor[HTML]{D3FFB6}\textbf{813949}                  & 805644                                                    \\ \hline
\multicolumn{1}{|l|}{\cellcolor[HTML]{FCE6AB}F49}  & \cellcolor[HTML]{D3FFB6}\textbf{653176}                  & 639675                                                    \\ \hline
\multicolumn{1}{|l|}{\cellcolor[HTML]{FCE6AB}F50}  & \cellcolor[HTML]{D3FFB6}\textbf{50543}                   & 49255.1                                                   \\ \hline
\multicolumn{1}{|l|}{\cellcolor[HTML]{FCE6AB}F51}  & \cellcolor[HTML]{D3FFB6}\textbf{211429}                  & 206394                                                    \\ \hline
\multicolumn{1}{|l|}{\cellcolor[HTML]{FCE6AB}F52}  & \cellcolor[HTML]{D3FFB6}\textbf{244518}                  & 243532                                                    \\ \hline
\multicolumn{1}{|l|}{\cellcolor[HTML]{FCE6AB}F53}  & \cellcolor[HTML]{D3FFB6}\textbf{227361}                  & 226204                                                    \\ \hline
\multicolumn{1}{|l|}{\cellcolor[HTML]{FCE6AB}F54}  & \cellcolor[HTML]{D3FFB6}\textbf{193634}                  & 189500                                                    \\ \hline
\multicolumn{1}{|l|}{\cellcolor[HTML]{FCE6AB}F55}  & \cellcolor[HTML]{D3FFB6}\textbf{81477.9}                 & 78811.9                                                   \\ \hline
\multicolumn{1}{|l|}{\cellcolor[HTML]{FCE6AB}F56}  & \cellcolor[HTML]{D3FFB6}\textbf{61504}                   & 59037                                                     \\ \hline
\multicolumn{1}{|l|}{\cellcolor[HTML]{FCE6AB}F57}  & \cellcolor[HTML]{D3FFB6}\textbf{150655}                  & 146471                                                    \\ \hline
\multicolumn{1}{|l|}{\cellcolor[HTML]{FCE6AB}F58}  & \cellcolor[HTML]{D3FFB6}\textbf{51459}                   & 49706.3                                                   \\ \hline
\multicolumn{1}{|l|}{\cellcolor[HTML]{FCE6AB}F59}  & \cellcolor[HTML]{D3FFB6}\textbf{287285}                  & 283170                                                    \\ \hline
\multicolumn{1}{|l|}{\cellcolor[HTML]{FCE6AB}F60}  & \cellcolor[HTML]{D3FFB6}\textbf{383329}                  & 374933                                                    \\ \hline
\multicolumn{1}{|l|}{\cellcolor[HTML]{FCE6AB}F61}  & \cellcolor[HTML]{D3FFB6}\textbf{214848}                  & 205342                                                    \\ \hline
\multicolumn{1}{|l|}{\cellcolor[HTML]{FCE6AB}F62}  & \cellcolor[HTML]{D3FFB6}\textbf{225192}                  & 219207                                                    \\ \hline
\multicolumn{1}{|l|}{\cellcolor[HTML]{FCE6AB}F63}  & \cellcolor[HTML]{D3FFB6}\textbf{225684}                  & 220104                                                    \\ \hline
\multicolumn{1}{|l|}{\cellcolor[HTML]{FCE6AB}F64}  & 489291                                                   & \cellcolor[HTML]{D3FFB6}\textbf{490903}                   \\ \hline
\multicolumn{1}{|l|}{\cellcolor[HTML]{FCE6AB}F65}  & \cellcolor[HTML]{D3FFB6}\textbf{440828}                  & 439020                                                    \\ \hline
\multicolumn{1}{|l|}{\cellcolor[HTML]{FCE6AB}F66}  & \cellcolor[HTML]{D3FFB6}\textbf{220101}                  & 214588                                                    \\ \hline
\multicolumn{1}{|l|}{\cellcolor[HTML]{FCE6AB}F67}  & \cellcolor[HTML]{D3FFB6}\textbf{324902}                  & 318589                                                    \\ \hline
\multicolumn{1}{|l|}{\cellcolor[HTML]{FCE6AB}F68}  & \cellcolor[HTML]{D3FFB6}\textbf{110572}                  & 108423                                                    \\ \hline
\multicolumn{1}{|l|}{\cellcolor[HTML]{FCE6AB}F69}  & \cellcolor[HTML]{D3FFB6}\textbf{150784}                  & 148614                                                    \\ \hline
\multicolumn{1}{|l|}{\cellcolor[HTML]{FCE6AB}F70}  & \cellcolor[HTML]{D3FFB6}\textbf{452607}                  & 445441                                                    \\ \hline
\multicolumn{1}{|l|}{\cellcolor[HTML]{FCE6AB}F71}  & \cellcolor[HTML]{D3FFB6}\textbf{281795}                  & 277919                                                    \\ \hline
\multicolumn{1}{|l|}{\cellcolor[HTML]{FCE6AB}F72}  & \cellcolor[HTML]{D3FFB6}\textbf{66610.6}                 & 65806.9                                                   \\ \hline
\multicolumn{1}{|l|}{\cellcolor[HTML]{FCE6AB}F73}  & \cellcolor[HTML]{D3FFB6}\textbf{133906}                  & 133560                                                    \\ \hline
\multicolumn{1}{|l|}{\cellcolor[HTML]{FCE6AB}F74}  & \cellcolor[HTML]{D3FFB6}\textbf{146870}                  & 144762                                                    \\ \hline
\multicolumn{1}{|l|}{\cellcolor[HTML]{FCE6AB}F75}  & \cellcolor[HTML]{D3FFB6}\textbf{234757}                  & 232794                                                    \\ \hline
\multicolumn{1}{|l|}{\cellcolor[HTML]{FCE6AB}F76}  & \cellcolor[HTML]{D3FFB6}\textbf{275504}                  & 269788                                                    \\ \hline
\multicolumn{1}{|l|}{\cellcolor[HTML]{FCE6AB}F77}  & \cellcolor[HTML]{D3FFB6}\textbf{620630}                  & 616128                                                    \\ \hline
\multicolumn{1}{|l|}{\cellcolor[HTML]{FCE6AB}F78}  & \cellcolor[HTML]{D3FFB6}\textbf{534844}                  & 525471                                                    \\ \hline
\multicolumn{1}{|l|}{\cellcolor[HTML]{FCE6AB}F79}  & \cellcolor[HTML]{D3FFB6}\textbf{379471}                  & 370202                                                    \\ \hline
\multicolumn{1}{|l|}{\cellcolor[HTML]{FCE6AB}F80}  & \cellcolor[HTML]{D3FFB6}\textbf{30802.7}                 & 30668.8                                                   \\ \hline
\multicolumn{1}{|l|}{\cellcolor[HTML]{FCE6AB}F81}  & \cellcolor[HTML]{D3FFB6}\textbf{269072}                  & 261697                                                    \\ \hline
\multicolumn{1}{|l|}{\cellcolor[HTML]{FCE6AB}F82}  & \cellcolor[HTML]{D3FFB6}\textbf{453902}                  & 441509                                                    \\ \hline
\multicolumn{1}{|l|}{\cellcolor[HTML]{FCE6AB}F83}  & \cellcolor[HTML]{D3FFB6}\textbf{15967.4}                 & 15628.8                                                   \\ \hline
\multicolumn{1}{|l|}{\cellcolor[HTML]{FCE6AB}F84}  & \cellcolor[HTML]{D3FFB6}\textbf{253405}                  & 245342                                                    \\ \hline
\multicolumn{1}{|l|}{\cellcolor[HTML]{FCE6AB}F85}  & \cellcolor[HTML]{D3FFB6}\textbf{494802}                  & 487566                                                    \\ \hline
\multicolumn{1}{|l|}{\cellcolor[HTML]{FCE6AB}F86}  & \cellcolor[HTML]{D3FFB6}\textbf{9783.9}                  & 9749.8                                                    \\ \hline
\multicolumn{1}{|l|}{\cellcolor[HTML]{FCE6AB}F87}  & \cellcolor[HTML]{D3FFB6}\textbf{237551}                  & 226145                                                    \\ \hline
\multicolumn{1}{|l|}{\cellcolor[HTML]{FCE6AB}F88}  & \cellcolor[HTML]{D3FFB6}\textbf{1017990}                 & 1009100                                                   \\ \hline
\multicolumn{1}{|l|}{\cellcolor[HTML]{FCE6AB}F89}  & \cellcolor[HTML]{D3FFB6}\textbf{483693}                  & 469319                                                    \\ \hline
\multicolumn{1}{|l|}{\cellcolor[HTML]{FCE6AB}F90}  & \cellcolor[HTML]{D3FFB6}\textbf{110079}                  & 105789                                                    \\ \hline
\multicolumn{1}{|l|}{\cellcolor[HTML]{FCE6AB}F91}  & \cellcolor[HTML]{D3FFB6}\textbf{890216}                  & 873907                                                    \\ \hline
\multicolumn{1}{|l|}{\cellcolor[HTML]{FCE6AB}F92}  & \cellcolor[HTML]{D3FFB6}\textbf{310174}                  & 295377                                                    \\ \hline
\multicolumn{1}{|l|}{\cellcolor[HTML]{FCE6AB}F93}  & \cellcolor[HTML]{D3FFB6}\textbf{722245}                  & 706744                                                    \\ \hline
\multicolumn{1}{|l|}{\cellcolor[HTML]{FCE6AB}F94}  & \cellcolor[HTML]{D3FFB6}\textbf{738970}                  & 719059                                                    \\ \hline
\multicolumn{1}{|l|}{\cellcolor[HTML]{FCE6AB}F95}  & \cellcolor[HTML]{D3FFB6}\textbf{46951.9}                 & 46737.7                                                   \\ \hline
\multicolumn{1}{|l|}{\cellcolor[HTML]{FCE6AB}F96}  & \cellcolor[HTML]{D3FFB6}\textbf{770539}                  & 751434                                                    \\ \hline
\multicolumn{1}{|l|}{\cellcolor[HTML]{FCE6AB}F97}  & \cellcolor[HTML]{D3FFB6}\textbf{766150}                  & 752530                                                    \\ \hline
\multicolumn{1}{|l|}{\cellcolor[HTML]{FFFFC7}Best} & \multicolumn{1}{c|}{\cellcolor[HTML]{D3FFB6}\textbf{87}} & \multicolumn{1}{c|}{10}                                   \\ \hline
\end{tabular}
\caption{Solo valores}
\end{table}

% Please add the following required packages to your document preamble:
% \usepackage[table,xcdraw]{xcolor}
% If you use beamer only pass "xcolor=table" option, i.e. \documentclass[xcolor=table]{beamer}
\begin{table}[]
\begin{tabular}{lrr|}
\cline{2-3}
\multicolumn{1}{l|}{}                             & \multicolumn{1}{l|}{\cellcolor[HTML]{FFFFC7}GACEPwGRASP}      & \multicolumn{1}{l|}{\cellcolor[HTML]{FFFFC7}GACEPwoGRASP} \\ \hline
\multicolumn{1}{|l|}{\cellcolor[HTML]{FCE6AB}F01} & \multicolumn{1}{r|}{203870}                                   & \cellcolor[HTML]{D3FFB6}\textbf{203917}                   \\ \hline
\multicolumn{1}{|l|}{\cellcolor[HTML]{FCE6AB}F02} & \multicolumn{1}{r|}{\cellcolor[HTML]{D3FFB6}\textbf{88229.5}} & 86893.8                                                   \\ \hline
\multicolumn{1}{|l|}{\cellcolor[HTML]{FCE6AB}F03} & \multicolumn{1}{r|}{\cellcolor[HTML]{D3FFB6}\textbf{202940}}  & 202056                                                    \\ \hline
\multicolumn{1}{|l|}{\cellcolor[HTML]{FCE6AB}F04} & \multicolumn{1}{r|}{\cellcolor[HTML]{D3FFB6}\textbf{235610}}  & 234528                                                    \\ \hline
\multicolumn{1}{|l|}{\cellcolor[HTML]{FCE6AB}F05} & \multicolumn{1}{r|}{238439}                                   & \cellcolor[HTML]{D3FFB6}\textbf{238931}                   \\ \hline
\multicolumn{1}{|l|}{\cellcolor[HTML]{FCE6AB}F06} & \multicolumn{1}{r|}{\cellcolor[HTML]{D3FFB6}\textbf{81325.7}} & 78698.4                                                   \\ \hline
\multicolumn{1}{|l|}{\cellcolor[HTML]{FCE6AB}F07} & \multicolumn{1}{r|}{\cellcolor[HTML]{D3FFB6}\textbf{10867.2}} & 10611.2                                                   \\ \hline
\multicolumn{1}{|l|}{\cellcolor[HTML]{FCE6AB}F08} & \multicolumn{1}{r|}{\cellcolor[HTML]{D3FFB6}\textbf{67669.3}} & 67007.1                                                   \\ \hline
\multicolumn{1}{|l|}{\cellcolor[HTML]{FCE6AB}F09} & \multicolumn{1}{r|}{239982}                                   & \cellcolor[HTML]{D3FFB6}\textbf{241977}                   \\ \hline
\multicolumn{1}{|l|}{\cellcolor[HTML]{FCE6AB}F10} & \multicolumn{1}{r|}{\cellcolor[HTML]{D3FFB6}\textbf{26642.1}} & 26107.5                                                   \\ \hline
\multicolumn{1}{|l|}{\cellcolor[HTML]{FCE6AB}F11} & \multicolumn{1}{r|}{\cellcolor[HTML]{D3FFB6}\textbf{20217.6}} & 20195.7                                                   \\ \hline
\multicolumn{1}{|l|}{\cellcolor[HTML]{FCE6AB}F12} & \multicolumn{1}{r|}{\cellcolor[HTML]{D3FFB6}\textbf{58173.8}} & 58108.8                                                   \\ \hline
\multicolumn{1}{|l|}{\cellcolor[HTML]{FCE6AB}F13} & \multicolumn{1}{r|}{\cellcolor[HTML]{D3FFB6}\textbf{3967.14}} & 3841.06                                                   \\ \hline
\multicolumn{1}{|l|}{\cellcolor[HTML]{FCE6AB}F14} & \multicolumn{1}{r|}{\cellcolor[HTML]{D3FFB6}\textbf{53183.3}} & 52534.5                                                   \\ \hline
\multicolumn{1}{|l|}{\cellcolor[HTML]{FCE6AB}F15} & \multicolumn{1}{r|}{62310.9}                                  & \cellcolor[HTML]{D3FFB6}\textbf{62661.5}                  \\ \hline
\multicolumn{1}{|l|}{\cellcolor[HTML]{FCE6AB}F16} & \multicolumn{1}{r|}{\cellcolor[HTML]{D3FFB6}\textbf{38963}}   & 38120.5                                                   \\ \hline
\multicolumn{1}{|l|}{\cellcolor[HTML]{FCE6AB}F17} & \multicolumn{1}{r|}{\cellcolor[HTML]{D3FFB6}\textbf{15641.2}} & 15454.7                                                   \\ \hline
\multicolumn{1}{|l|}{\cellcolor[HTML]{FCE6AB}F18} & \multicolumn{1}{r|}{\cellcolor[HTML]{D3FFB6}\textbf{22181.6}} & 21423.6                                                   \\ \hline
\multicolumn{1}{|l|}{\cellcolor[HTML]{FCE6AB}F19} & \multicolumn{1}{r|}{\cellcolor[HTML]{D3FFB6}\textbf{37575.9}} & 36782.8                                                   \\ \hline
\multicolumn{1}{|l|}{\cellcolor[HTML]{FCE6AB}F20} & \multicolumn{1}{r|}{94259.5}                                  & \cellcolor[HTML]{D3FFB6}\textbf{94302.4}                  \\ \hline
\multicolumn{1}{|l|}{\cellcolor[HTML]{FCE6AB}F21} & \multicolumn{1}{r|}{\cellcolor[HTML]{D3FFB6}\textbf{89232.8}} & 88189.9                                                   \\ \hline
\multicolumn{1}{|l|}{\cellcolor[HTML]{FCE6AB}F22} & \multicolumn{1}{r|}{\cellcolor[HTML]{D3FFB6}\textbf{110642}}  & 108332                                                    \\ \hline
\multicolumn{1}{|l|}{\cellcolor[HTML]{FCE6AB}F23} & \multicolumn{1}{r|}{\cellcolor[HTML]{D3FFB6}\textbf{37247.4}} & 36961.9                                                   \\ \hline
\multicolumn{1}{|l|}{\cellcolor[HTML]{FCE6AB}F24} & \multicolumn{1}{r|}{110026}                                   & \cellcolor[HTML]{D3FFB6}\textbf{110489}                   \\ \hline
\multicolumn{1}{|l|}{\cellcolor[HTML]{FCE6AB}F25} & \multicolumn{1}{r|}{\cellcolor[HTML]{D3FFB6}\textbf{60821.5}} & 57749                                                     \\ \hline
\multicolumn{1}{|l|}{\cellcolor[HTML]{FCE6AB}F26} & \multicolumn{1}{r|}{\cellcolor[HTML]{D3FFB6}\textbf{18066.4}} & 17239.7                                                   \\ \hline
\multicolumn{1}{|l|}{\cellcolor[HTML]{FCE6AB}F27} & \multicolumn{1}{r|}{\cellcolor[HTML]{D3FFB6}\textbf{55937.8}} & 55209.6                                                   \\ \hline
\multicolumn{1}{|l|}{\cellcolor[HTML]{FCE6AB}F28} & \multicolumn{1}{r|}{\cellcolor[HTML]{D3FFB6}\textbf{58848.7}} & 57800.7                                                   \\ \hline
\multicolumn{1}{|l|}{\cellcolor[HTML]{FCE6AB}F29} & \multicolumn{1}{r|}{\cellcolor[HTML]{D3FFB6}\textbf{73569.8}} & 72183.1                                                   \\ \hline
\multicolumn{1}{|l|}{\cellcolor[HTML]{FCE6AB}F30} & \multicolumn{1}{r|}{\cellcolor[HTML]{D3FFB6}\textbf{151250}}  & 151073                                                    \\ \hline
\multicolumn{1}{|l|}{\cellcolor[HTML]{FCE6AB}F31} & \multicolumn{1}{r|}{191323}                                   & \cellcolor[HTML]{D3FFB6}\textbf{191744}                   \\ \hline
\multicolumn{1}{|l|}{\cellcolor[HTML]{FCE6AB}F32} & \multicolumn{1}{r|}{\cellcolor[HTML]{D3FFB6}\textbf{104160}}  & 101933                                                    \\ \hline
\multicolumn{1}{|l|}{\cellcolor[HTML]{FCE6AB}F33} & \multicolumn{1}{r|}{\cellcolor[HTML]{D3FFB6}\textbf{66522.7}} & 65749.4                                                   \\ \hline
\multicolumn{1}{|l|}{\cellcolor[HTML]{FCE6AB}F34} & \multicolumn{1}{r|}{\cellcolor[HTML]{D3FFB6}\textbf{80481.7}} & 78826.7                                                   \\ \hline
\multicolumn{1}{|l|}{\cellcolor[HTML]{FCE6AB}F35} & \multicolumn{1}{r|}{\cellcolor[HTML]{D3FFB6}\textbf{31122.7}} & 30323.9                                                   \\ \hline
\multicolumn{1}{|l|}{\cellcolor[HTML]{FCE6AB}F36} & \multicolumn{1}{r|}{\cellcolor[HTML]{D3FFB6}\textbf{153157}}  & 150580                                                    \\ \hline
\multicolumn{1}{|l|}{\cellcolor[HTML]{FCE6AB}F37} & \multicolumn{1}{r|}{\cellcolor[HTML]{D3FFB6}\textbf{120369}}  & 118139                                                    \\ \hline
\multicolumn{1}{|l|}{\cellcolor[HTML]{FCE6AB}F38} & \multicolumn{1}{r|}{\cellcolor[HTML]{D3FFB6}\textbf{21816.6}} & 21744.5                                                   \\ \hline
\multicolumn{1}{|l|}{\cellcolor[HTML]{FCE6AB}F39} & \multicolumn{1}{r|}{\cellcolor[HTML]{D3FFB6}\textbf{112427}}  & 111748                                                    \\ \hline
\multicolumn{1}{|l|}{\cellcolor[HTML]{FCE6AB}F40} & \multicolumn{1}{r|}{\cellcolor[HTML]{D3FFB6}\textbf{379570}}  & 371906                                                    \\ \hline
\multicolumn{1}{|l|}{\cellcolor[HTML]{FCE6AB}F41} & \multicolumn{1}{r|}{958934}                                   & \cellcolor[HTML]{D3FFB6}\textbf{960695}                   \\ \hline
\multicolumn{1}{|l|}{\cellcolor[HTML]{FCE6AB}F42} & \multicolumn{1}{r|}{\cellcolor[HTML]{D3FFB6}\textbf{315558}}  & 310368                                                    \\ \hline
\multicolumn{1}{|l|}{\cellcolor[HTML]{FCE6AB}F43} & \multicolumn{1}{r|}{\cellcolor[HTML]{D3FFB6}\textbf{32375.5}} & 31956                                                     \\ \hline
\multicolumn{1}{|l|}{\cellcolor[HTML]{FCE6AB}F44} & \multicolumn{1}{r|}{106229}                                   & \cellcolor[HTML]{D3FFB6}\textbf{107874}                   \\ \hline
\multicolumn{1}{|l|}{\cellcolor[HTML]{FCE6AB}F45} & \multicolumn{1}{r|}{\cellcolor[HTML]{D3FFB6}\textbf{813622}}  & 804278                                                    \\ \hline
\multicolumn{1}{|l|}{\cellcolor[HTML]{FCE6AB}F46} & \multicolumn{1}{r|}{\cellcolor[HTML]{D3FFB6}\textbf{44144.1}} & 43930.7                                                   \\ \hline
\multicolumn{1}{|l|}{\cellcolor[HTML]{FCE6AB}F47} & \multicolumn{1}{r|}{\cellcolor[HTML]{D3FFB6}\textbf{726019}}  & 711466                                                    \\ \hline
\multicolumn{1}{|l|}{\cellcolor[HTML]{FCE6AB}F48} & \multicolumn{1}{r|}{\cellcolor[HTML]{D3FFB6}\textbf{813949}}  & 805644                                                    \\ \hline
\multicolumn{1}{|l|}{\cellcolor[HTML]{FCE6AB}F49} & \multicolumn{1}{r|}{\cellcolor[HTML]{D3FFB6}\textbf{653176}}  & 639675                                                    \\ \hline
\multicolumn{1}{|l|}{\cellcolor[HTML]{FCE6AB}F50} & \multicolumn{1}{r|}{\cellcolor[HTML]{D3FFB6}\textbf{50543}}   & 49255.1                                                   \\ \hline
\multicolumn{1}{|l|}{\cellcolor[HTML]{FCE6AB}F51} & \multicolumn{1}{r|}{\cellcolor[HTML]{D3FFB6}\textbf{211429}}  & 206394                                                    \\ \hline
\multicolumn{1}{|l|}{\cellcolor[HTML]{FCE6AB}F52} & \multicolumn{1}{r|}{\cellcolor[HTML]{D3FFB6}\textbf{244518}}  & 243532                                                    \\ \hline
\multicolumn{1}{|l|}{\cellcolor[HTML]{FCE6AB}F53} & \multicolumn{1}{r|}{\cellcolor[HTML]{D3FFB6}\textbf{227361}}  & 226204                                                    \\ \hline
\multicolumn{1}{|l|}{\cellcolor[HTML]{FCE6AB}F54} & \multicolumn{1}{r|}{\cellcolor[HTML]{D3FFB6}\textbf{193634}}  & 189500                                                    \\ \hline
\multicolumn{1}{|l|}{\cellcolor[HTML]{FCE6AB}F55} & \multicolumn{1}{r|}{\cellcolor[HTML]{D3FFB6}\textbf{81477.9}} & 78811.9                                                   \\ \hline
\multicolumn{1}{|l|}{\cellcolor[HTML]{FCE6AB}F56} & \multicolumn{1}{r|}{\cellcolor[HTML]{D3FFB6}\textbf{61504}}   & 59037                                                     \\ \hline
\multicolumn{1}{|l|}{\cellcolor[HTML]{FCE6AB}F57} & \multicolumn{1}{r|}{\cellcolor[HTML]{D3FFB6}\textbf{150655}}  & 146471                                                    \\ \hline
\multicolumn{1}{|l|}{\cellcolor[HTML]{FCE6AB}F58} & \multicolumn{1}{r|}{\cellcolor[HTML]{D3FFB6}\textbf{51459}}   & 49706.3                                                   \\ \hline
\multicolumn{1}{|l|}{\cellcolor[HTML]{FCE6AB}F59} & \multicolumn{1}{r|}{\cellcolor[HTML]{D3FFB6}\textbf{287285}}  & 283170                                                    \\ \hline
\multicolumn{1}{|l|}{\cellcolor[HTML]{FCE6AB}F60} & \multicolumn{1}{r|}{\cellcolor[HTML]{D3FFB6}\textbf{383329}}  & 374933                                                    \\ \hline
\multicolumn{1}{|l|}{\cellcolor[HTML]{FCE6AB}F61} & \multicolumn{1}{r|}{\cellcolor[HTML]{D3FFB6}\textbf{214848}}  & 205342                                                    \\ \hline
\multicolumn{1}{|l|}{\cellcolor[HTML]{FCE6AB}F62} & \multicolumn{1}{r|}{\cellcolor[HTML]{D3FFB6}\textbf{225192}}  & 219207                                                    \\ \hline
\multicolumn{1}{|l|}{\cellcolor[HTML]{FCE6AB}F63} & \multicolumn{1}{r|}{\cellcolor[HTML]{D3FFB6}\textbf{225684}}  & 220104                                                    \\ \hline
\multicolumn{1}{|l|}{\cellcolor[HTML]{FCE6AB}F64} & \multicolumn{1}{r|}{489291}                                   & \cellcolor[HTML]{D3FFB6}\textbf{490903}                   \\ \hline
\multicolumn{1}{|l|}{\cellcolor[HTML]{FCE6AB}F65} & \multicolumn{1}{r|}{\cellcolor[HTML]{D3FFB6}\textbf{440828}}  & 439020                                                    \\ \hline
\multicolumn{1}{|l|}{\cellcolor[HTML]{FCE6AB}F66} & \multicolumn{1}{r|}{\cellcolor[HTML]{D3FFB6}\textbf{220101}}  & 214588                                                    \\ \hline
\multicolumn{1}{|l|}{\cellcolor[HTML]{FCE6AB}F67} & \multicolumn{1}{r|}{\cellcolor[HTML]{D3FFB6}\textbf{324902}}  & 318589                                                    \\ \hline
\multicolumn{1}{|l|}{\cellcolor[HTML]{FCE6AB}F68} & \multicolumn{1}{r|}{\cellcolor[HTML]{D3FFB6}\textbf{110572}}  & 108423                                                    \\ \hline
\multicolumn{1}{|l|}{\cellcolor[HTML]{FCE6AB}F69} & \multicolumn{1}{r|}{\cellcolor[HTML]{D3FFB6}\textbf{150784}}  & 148614                                                    \\ \hline
\multicolumn{1}{|l|}{\cellcolor[HTML]{FCE6AB}F70} & \multicolumn{1}{r|}{\cellcolor[HTML]{D3FFB6}\textbf{452607}}  & 445441                                                    \\ \hline
\multicolumn{1}{|l|}{\cellcolor[HTML]{FCE6AB}F71} & \multicolumn{1}{r|}{\cellcolor[HTML]{D3FFB6}\textbf{281795}}  & 277919                                                    \\ \hline
\multicolumn{1}{|l|}{\cellcolor[HTML]{FCE6AB}F72} & \multicolumn{1}{r|}{\cellcolor[HTML]{D3FFB6}\textbf{66610.6}} & 65806.9                                                   \\ \hline
\multicolumn{1}{|l|}{\cellcolor[HTML]{FCE6AB}F73} & \multicolumn{1}{r|}{\cellcolor[HTML]{D3FFB6}\textbf{133906}}  & 133560                                                    \\ \hline
\multicolumn{1}{|l|}{\cellcolor[HTML]{FCE6AB}F74} & \multicolumn{1}{r|}{\cellcolor[HTML]{D3FFB6}\textbf{146870}}  & 144762                                                    \\ \hline
\multicolumn{1}{|l|}{\cellcolor[HTML]{FCE6AB}F75} & \multicolumn{1}{r|}{\cellcolor[HTML]{D3FFB6}\textbf{234757}}  & 232794                                                    \\ \hline
\multicolumn{1}{|l|}{\cellcolor[HTML]{FCE6AB}F76} & \multicolumn{1}{r|}{\cellcolor[HTML]{D3FFB6}\textbf{275504}}  & 269788                                                    \\ \hline
\multicolumn{1}{|l|}{\cellcolor[HTML]{FCE6AB}F77} & \multicolumn{1}{r|}{\cellcolor[HTML]{D3FFB6}\textbf{620630}}  & 616128                                                    \\ \hline
\multicolumn{1}{|l|}{\cellcolor[HTML]{FCE6AB}F78} & \multicolumn{1}{r|}{\cellcolor[HTML]{D3FFB6}\textbf{534844}}  & 525471                                                    \\ \hline
\multicolumn{1}{|l|}{\cellcolor[HTML]{FCE6AB}F79} & \multicolumn{1}{r|}{\cellcolor[HTML]{D3FFB6}\textbf{379471}}  & 370202                                                    \\ \hline
\multicolumn{1}{|l|}{\cellcolor[HTML]{FCE6AB}F80} & \multicolumn{1}{r|}{\cellcolor[HTML]{D3FFB6}\textbf{30802.7}} & 30668.8                                                   \\ \hline
\multicolumn{1}{|l|}{\cellcolor[HTML]{FCE6AB}F81} & \multicolumn{1}{r|}{\cellcolor[HTML]{D3FFB6}\textbf{269072}}  & 261697                                                    \\ \hline
\multicolumn{1}{|l|}{\cellcolor[HTML]{FCE6AB}F82} & \multicolumn{1}{r|}{\cellcolor[HTML]{D3FFB6}\textbf{453902}}  & 441509                                                    \\ \hline
\multicolumn{1}{|l|}{\cellcolor[HTML]{FCE6AB}F83} & \multicolumn{1}{r|}{\cellcolor[HTML]{D3FFB6}\textbf{15967.4}} & 15628.8                                                   \\ \hline
\multicolumn{1}{|l|}{\cellcolor[HTML]{FCE6AB}F84} & \multicolumn{1}{r|}{\cellcolor[HTML]{D3FFB6}\textbf{253405}}  & 245342                                                    \\ \hline
\multicolumn{1}{|l|}{\cellcolor[HTML]{FCE6AB}F85} & \multicolumn{1}{r|}{\cellcolor[HTML]{D3FFB6}\textbf{494802}}  & 487566                                                    \\ \hline
\multicolumn{1}{|l|}{\cellcolor[HTML]{FCE6AB}F86} & \multicolumn{1}{r|}{\cellcolor[HTML]{D3FFB6}\textbf{9783.9}}  & 9749.8                                                    \\ \hline
\multicolumn{1}{|l|}{\cellcolor[HTML]{FCE6AB}F87} & \multicolumn{1}{r|}{\cellcolor[HTML]{D3FFB6}\textbf{237551}}  & 226145                                                    \\ \hline
\multicolumn{1}{|l|}{\cellcolor[HTML]{FCE6AB}F88} & \multicolumn{1}{r|}{\cellcolor[HTML]{D3FFB6}\textbf{1017990}} & 1009100                                                   \\ \hline
\multicolumn{1}{|l|}{\cellcolor[HTML]{FCE6AB}F89} & \multicolumn{1}{r|}{\cellcolor[HTML]{D3FFB6}\textbf{483693}}  & 469319                                                    \\ \hline
\multicolumn{1}{|l|}{\cellcolor[HTML]{FCE6AB}F90} & \multicolumn{1}{r|}{\cellcolor[HTML]{D3FFB6}\textbf{110079}}  & 105789                                                    \\ \hline
\multicolumn{1}{|l|}{\cellcolor[HTML]{FCE6AB}F91} & \multicolumn{1}{r|}{\cellcolor[HTML]{D3FFB6}\textbf{890216}}  & 873907                                                    \\ \hline
\multicolumn{1}{|l|}{\cellcolor[HTML]{FCE6AB}F92} & \multicolumn{1}{r|}{\cellcolor[HTML]{D3FFB6}\textbf{310174}}  & 295377                                                    \\ \hline
\multicolumn{1}{|l|}{\cellcolor[HTML]{FCE6AB}F93} & \multicolumn{1}{r|}{\cellcolor[HTML]{D3FFB6}\textbf{722245}}  & 706744                                                    \\ \hline
\multicolumn{1}{|l|}{\cellcolor[HTML]{FCE6AB}F94} & \multicolumn{1}{r|}{\cellcolor[HTML]{D3FFB6}\textbf{738970}}  & 719059                                                    \\ \hline
\multicolumn{1}{|l|}{\cellcolor[HTML]{FCE6AB}F95} & \multicolumn{1}{r|}{\cellcolor[HTML]{D3FFB6}\textbf{46951.9}} & 46737.7                                                   \\ \hline
\multicolumn{1}{|l|}{\cellcolor[HTML]{FCE6AB}F96} & \multicolumn{1}{r|}{\cellcolor[HTML]{D3FFB6}\textbf{770539}}  & 751434                                                    \\ \hline
\multicolumn{1}{|l|}{\cellcolor[HTML]{FCE6AB}F97} & \multicolumn{1}{r|}{\cellcolor[HTML]{D3FFB6}\textbf{766150}}  & 752530                                                    \\ \hline
\multicolumn{1}{|l}{\cellcolor[HTML]{FFFFC7}Best} & \multicolumn{1}{c}{\cellcolor[HTML]{D3FFB6}\textbf{87}}       & \multicolumn{1}{c|}{10}                                   \\
\multicolumn{1}{|l}{Ranking}                      & \multicolumn{1}{c}{\cellcolor[HTML]{D3FFB6}\textbf{1.103093}} & \multicolumn{1}{c|}{1.896907}                             \\ \hline
\end{tabular}
\caption{Valores+Ranking Final}
\end{table}

% Please add the following required packages to your document preamble:
% \usepackage{multirow}
% \usepackage[table,xcdraw]{xcolor}
% If you use beamer only pass "xcolor=table" option, i.e. \documentclass[xcolor=table]{beamer}
\begin{table}[]
\begin{tabular}{ll|r|r|}
\cline{3-4}
                                                                         &                                 & \multicolumn{1}{l|}{\cellcolor[HTML]{FFFFC7}GACEPwGRASP}       & \multicolumn{1}{l|}{\cellcolor[HTML]{FFFFC7}GACEPwoGRASP} \\ \hline
\multicolumn{1}{|l|}{\cellcolor[HTML]{ECF4FF}}                           & \cellcolor[HTML]{FCE6AB}F01     & 203870                                                         & \cellcolor[HTML]{D3FFB6}\textbf{203917}                   \\ \cline{2-4} 
\multicolumn{1}{|l|}{\cellcolor[HTML]{ECF4FF}}                           & \cellcolor[HTML]{FCE6AB}F02     & \cellcolor[HTML]{D3FFB6}\textbf{88229.5}                       & 86893.8                                                   \\ \cline{2-4} 
\multicolumn{1}{|l|}{\cellcolor[HTML]{ECF4FF}}                           & \cellcolor[HTML]{FCE6AB}F03     & \cellcolor[HTML]{D3FFB6}\textbf{202940}                        & 202056                                                    \\ \cline{2-4} 
\multicolumn{1}{|l|}{\cellcolor[HTML]{ECF4FF}}                           & \cellcolor[HTML]{FCE6AB}F04     & \cellcolor[HTML]{D3FFB6}\textbf{235610}                        & 234528                                                    \\ \cline{2-4} 
\multicolumn{1}{|l|}{\cellcolor[HTML]{ECF4FF}}                           & \cellcolor[HTML]{FCE6AB}F05     & 238439                                                         & \cellcolor[HTML]{D3FFB6}\textbf{238931}                   \\ \cline{2-4} 
\multicolumn{1}{|l|}{\cellcolor[HTML]{ECF4FF}}                           & \cellcolor[HTML]{FCE6AB}F06     & \cellcolor[HTML]{D3FFB6}\textbf{81325.7}                       & 78698.4                                                   \\ \cline{2-4} 
\multicolumn{1}{|l|}{\cellcolor[HTML]{ECF4FF}}                           & \cellcolor[HTML]{FCE6AB}F07     & \cellcolor[HTML]{D3FFB6}\textbf{10867.2}                       & 10611.2                                                   \\ \cline{2-4} 
\multicolumn{1}{|l|}{\cellcolor[HTML]{ECF4FF}}                           & \cellcolor[HTML]{FCE6AB}F08     & \cellcolor[HTML]{D3FFB6}\textbf{67669.3}                       & 67007.1                                                   \\ \cline{2-4} 
\multicolumn{1}{|l|}{\cellcolor[HTML]{ECF4FF}}                           & \cellcolor[HTML]{FCE6AB}F09     & 239982                                                         & \cellcolor[HTML]{D3FFB6}\textbf{241977}                   \\ \cline{2-4} 
\multicolumn{1}{|l|}{\cellcolor[HTML]{ECF4FF}}                           & \cellcolor[HTML]{FCE6AB}F10     & \cellcolor[HTML]{D3FFB6}\textbf{26642.1}                       & 26107.5                                                   \\ \cline{2-4} 
\multicolumn{1}{|l|}{\cellcolor[HTML]{ECF4FF}}                           & \cellcolor[HTML]{FCE6AB}F11     & \cellcolor[HTML]{D3FFB6}\textbf{20217.6}                       & 20195.7                                                   \\ \cline{2-4} 
\multicolumn{1}{|l|}{\cellcolor[HTML]{ECF4FF}}                           & \cellcolor[HTML]{FCE6AB}F12     & \cellcolor[HTML]{D3FFB6}\textbf{58173.8}                       & 58108.8                                                   \\ \cline{2-4} 
\multicolumn{1}{|l|}{\cellcolor[HTML]{ECF4FF}}                           & \cellcolor[HTML]{FCE6AB}F13     & \cellcolor[HTML]{D3FFB6}\textbf{3967.14}                       & 3841.06                                                   \\ \cline{2-4} 
\multicolumn{1}{|l|}{\cellcolor[HTML]{ECF4FF}}                           & \cellcolor[HTML]{FCE6AB}F14     & \cellcolor[HTML]{D3FFB6}\textbf{53183.3}                       & 52534.5                                                   \\ \cline{2-4} 
\multicolumn{1}{|l|}{\cellcolor[HTML]{ECF4FF}}                           & \cellcolor[HTML]{FCE6AB}F15     & 62310.9                                                        & \cellcolor[HTML]{D3FFB6}\textbf{62661.5}                  \\ \cline{2-4} 
\multicolumn{1}{|l|}{\cellcolor[HTML]{ECF4FF}}                           & \cellcolor[HTML]{FCE6AB}F16     & \cellcolor[HTML]{D3FFB6}\textbf{38963}                         & 38120.5                                                   \\ \cline{2-4} 
\multicolumn{1}{|l|}{\cellcolor[HTML]{ECF4FF}}                           & \cellcolor[HTML]{FCE6AB}F17     & \cellcolor[HTML]{D3FFB6}\textbf{15641.2}                       & 15454.7                                                   \\ \cline{2-4} 
\multicolumn{1}{|l|}{\cellcolor[HTML]{ECF4FF}}                           & \cellcolor[HTML]{FCE6AB}F18     & \cellcolor[HTML]{D3FFB6}\textbf{22181.6}                       & 21423.6                                                   \\ \cline{2-4} 
\multicolumn{1}{|l|}{\cellcolor[HTML]{ECF4FF}}                           & \cellcolor[HTML]{FCE6AB}F19     & \cellcolor[HTML]{D3FFB6}\textbf{37575.9}                       & 36782.8                                                   \\ \cline{2-4} 
\multicolumn{1}{|l|}{\cellcolor[HTML]{ECF4FF}}                           & \cellcolor[HTML]{FCE6AB}F20     & 94259.5                                                        & \cellcolor[HTML]{D3FFB6}\textbf{94302.4}                  \\ \cline{2-4} 
\multicolumn{1}{|l|}{\cellcolor[HTML]{ECF4FF}}                           & \cellcolor[HTML]{FCE6AB}F21     & \cellcolor[HTML]{D3FFB6}\textbf{89232.8}                       & 88189.9                                                   \\ \cline{2-4} 
\multicolumn{1}{|l|}{\cellcolor[HTML]{ECF4FF}}                           & \cellcolor[HTML]{FCE6AB}F22     & \cellcolor[HTML]{D3FFB6}\textbf{110642}                        & 108332                                                    \\ \cline{2-4} 
\multicolumn{1}{|l|}{\cellcolor[HTML]{ECF4FF}}                           & \cellcolor[HTML]{FCE6AB}F23     & \cellcolor[HTML]{D3FFB6}\textbf{37247.4}                       & 36961.9                                                   \\ \cline{2-4} 
\multicolumn{1}{|l|}{\cellcolor[HTML]{ECF4FF}}                           & \cellcolor[HTML]{FCE6AB}F24     & 110026                                                         & \cellcolor[HTML]{D3FFB6}\textbf{110489}                   \\ \cline{2-4} 
\multicolumn{1}{|l|}{\cellcolor[HTML]{ECF4FF}}                           & \cellcolor[HTML]{FCE6AB}F25     & \cellcolor[HTML]{D3FFB6}\textbf{60821.5}                       & 57749                                                     \\ \cline{2-4} 
\multicolumn{1}{|l|}{\cellcolor[HTML]{ECF4FF}}                           & \cellcolor[HTML]{FCE6AB}F26     & \cellcolor[HTML]{D3FFB6}\textbf{18066.4}                       & 17239.7                                                   \\ \cline{2-4} 
\multicolumn{1}{|l|}{\cellcolor[HTML]{ECF4FF}}                           & \cellcolor[HTML]{FCE6AB}F27     & \cellcolor[HTML]{D3FFB6}\textbf{55937.8}                       & 55209.6                                                   \\ \cline{2-4} 
\multicolumn{1}{|l|}{\cellcolor[HTML]{ECF4FF}}                           & \cellcolor[HTML]{FCE6AB}F28     & \cellcolor[HTML]{D3FFB6}\textbf{58848.7}                       & 57800.7                                                   \\ \cline{2-4} 
\multicolumn{1}{|l|}{\cellcolor[HTML]{ECF4FF}}                           & \cellcolor[HTML]{FCE6AB}F29     & \cellcolor[HTML]{D3FFB6}\textbf{73569.8}                       & 72183.1                                                   \\ \cline{2-4} 
\multicolumn{1}{|l|}{\cellcolor[HTML]{ECF4FF}}                           & \cellcolor[HTML]{FCE6AB}F30     & \cellcolor[HTML]{D3FFB6}\textbf{151250}                        & 151073                                                    \\ \cline{2-4} 
\multicolumn{1}{|l|}{\cellcolor[HTML]{ECF4FF}}                           & \cellcolor[HTML]{FCE6AB}F31     & 191323                                                         & \cellcolor[HTML]{D3FFB6}\textbf{191744}                   \\ \cline{2-4} 
\multicolumn{1}{|l|}{\cellcolor[HTML]{ECF4FF}}                           & \cellcolor[HTML]{FCE6AB}F32     & \cellcolor[HTML]{D3FFB6}\textbf{104160}                        & 101933                                                    \\ \cline{2-4} 
\multicolumn{1}{|l|}{\cellcolor[HTML]{ECF4FF}}                           & \cellcolor[HTML]{FCE6AB}F33     & \cellcolor[HTML]{D3FFB6}\textbf{66522.7}                       & 65749.4                                                   \\ \cline{2-4} 
\multicolumn{1}{|l|}{\cellcolor[HTML]{ECF4FF}}                           & \cellcolor[HTML]{FCE6AB}F34     & \cellcolor[HTML]{D3FFB6}\textbf{80481.7}                       & 78826.7                                                   \\ \cline{2-4} 
\multicolumn{1}{|l|}{\cellcolor[HTML]{ECF4FF}}                           & \cellcolor[HTML]{FCE6AB}F35     & \cellcolor[HTML]{D3FFB6}\textbf{31122.7}                       & 30323.9                                                   \\ \cline{2-4} 
\multicolumn{1}{|l|}{\cellcolor[HTML]{ECF4FF}}                           & \cellcolor[HTML]{FCE6AB}F36     & \cellcolor[HTML]{D3FFB6}\textbf{153157}                        & 150580                                                    \\ \cline{2-4} 
\multicolumn{1}{|l|}{\cellcolor[HTML]{ECF4FF}}                           & \cellcolor[HTML]{FCE6AB}F37     & \cellcolor[HTML]{D3FFB6}\textbf{120369}                        & 118139                                                    \\ \cline{2-4} 
\multicolumn{1}{|l|}{\cellcolor[HTML]{ECF4FF}}                           & \cellcolor[HTML]{FCE6AB}F38     & \cellcolor[HTML]{D3FFB6}\textbf{21816.6}                       & 21744.5                                                   \\ \cline{2-4} 
\multicolumn{1}{|l|}{\multirow{-39}{*}{\cellcolor[HTML]{ECF4FF}n = 100}} & \cellcolor[HTML]{FCE6AB}F39     & \cellcolor[HTML]{D3FFB6}\textbf{112427}                        & 111748                                                    \\ \hline
\multicolumn{1}{|l|}{\cellcolor[HTML]{ECF4FF}}                           & \cellcolor[HTML]{FCE6AB}F40     & \cellcolor[HTML]{D3FFB6}\textbf{379570}                        & 371906                                                    \\ \cline{2-4} 
\multicolumn{1}{|l|}{\cellcolor[HTML]{ECF4FF}}                           & \cellcolor[HTML]{FCE6AB}F41     & 958934                                                         & \cellcolor[HTML]{D3FFB6}\textbf{960695}                   \\ \cline{2-4} 
\multicolumn{1}{|l|}{\cellcolor[HTML]{ECF4FF}}                           & \cellcolor[HTML]{FCE6AB}F42     & \cellcolor[HTML]{D3FFB6}\textbf{315558}                        & 310368                                                    \\ \cline{2-4} 
\multicolumn{1}{|l|}{\cellcolor[HTML]{ECF4FF}}                           & \cellcolor[HTML]{FCE6AB}F43     & \cellcolor[HTML]{D3FFB6}\textbf{32375.5}                       & 31956                                                     \\ \cline{2-4} 
\multicolumn{1}{|l|}{\cellcolor[HTML]{ECF4FF}}                           & \cellcolor[HTML]{FCE6AB}F44     & 106229                                                         & \cellcolor[HTML]{D3FFB6}\textbf{107874}                   \\ \cline{2-4} 
\multicolumn{1}{|l|}{\cellcolor[HTML]{ECF4FF}}                           & \cellcolor[HTML]{FCE6AB}F45     & \cellcolor[HTML]{D3FFB6}\textbf{813622}                        & 804278                                                    \\ \cline{2-4} 
\multicolumn{1}{|l|}{\cellcolor[HTML]{ECF4FF}}                           & \cellcolor[HTML]{FCE6AB}F46     & \cellcolor[HTML]{D3FFB6}\textbf{44144.1}                       & 43930.7                                                   \\ \cline{2-4} 
\multicolumn{1}{|l|}{\cellcolor[HTML]{ECF4FF}}                           & \cellcolor[HTML]{FCE6AB}F47     & \cellcolor[HTML]{D3FFB6}\textbf{726019}                        & 711466                                                    \\ \cline{2-4} 
\multicolumn{1}{|l|}{\cellcolor[HTML]{ECF4FF}}                           & \cellcolor[HTML]{FCE6AB}F48     & \cellcolor[HTML]{D3FFB6}\textbf{813949}                        & 805644                                                    \\ \cline{2-4} 
\multicolumn{1}{|l|}{\cellcolor[HTML]{ECF4FF}}                           & \cellcolor[HTML]{FCE6AB}F49     & \cellcolor[HTML]{D3FFB6}\textbf{653176}                        & 639675                                                    \\ \cline{2-4} 
\multicolumn{1}{|l|}{\cellcolor[HTML]{ECF4FF}}                           & \cellcolor[HTML]{FCE6AB}F50     & \cellcolor[HTML]{D3FFB6}\textbf{50543}                         & 49255.1                                                   \\ \cline{2-4} 
\multicolumn{1}{|l|}{\cellcolor[HTML]{ECF4FF}}                           & \cellcolor[HTML]{FCE6AB}F51     & \cellcolor[HTML]{D3FFB6}\textbf{211429}                        & 206394                                                    \\ \cline{2-4} 
\multicolumn{1}{|l|}{\cellcolor[HTML]{ECF4FF}}                           & \cellcolor[HTML]{FCE6AB}F52     & \cellcolor[HTML]{D3FFB6}\textbf{244518}                        & 243532                                                    \\ \cline{2-4} 
\multicolumn{1}{|l|}{\cellcolor[HTML]{ECF4FF}}                           & \cellcolor[HTML]{FCE6AB}F53     & \cellcolor[HTML]{D3FFB6}\textbf{227361}                        & 226204                                                    \\ \cline{2-4} 
\multicolumn{1}{|l|}{\cellcolor[HTML]{ECF4FF}}                           & \cellcolor[HTML]{FCE6AB}F54     & \cellcolor[HTML]{D3FFB6}\textbf{193634}                        & 189500                                                    \\ \cline{2-4} 
\multicolumn{1}{|l|}{\cellcolor[HTML]{ECF4FF}}                           & \cellcolor[HTML]{FCE6AB}F55     & \cellcolor[HTML]{D3FFB6}\textbf{81477.9}                       & 78811.9                                                   \\ \cline{2-4} 
\multicolumn{1}{|l|}{\cellcolor[HTML]{ECF4FF}}                           & \cellcolor[HTML]{FCE6AB}F56     & \cellcolor[HTML]{D3FFB6}\textbf{61504}                         & 59037                                                     \\ \cline{2-4} 
\multicolumn{1}{|l|}{\cellcolor[HTML]{ECF4FF}}                           & \cellcolor[HTML]{FCE6AB}F57     & \cellcolor[HTML]{D3FFB6}\textbf{150655}                        & 146471                                                    \\ \cline{2-4} 
\multicolumn{1}{|l|}{\cellcolor[HTML]{ECF4FF}}                           & \cellcolor[HTML]{FCE6AB}F58     & \cellcolor[HTML]{D3FFB6}\textbf{51459}                         & 49706.3                                                   \\ \cline{2-4} 
\multicolumn{1}{|l|}{\cellcolor[HTML]{ECF4FF}}                           & \cellcolor[HTML]{FCE6AB}F59     & \cellcolor[HTML]{D3FFB6}\textbf{287285}                        & 283170                                                    \\ \cline{2-4} 
\multicolumn{1}{|l|}{\cellcolor[HTML]{ECF4FF}}                           & \cellcolor[HTML]{FCE6AB}F60     & \cellcolor[HTML]{D3FFB6}\textbf{383329}                        & 374933                                                    \\ \cline{2-4} 
\multicolumn{1}{|l|}{\cellcolor[HTML]{ECF4FF}}                           & \cellcolor[HTML]{FCE6AB}F61     & \cellcolor[HTML]{D3FFB6}\textbf{214848}                        & 205342                                                    \\ \cline{2-4} 
\multicolumn{1}{|l|}{\cellcolor[HTML]{ECF4FF}}                           & \cellcolor[HTML]{FCE6AB}F62     & \cellcolor[HTML]{D3FFB6}\textbf{225192}                        & 219207                                                    \\ \cline{2-4} 
\multicolumn{1}{|l|}{\cellcolor[HTML]{ECF4FF}}                           & \cellcolor[HTML]{FCE6AB}F63     & \cellcolor[HTML]{D3FFB6}\textbf{225684}                        & 220104                                                    \\ \cline{2-4} 
\multicolumn{1}{|l|}{\cellcolor[HTML]{ECF4FF}}                           & \cellcolor[HTML]{FCE6AB}F64     & 489291                                                         & \cellcolor[HTML]{D3FFB6}\textbf{490903}                   \\ \cline{2-4} 
\multicolumn{1}{|l|}{\cellcolor[HTML]{ECF4FF}}                           & \cellcolor[HTML]{FCE6AB}F65     & \cellcolor[HTML]{D3FFB6}\textbf{440828}                        & 439020                                                    \\ \cline{2-4} 
\multicolumn{1}{|l|}{\cellcolor[HTML]{ECF4FF}}                           & \cellcolor[HTML]{FCE6AB}F66     & \cellcolor[HTML]{D3FFB6}\textbf{220101}                        & 214588                                                    \\ \cline{2-4} 
\multicolumn{1}{|l|}{\cellcolor[HTML]{ECF4FF}}                           & \cellcolor[HTML]{FCE6AB}F67     & \cellcolor[HTML]{D3FFB6}\textbf{324902}                        & 318589                                                    \\ \cline{2-4} 
\multicolumn{1}{|l|}{\cellcolor[HTML]{ECF4FF}}                           & \cellcolor[HTML]{FCE6AB}F68     & \cellcolor[HTML]{D3FFB6}\textbf{110572}                        & 108423                                                    \\ \cline{2-4} 
\multicolumn{1}{|l|}{\cellcolor[HTML]{ECF4FF}}                           & \cellcolor[HTML]{FCE6AB}F69     & \cellcolor[HTML]{D3FFB6}\textbf{150784}                        & 148614                                                    \\ \cline{2-4} 
\multicolumn{1}{|l|}{\cellcolor[HTML]{ECF4FF}}                           & \cellcolor[HTML]{FCE6AB}F70     & \cellcolor[HTML]{D3FFB6}\textbf{452607}                        & 445441                                                    \\ \cline{2-4} 
\multicolumn{1}{|l|}{\cellcolor[HTML]{ECF4FF}}                           & \cellcolor[HTML]{FCE6AB}F71     & \cellcolor[HTML]{D3FFB6}\textbf{281795}                        & 277919                                                    \\ \cline{2-4} 
\multicolumn{1}{|l|}{\cellcolor[HTML]{ECF4FF}}                           & \cellcolor[HTML]{FCE6AB}F72     & \cellcolor[HTML]{D3FFB6}\textbf{66610.6}                       & 65806.9                                                   \\ \cline{2-4} 
\multicolumn{1}{|l|}{\cellcolor[HTML]{ECF4FF}}                           & \cellcolor[HTML]{FCE6AB}F73     & \cellcolor[HTML]{D3FFB6}\textbf{133906}                        & 133560                                                    \\ \cline{2-4} 
\multicolumn{1}{|l|}{\cellcolor[HTML]{ECF4FF}}                           & \cellcolor[HTML]{FCE6AB}F74     & \cellcolor[HTML]{D3FFB6}\textbf{146870}                        & 144762                                                    \\ \cline{2-4} 
\multicolumn{1}{|l|}{\cellcolor[HTML]{ECF4FF}}                           & \cellcolor[HTML]{FCE6AB}F75     & \cellcolor[HTML]{D3FFB6}\textbf{234757}                        & 232794                                                    \\ \cline{2-4} 
\multicolumn{1}{|l|}{\cellcolor[HTML]{ECF4FF}}                           & \cellcolor[HTML]{FCE6AB}F76     & \cellcolor[HTML]{D3FFB6}\textbf{275504}                        & 269788                                                    \\ \cline{2-4} 
\multicolumn{1}{|l|}{\cellcolor[HTML]{ECF4FF}}                           & \cellcolor[HTML]{FCE6AB}F77     & \cellcolor[HTML]{D3FFB6}\textbf{620630}                        & 616128                                                    \\ \cline{2-4} 
\multicolumn{1}{|l|}{\multirow{-39}{*}{\cellcolor[HTML]{ECF4FF}n = 200}} & \cellcolor[HTML]{FCE6AB}F78     & \cellcolor[HTML]{D3FFB6}\textbf{534844}                        & 525471                                                    \\ \hline
\multicolumn{1}{|l|}{\cellcolor[HTML]{ECF4FF}}                           & \cellcolor[HTML]{FCE6AB}F79     & \cellcolor[HTML]{D3FFB6}\textbf{379471}                        & 370202                                                    \\ \cline{2-4} 
\multicolumn{1}{|l|}{\cellcolor[HTML]{ECF4FF}}                           & \cellcolor[HTML]{FCE6AB}F80     & \cellcolor[HTML]{D3FFB6}\textbf{30802.7}                       & 30668.8                                                   \\ \cline{2-4} 
\multicolumn{1}{|l|}{\cellcolor[HTML]{ECF4FF}}                           & \cellcolor[HTML]{FCE6AB}F81     & \cellcolor[HTML]{D3FFB6}\textbf{269072}                        & 261697                                                    \\ \cline{2-4} 
\multicolumn{1}{|l|}{\cellcolor[HTML]{ECF4FF}}                           & \cellcolor[HTML]{FCE6AB}F82     & \cellcolor[HTML]{D3FFB6}\textbf{453902}                        & 441509                                                    \\ \cline{2-4} 
\multicolumn{1}{|l|}{\cellcolor[HTML]{ECF4FF}}                           & \cellcolor[HTML]{FCE6AB}F83     & \cellcolor[HTML]{D3FFB6}\textbf{15967.4}                       & 15628.8                                                   \\ \cline{2-4} 
\multicolumn{1}{|l|}{\cellcolor[HTML]{ECF4FF}}                           & \cellcolor[HTML]{FCE6AB}F84     & \cellcolor[HTML]{D3FFB6}\textbf{253405}                        & 245342                                                    \\ \cline{2-4} 
\multicolumn{1}{|l|}{\cellcolor[HTML]{ECF4FF}}                           & \cellcolor[HTML]{FCE6AB}F85     & \cellcolor[HTML]{D3FFB6}\textbf{494802}                        & 487566                                                    \\ \cline{2-4} 
\multicolumn{1}{|l|}{\cellcolor[HTML]{ECF4FF}}                           & \cellcolor[HTML]{FCE6AB}F86     & \cellcolor[HTML]{D3FFB6}\textbf{9783.9}                        & 9749.8                                                    \\ \cline{2-4} 
\multicolumn{1}{|l|}{\cellcolor[HTML]{ECF4FF}}                           & \cellcolor[HTML]{FCE6AB}F87     & \cellcolor[HTML]{D3FFB6}\textbf{237551}                        & 226145                                                    \\ \cline{2-4} 
\multicolumn{1}{|l|}{\cellcolor[HTML]{ECF4FF}}                           & \cellcolor[HTML]{FCE6AB}F88     & \cellcolor[HTML]{D3FFB6}\textbf{1017990}                       & 1009100                                                   \\ \cline{2-4} 
\multicolumn{1}{|l|}{\cellcolor[HTML]{ECF4FF}}                           & \cellcolor[HTML]{FCE6AB}F89     & \cellcolor[HTML]{D3FFB6}\textbf{483693}                        & 469319                                                    \\ \cline{2-4} 
\multicolumn{1}{|l|}{\cellcolor[HTML]{ECF4FF}}                           & \cellcolor[HTML]{FCE6AB}F90     & \cellcolor[HTML]{D3FFB6}\textbf{110079}                        & 105789                                                    \\ \cline{2-4} 
\multicolumn{1}{|l|}{\cellcolor[HTML]{ECF4FF}}                           & \cellcolor[HTML]{FCE6AB}F91     & \cellcolor[HTML]{D3FFB6}\textbf{890216}                        & 873907                                                    \\ \cline{2-4} 
\multicolumn{1}{|l|}{\cellcolor[HTML]{ECF4FF}}                           & \cellcolor[HTML]{FCE6AB}F92     & \cellcolor[HTML]{D3FFB6}\textbf{310174}                        & 295377                                                    \\ \cline{2-4} 
\multicolumn{1}{|l|}{\cellcolor[HTML]{ECF4FF}}                           & \cellcolor[HTML]{FCE6AB}F93     & \cellcolor[HTML]{D3FFB6}\textbf{722245}                        & 706744                                                    \\ \cline{2-4} 
\multicolumn{1}{|l|}{\cellcolor[HTML]{ECF4FF}}                           & \cellcolor[HTML]{FCE6AB}F94     & \cellcolor[HTML]{D3FFB6}\textbf{738970}                        & 719059                                                    \\ \cline{2-4} 
\multicolumn{1}{|l|}{\cellcolor[HTML]{ECF4FF}}                           & \cellcolor[HTML]{FCE6AB}F95     & \cellcolor[HTML]{D3FFB6}\textbf{46951.9}                       & 46737.7                                                   \\ \cline{2-4} 
\multicolumn{1}{|l|}{\cellcolor[HTML]{ECF4FF}}                           & \cellcolor[HTML]{FCE6AB}F96     & \cellcolor[HTML]{D3FFB6}\textbf{770539}                        & 751434                                                    \\ \cline{2-4} 
\multicolumn{1}{|l|}{\multirow{-19}{*}{\cellcolor[HTML]{ECF4FF}n = 300}} & \cellcolor[HTML]{FCE6AB}F97     & \cellcolor[HTML]{D3FFB6}\textbf{766150}                        & 752530                                                    \\ \hline
\multicolumn{1}{l|}{}                                                    & \cellcolor[HTML]{FFFFC7}Best    & \multicolumn{1}{c|}{\cellcolor[HTML]{D3FFB6}\textbf{87}}       & \multicolumn{1}{c|}{10}                                   \\
\multicolumn{1}{l|}{}                                                    & \cellcolor[HTML]{FFFFC7}Ranking & \multicolumn{1}{c|}{\cellcolor[HTML]{D3FFB6}\textbf{1.103093}} & \multicolumn{1}{c|}{1.896907}                             \\ \cline{2-4} 
\end{tabular}
\caption{Valores+Ranking Final + nProblemas}
\end{table}

\newpage
\KOMAoptions{paper=landscape,pagesize}
% Please add the following required packages to your document preamble:
% \usepackage[table,xcdraw]{xcolor}
% If you use beamer only pass "xcolor=table" option, i.e. \documentclass[xcolor=table]{beamer}
\begin{table}[]
\begin{tabular}{c|c|c|c|c|c|c|c|}
\cline{2-8}
\multicolumn{1}{l|}{}                                      & \cellcolor[HTML]{FFFFC7}\textbf{AGEU}                   & \cellcolor[HTML]{FFFFC7}\textbf{GACEP3103\_5\_4-2} & \cellcolor[HTML]{FFFFC7}\textbf{GACEPCHC}               & \cellcolor[HTML]{FFFFC7}\textbf{GACEPMutacion}          & \cellcolor[HTML]{FFFFC7}\textbf{GACEPTotal} & \cellcolor[HTML]{FFFFC7}\textbf{GACEPc50} & \cellcolor[HTML]{FFFFC7}\textbf{GACEPorGRASP} \\ \hline
\multicolumn{1}{|c|}{\cellcolor[HTML]{FCE6AB}\textbf{1}}   & 3.463918                                                & \cellcolor[HTML]{D3FFB6}\textbf{6.773196}          & \cellcolor[HTML]{FFCCC9}{\color[HTML]{000000} 2.896907} & 3.463918                                                & 3.463918                                    & 3.659794                                  & 4.278351                                      \\ \hline
\multicolumn{1}{|c|}{\cellcolor[HTML]{FCE6AB}\textbf{2}}   & 3.391753                                                & \cellcolor[HTML]{D3FFB6}\textbf{6.742268}          & \cellcolor[HTML]{FFCCC9}{\color[HTML]{000000} 2.845361} & 3.391753                                                & 3.391753                                    & 3.649485                                  & 4.587629                                      \\ \hline
\multicolumn{1}{|c|}{\cellcolor[HTML]{FCE6AB}\textbf{3}}   & 3.360825                                                & \cellcolor[HTML]{D3FFB6}\textbf{6.680412}          & \cellcolor[HTML]{FFCCC9}{\color[HTML]{000000} 2.56701}  & 3.360825                                                & 3.360825                                    & 3.896907                                  & 4.773196                                      \\ \hline
\multicolumn{1}{|c|}{\cellcolor[HTML]{FCE6AB}\textbf{5}}   & 3.505155                                                & \cellcolor[HTML]{D3FFB6}\textbf{6.340206}          & \cellcolor[HTML]{FFCCC9}{\color[HTML]{000000} 1.876289} & 3.505155                                                & 3.505155                                    & 3.917526                                  & 5.350515                                      \\ \hline
\multicolumn{1}{|c|}{\cellcolor[HTML]{FCE6AB}\textbf{10}}  & 3.690722                                                & 5.391753                                           & \cellcolor[HTML]{FFCCC9}{\color[HTML]{000000} 1.329897} & 3.690722                                                & 3.690722                                    & 3.886598                                  & \cellcolor[HTML]{D3FFB6}\textbf{6.319588}     \\ \hline
\multicolumn{1}{|c|}{\cellcolor[HTML]{FCE6AB}\textbf{20}}  & 3.561856                                                & 4.489691                                           & 4.71134                                                 & \cellcolor[HTML]{FFCCC9}{\color[HTML]{000000} 3.231959} & 3.412371                                    & 3.386598                                  & \cellcolor[HTML]{D3FFB6}\textbf{5.206186}     \\ \hline
\multicolumn{1}{|c|}{\cellcolor[HTML]{FCE6AB}\textbf{30}}  & 3.530928                                                & 4.561856                                           & 4.082474                                                & \cellcolor[HTML]{FFCCC9}{\color[HTML]{000000} 2.798969} & 3.71134                                     & 3.5                                       & \cellcolor[HTML]{D3FFB6}\textbf{5.814433}     \\ \hline
\multicolumn{1}{|c|}{\cellcolor[HTML]{FCE6AB}\textbf{40}}  & 2.427835                                                & 5.324742                                           & 3.247423                                                & \cellcolor[HTML]{FFCCC9}{\color[HTML]{000000} 2.396907} & 4.28866                                     & 4.06701                                   & \cellcolor[HTML]{D3FFB6}\textbf{6.247423}     \\ \hline
\multicolumn{1}{|c|}{\cellcolor[HTML]{FCE6AB}\textbf{50}}  & \cellcolor[HTML]{FFCCC9}{\color[HTML]{000000} 2.376289} & 5.572165                                           & 2.917526                                                & 2.469072                                                & 4.226804                                    & 4.314433                                  & \cellcolor[HTML]{D3FFB6}\textbf{6.123711}     \\ \hline
\multicolumn{1}{|c|}{\cellcolor[HTML]{FCE6AB}\textbf{60}}  & \cellcolor[HTML]{FFCCC9}{\color[HTML]{000000} 2.340206} & \cellcolor[HTML]{D3FFB6}\textbf{5.963918}          & 2.896907                                                & 2.57732                                                 & 4.154639                                    & 4.108247                                  & 5.958763                                      \\ \hline
\multicolumn{1}{|c|}{\cellcolor[HTML]{FCE6AB}\textbf{70}}  & \cellcolor[HTML]{FFCCC9}{\color[HTML]{000000} 2.43299}  & \cellcolor[HTML]{D3FFB6}\textbf{6.489691}          & 2.917526                                                & 2.515464                                                & 3.979381                                    & 3.984536                                  & 5.680412                                      \\ \hline
\multicolumn{1}{|c|}{\cellcolor[HTML]{FCE6AB}\textbf{80}}  & \cellcolor[HTML]{FFCCC9}{\color[HTML]{000000} 2.365979} & \cellcolor[HTML]{D3FFB6}\textbf{6.582474}          & 3.072165                                                & 2.510309                                                & 3.979381                                    & 3.891753                                  & 5.597938                                      \\ \hline
\multicolumn{1}{|c|}{\cellcolor[HTML]{FCE6AB}\textbf{90}}  & \cellcolor[HTML]{FFCCC9}{\color[HTML]{000000} 2.278351} & \cellcolor[HTML]{D3FFB6}\textbf{6.840206}          & 3.164948                                                & 2.56701                                                 & 3.948454                                    & 3.809278                                  & 5.391753                                      \\ \hline
\multicolumn{1}{|c|}{\cellcolor[HTML]{FCE6AB}\textbf{100}} & \cellcolor[HTML]{FFCCC9}{\color[HTML]{000000} 2.350515} & \cellcolor[HTML]{D3FFB6}\textbf{6.840206}          & 3.134021                                                & 2.546392                                                & 3.938144                                    & 3.819588                                  & 5.371134                                      \\ \hline
\end{tabular}
\caption{Ranking Mejor-Peor}
\end{table}

\newpage
% Please add the following required packages to your document preamble:
% \usepackage[table,xcdraw]{xcolor}
% If you use beamer only pass "xcolor=table" option, i.e. \documentclass[xcolor=table]{beamer}
\begin{table}[H]
\begin{tabular}{c|c|c|c|c|c|c|c|
>{\columncolor[HTML]{ECF4FF}}c |}
\cline{2-9}
\multicolumn{1}{l|}{}                                                    & \cellcolor[HTML]{FFFFC7}\textbf{AGEU}    & \cellcolor[HTML]{FFFFC7}\textbf{GACEP3103} & \cellcolor[HTML]{FFFFC7}\textbf{GACEPCHC} & \cellcolor[HTML]{FFFFC7}\textbf{Mutacion} & \cellcolor[HTML]{FFFFC7}\textbf{GACEPTotal} & \cellcolor[HTML]{FFFFC7}\textbf{c50} & \cellcolor[HTML]{FFFFC7}\textbf{orGRASP} & \cellcolor[HTML]{FCE6AB}\textbf{Error Acumulado(\%)} \\ \hline
\multicolumn{1}{|c|}{\cellcolor[HTML]{FFFFC7}\textbf{AGEU}}              & 1                                        & {\color[HTML]{FF0000} 3.54E-17}                    & 0.655466                                  & {\color[HTML]{FF0000} 0.009625}                & {\color[HTML]{FF0000} 6.79E-07}             & {\color[HTML]{FF0000} 1.21E-06}           & {\color[HTML]{0000FF} 0}                      & 26.49081                                                  \\ \hline
\multicolumn{1}{|c|}{\cellcolor[HTML]{FFFFC7}\textbf{GACEP3103}} & {\color[HTML]{0000FF} \textbf{3.54E-17}} & 1                                                  & {\color[HTML]{0000FF} \textbf{1.64E-17}}  & {\color[HTML]{0000FF} \textbf{3.77E-17}}       & {\color[HTML]{0000FF} \textbf{3.77E-17}}    & {\color[HTML]{0000FF} \textbf{2.72E-17}}  & {\color[HTML]{0000FF} \textbf{2.78E-16}}      & 26.49081                                                  \\ \hline
\multicolumn{1}{|c|}{\cellcolor[HTML]{FFFFC7}\textbf{GACEPCHC}}          & 0.655466                                 & {\color[HTML]{FF0000} 1.64E-17}                    & 1                                         & 0.735187                                       & {\color[HTML]{FF0000} 0.001855}             & {\color[HTML]{FF0000} 0.003562}           & {\color[HTML]{FF0000} 9E-10}                  & 26.49081                                                  \\ \hline
\multicolumn{1}{|c|}{\cellcolor[HTML]{FFFFC7}\textbf{Mutacion}}     & {\color[HTML]{0000FF} \textbf{0.009625}} & {\color[HTML]{FF0000} 3.77E-17}                    & 0.735187                                  & 1                                              & {\color[HTML]{FF0000} 1.53E-06}             & {\color[HTML]{FF0000} 3.46E-06}           & {\color[HTML]{0000FF} 0}                      & 26.49081                                                  \\ \hline
\multicolumn{1}{|c|}{\cellcolor[HTML]{FFFFC7}\textbf{GACEPTotal}}        & {\color[HTML]{0000FF} \textbf{6.79E-07}} & {\color[HTML]{FF0000} 3.77E-17}                    & {\color[HTML]{0000FF} \textbf{0.001855}}  & {\color[HTML]{0000FF} \textbf{1.53E-06}}       & 1                                           & 0.233648                                  & {\color[HTML]{0000FF} 0}                      & 26.49081                                                  \\ \hline
\multicolumn{1}{|c|}{\cellcolor[HTML]{FFFFC7}\textbf{GACEPc50}}          & {\color[HTML]{0000FF} \textbf{1.21E-06}} & {\color[HTML]{FF0000} 2.72E-17}                    & {\color[HTML]{0000FF} \textbf{0.003562}}  & {\color[HTML]{0000FF} \textbf{3.46E-06}}       & 0.233648                                    & 1                                         & {\color[HTML]{0000FF} 0}                      & 26.49081                                                  \\ \hline
\multicolumn{1}{|c|}{\cellcolor[HTML]{FFFFC7}\textbf{GACEPorGRASP}}      & {\color[HTML]{0000FF} \textbf{0}}        & {\color[HTML]{FF0000} 2.78E-16}                    & {\color[HTML]{0000FF} \textbf{9E-10}}     & {\color[HTML]{0000FF} \textbf{0}}              & {\color[HTML]{0000FF} \textbf{0}}           & {\color[HTML]{0000FF} \textbf{0}}         & 1                                             & 26.49081                                                  \\ \hline
\end{tabular}
\caption{Wilcoxon}
\end{table}

\newpage
\KOMAoptions{paper=portrait,pagesize}
% Please add the following required packages to your document preamble:
% \usepackage[table,xcdraw]{xcolor}
% If you use beamer only pass "xcolor=table" option, i.e. \documentclass[xcolor=table]{beamer}
\begin{table}[H]
\begin{tabular}{|
>{\columncolor[HTML]{FFFFC7}}c |l|c|c|c|}
\hline
\multicolumn{1}{|l|}{\cellcolor[HTML]{ECF4FF}Algoritmos vs GACEP3103} & \multicolumn{1}{c|}{\cellcolor[HTML]{FFFC9E}\textbf{Original}} & \cellcolor[HTML]{FFFC9E}\textbf{Holm}    & \cellcolor[HTML]{FFFC9E}\textbf{Hommel}  & \cellcolor[HTML]{FFFC9E}\textbf{Hochberg} \\ \hline
\textbf{AGEU}                                                         & 3.54446E-17                                                    & {\color[HTML]{0000FF} \textbf{1.42E-16}} & {\color[HTML]{0000FF} \textbf{7.09E-17}} & {\color[HTML]{0000FF} \textbf{7.54E-17}}  \\ \hline
\textbf{GACEPCHC}                                                     & 1.63701E-17                                                    & {\color[HTML]{0000FF} \textbf{9.82E-17}} & {\color[HTML]{0000FF} \textbf{5.03E-17}} & {\color[HTML]{0000FF} \textbf{7.54E-17}}  \\ \hline
\textbf{GACEPMutacion}                                                & 3.7691E-17                                                     & {\color[HTML]{0000FF} \textbf{1.42E-16}} & {\color[HTML]{0000FF} \textbf{7.54E-17}} & {\color[HTML]{0000FF} \textbf{7.54E-17}}  \\ \hline
\textbf{GACEPTotal}                                                   & 3.7691E-17                                                     & {\color[HTML]{0000FF} \textbf{1.42E-16}} & {\color[HTML]{0000FF} \textbf{7.54E-17}} & {\color[HTML]{0000FF} \textbf{7.54E-17}}  \\ \hline
\textbf{GACEPc50}                                                     & 2.7223E-17                                                     & {\color[HTML]{0000FF} \textbf{1.36E-16}} & {\color[HTML]{0000FF} \textbf{5.65E-17}} & {\color[HTML]{0000FF} \textbf{7.54E-17}}  \\ \hline
\textbf{GACEPorGRASP}                                                 & 2.78282E-16                                                    & {\color[HTML]{0000FF} \textbf{2.78E-16}} & {\color[HTML]{0000FF} \textbf{2.78E-16}} & {\color[HTML]{0000FF} \textbf{2.78E-16}}  \\ \hline
\end{tabular}
\caption{Comparaciones múltiples}
\end{table}
%
\chapter{Conclusiones y trabajo futuro}

El caso más habitual que se da en el diseño de algoritmos que se encuentra en la literatura implica tomar un algoritmo que es de por sí competitivo en lo relativo, al menos, al problema que se está tratando como base para el desarrollo que se va a realizar y se concluye con un análisis comparativo entre el propio algoritmo resultado del diseño y otros algoritmos competitivos de la literatura. 
Sin embargo, esto no ha sido posible en nuestro caso debido a la inexistencia de un algoritmo previo que tuviese como objetivo la resolución de problemas de optimización combinatorios \textit{expensive}. 
Por ello, este trabajo resulta ser un estudio novedoso en el ámbito de la resolución de dichos problemas: se ha tenido que empezar desde cero, desde un algoritmo genético clásico y básico como es el AGE que no aporta especialmente buenos resultados en este ámbito de trabajo, y repasar campos de estudio de diferentes problemas con el fin de obtener inspiración para superar los inconvenientes que han surgido a lo largo del desarrollo. 

Por este motivo, probablemente el mayor reto ha devenido en lidiar con los problemas propios de un proceso creativo (falta de inspiración, resultados insuficientes, \dots) y en la naturaleza de un proyecto individual de tan larga duración donde la planificación y el trabajo constante han sido vitales para lograr presentar un proyecto de calidad finalizado. 
En este sentido, las prácticas desarrollados previamente a lo largo del doble grado (principalmente en las asignaturas del Grado de Ingeniería Informática, donde la realización de diversos trabajos para la evaluación de la parte práctica eran frecuentes), si bien de menor envergadura y mucho más guiados, fueron los cimientos para afrontar dichas situaciones adversas y superarlas adecuadamente. 

Por lo tanto, podemos afirmar que este trabajo no solo destaca por presentar un algoritmo con el que se ha conseguido alcanzar satisfactoriamente el objetivo principal de este proyecto, esto es, el diseño de una metaheurística útil para ser aplicada en problemas combinatorios \textit{expensive}, sino que lo hace también por lo siguiente:

\begin{itemize}
	\item Se detalla cómo debe ser un proceso creativo desde el inicio cuyo objetivo sea el desarrollo de un algoritmo de esta clase.
	
	\item Se trata un problema que no ha sido explorado con anterioridad, por lo que es un estudio sin antecedentes en su campo.
	
	\item Se trata de un estudio con una gran utilidad, debido al creciente uso de problemas combinatorios en problemas complejos. 
\end{itemize}

Además, ha resultado un proyecto muy completo que ha posibilitado no solo afianzar los conocimientos teóricos adquiridos durante el doble grado, sino también ampliarlos y ponerlos en práctica: 
obviamente ha sido necesario un predominante conocimiento sobre asignaturas orientadas al estudio de algoritmos (como sería Metaheurística y Algorítmica), se han analizado resultados usando métodos con bases matemáticas (en el ámbito de la estadística, análisis y matemáticas aplicadas), se han consultado numerosas fuentes bibliográficas, se ha utilizado uno de los lenguajes de programación que más he usado durante estos cinco años (\texttt{C++}), se han utilizado herramientas experimentales\dots 
Como se puede comprobar, los conocimientos necesarios para llevar a cabo estas tareas provienen de una gran variedad de asignaturas del doble grado, evidenciando así la condición transversal de este proyecto. 

Personalmente he disfrutado en gran medida el experimentar de primera mano cómo es el procedimiento a seguir cuando se quiere desarrollar algo nuevo y propio, aunque ha habido muchos momentos de frustración porque no lograba alcanzar los resultados deseados, todos palidecían en comparación a la alegría que se siente cuando tras tanto esfuerzo consigues superar el problema con el que te enfrentabas. 
También me ha parecido muy interesante todo el proceso de búsqueda de información/inspiración, ya que siempre me ha llamado la atención la cantidad de ideas tan diferentes que se le pueden ocurrir a distintas personas cuando afrontan el mismo problema; esto me permite no solo aprender más acerca de los distintos operadores que se podrían implementar, sino también me permite descubrir otros puntos de vista, aumentando mis horizontes y mejorando mi capacidad de razonamiento. 
Sin embargo, la parte de documentar todo el trabajo en esta memoria se me ha hecho bastante tedioso, ya que era solo formalizar cosas que ya sabía y veía muy obvias por haber estado un curso entero trabajando en ello; supongo que así es cómo se sienten muchos profesores cuando intentan explicarnos cosas que para ellos son obvias y nosotros seguimos sin enterarnos. 

Para finalizar, cabe destacar que sigue siendo necesario realizar más estudios sobre cómo resolver problemas combinatorios \textit{expensive}, dado que son problemas con una gran cantidad de aplicaciones en el mundo real, y que el algoritmo aquí presentado, como cualquier otra metaheurística existente, no está terminado, puesto que siempre queda margen de mejora. 
Sería interesante que se partiese del algoritmo propuesto en este trabajo y se realizasen las modificaciones sobre él y/o se utilizase para realizar comparaciones de los nuevos algoritmos que se desarrollasen (ya que al fin y al cabo es el único existente para este tipo de problemas en el momento en que se está escribiendo esto). 
Algunas de las modificaciones que podrían ser interesantes de comprobar serían, entre otras:
\begin{itemize}
	\item Estudiar y aplicar otros operadores de cruce, ya que ha sido la única característica propia del AGE que se ha mantenido hasta el algoritmo final propuesto. 
	\item Estudiar la aplicación de parámetros auto-adaptativos que no requieran que el usuario lo pre-ajuste, ya que en la literatura se ha indicado como un enfoque prometedor de desarrollo de algoritmo genéticos. 
\end{itemize}
Es importante mencionar que no siempre tener un algoritmo más complejo implica que se vayan a obtener mejores resultados.
%%\chapter{Conclusiones y Trabajos Futuros}
%
%
%%\nocite{*}
\printbibliography
%
%\appendix
%\input{apendices/manual_usuario/manual_usuario}
%%\input{apendices/paper/paper}
%\input{glosario/entradas_glosario}
% \addcontentsline{toc}{chapter}{Glosario}
% \printglossary
\chapter*{}
\thispagestyle{empty}

\end{document}
