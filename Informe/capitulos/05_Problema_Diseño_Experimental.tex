\chapter{Problema y Diseño Experimental}

En este capítulo se presenta el problema que se aborda con el algoritmo creado, así como las necesidades que surgen por el interés de identificar el aporte de este frente al panorama actual del campo en términos de rendimiento, las soluciones que existen para satisfacer dicha necesidad y cuál de ellas es la más adecuada.

\section{Descripción del problema}

El problema que se va a abordar en este proyecto es el de la optimización de problemas de tipo combinatorio. 

\begin{definicion}
Un \textbf{problema de optimización combinatoria} se define como aquel en el que el conjunto de soluciones posibles es discreto. 
Es decir, se trata de un problema de optimización que involucra una cantidad finita o numerable de soluciones posibles.
\end{definicion}
Este tipo de problemas se diferencia de los problemas de optimización continuos, en los cuales el conjunto de soluciones posibles es infinito e incontable. 

Dentro del campo de la optimización combinatoria es común que la mayoría de los procesos de resolución de problemas no puedan garantizar la solución óptima, incluso dentro del contexto del modelo que se esté utilizando. 
Sin embargo, la aproximación al óptimo suele ser suficiente para resolver los problemas en la práctica. 

Con el fin de poder estudiar más a fondo el comportamiento de los distintos algoritmos que se han estado desarrollando era necesario elegir un problema determinado con el que trabajar. 
En nuestro caso, se ha elegido la generalización del problema comúnmente llamado ``problema de la mochila'' (\textit{Knapsack Problem} (KP)):  \textbf{\textit{Quadratic Knapsack Problem} (QKP)}. 

%Justificación de por qué hemos usado QKP
Antes de definir el problema, justificaremos por qué se ha elegido este problema. 
La primera razón, y posiblemente la más importante, es la ausencia de \textit{benchmarks} para problemas combinatorios \textit{expensive}, por lo que nos hemos visto obligados a crear el nuestro propio para un futuro. 
Ante este panorama también debemos encontrar un problema específico adecuado sobre el que trabajar desde cero y que resulte de interés. 
Como el objetivo inicial estaba relacionado con redes neuronales, buscamos un problema que puede tener representación binaria (\texttt{1} para la elección de elementos, \texttt{0} para el caso contrario). 
En este sentido, el QKP cumple con este requisito, además tiene interés añadido debido a que es un problema con restricciones y constituye una alternativa moderna de un problema clásico; además de que podemos generar instancias de este problema con distintos tamaños. 
Además, es un problema que tiene muchas aplicaciones en el mundo real en situaciones donde los recursos con distintos niveles de interacción tienen que distribuirse entre distintas tareas, por ejemplo, asignar compañeros de equipo a distintos proyectos donde las contribuciones de cada miembro se consideran de forma individual y por parejas. 
Por lo tanto, teniendo en cuenta la falta de \textit{benchmarks} y referencias, el QKP resulta ser una buena opción, ya que es un problema difícil, costoso y moderno. 


\subsection{Quadratic Knapsack Problem}
%The quadratic knapsack problem—a survey
Se procede a definir en profundidad dicho problema. 
En primer lugar, se tienen $n$ elementos donde cada elemento $j$ tiene un peso entero positivo $w_j$. 
Adicionalmente, se nos da una matriz de enteros no negativos de tamaño $n\times n$, $P = \{p_{ij}\}$, donde $p_{jj}$ es el beneficio asociado a elegir el elemento $j$ y $p_{ij}+p_{ji}$ es el beneficio que se alcanza si ambos elementos $i,j$, con $i<j$ son seleccionados. 
Consideramos que una combinación de elementos es una solución a QKP cuando peso el total (la suma del peso de todos los elementos seleccionados) no superan la capacidad máxima de la mochila dada, $c$. 
Así, el problema consiste el maximizar el beneficio total sin sobrepasar la capacidad máxima.

Por conveniencia en la notación, establecemos que $N=\{1,\dots,n\}$ denotará el conjunto de elementos. 
Representando la lista de elementos de forma binaria, $x_j$, para indicar si el elemento $j$ ha sido seleccionado (su valor será 0 si no ha sido seleccionado, 1 en caso contrario), el problema podrá ser formulado de la siguiente forma:
\begin{equation}
\begin{aligned}
  \text{maximizar} & \sum_{i\in N}\sum_{j\in N}p_{ij}x_ix_j \\
  \text{sujeto a } & \sum_{j\in N}w_jx_j \leq c\\
  & x_j\in \{0,1\}, j \in N 
\end{aligned}
\label{eq:QKP}
\end{equation}

Sin pérdida de generalidad, podemos suponer que:
\begin{itemize}
	\item $\max_{j\in N} w_j \leq c < \sum_{j\in N}w_j$
	\item La matriz de beneficios es simétrica, es decir, $p_{ij} = p_{ji}$, $\forall j > i$.
\end{itemize}

Una vez definido el problema, es fácil ver por qué es una versión generalizada del KP. 
KP se puede obtener a partir de QKP si $p_{ij} = 0$, para todo $i\neq j$. 
También se considera una versión restringida del \textit{Quadratic 0-1 Programming Problem} (QP), el cual se define como \ref{eq:QKP} sin la restricción de capacidad.

Como uno cabría esperar, debido a su generalidad, el QKP tiene un amplio espectro de aplicaciones. 
Witzgall \parencite{witzgallMathematicalMethodsSite1975} presentó un problema que surge en telecomunicación cuando un número de localizaciones para satélites tienen que ser seleccionados, tales que el tráfico global entre estas estaciones se maximice y la limitación de presupuesto se cumpla; este problema resulta ser un QKP. 

%En tanto que QKP es un problema $\mathcal{NP}-hard$, no podemos esperar encontrar una aproximación totalmente polinómica a no ser que $\mathcal{NP}=\mathcal{P}$. 
%Sin embargo, Rader y Woeginger %referencias
%desarrollaron un esquema de aproximación en tiempo completamente polinómico (FPTAS, \textit{Fully polynomial-time approximation scheme}) para el caso especial donde todos los beneficios $p_{ij}\geq 0$ y donde 

\subsection{Datos del problema}

Utilizaremos 97 archivos de datos generados aleatoriamente de %pedir referencia a página
, los cuales se pueden distribuir de forma que se indica en la Tabla \ref{DatosProblema}



Se entiende como ``densidad'' al porcentaje de beneficios combinados positivos, es decir, $p_{ij} > 0$. 
Particularmente QP tendría densidad 0\%.

Ahora bien, todos los archivos tienen el mismo formato, lo cual resulta útil para definir funciones capaces de obtener los datos más relevantes para nuestros algoritmos. 
El formato que siguen los archivos es el siguiente:
\begin{itemize}
	\item La referencia de la instancia: Su nombre.
	\item El número de variables ($n$)
	\item Los coeficientes lineales de la función objetivo $p_{jj}$
	\item Los coeficientes cuadráticos $p_{ij}$: representados en líneas
	\item Una línea en blanco
	\item 0 si la restricción es de tipo $\leq$, lo cual siempre va a ocurrir ya que estamos considerando instancias QKP.
	\item La capacidad $c$ de la mochila.
	\item Los coeficientes de capacidad/peso, $w_j$.
	\item Algunos comentarios.
\end{itemize}

\begin{table}[H]
\caption{Datos del Problema}
\label{DatosProblema}
\begin{tabular}{|c|c|c|}
\hline
\rowcolor[HTML]{F7EAC7} 
\multicolumn{1}{|l|}{\cellcolor[HTML]{F7EAC7}Número de variables} & \multicolumn{1}{l|}{\cellcolor[HTML]{F7EAC7}Densidad} & \multicolumn{1}{l|}{\cellcolor[HTML]{F7EAC7}Número de archivos} \\ \hline
\rowcolor[HTML]{DDFDFF} 
\cellcolor[HTML]{DAE8FC}                                          & 25\%                                                  & 10                                                              \\ \cline{2-3} 
\cellcolor[HTML]{DAE8FC}                                          & 50\%                                                  & 10                                                              \\ \cline{2-3} 
\rowcolor[HTML]{DDFDFF} 
\cellcolor[HTML]{DAE8FC}                                          & 75\%                                                  & 10                                                              \\ \cline{2-3} 
\multirow{-4}{*}{\cellcolor[HTML]{DAE8FC}n = 100}                 & 100\%                                                 & 9                                                              \\ \hline
\rowcolor[HTML]{DAE8FC} 
\cellcolor[HTML]{DDFDFF}                                          & 25\%                                                  & 9                                                              \\ \cline{2-3} 
\cellcolor[HTML]{DDFDFF}                                          & 50\%                                                  & 10                                                              \\ \cline{2-3} 
\rowcolor[HTML]{DAE8FC} 
\cellcolor[HTML]{DDFDFF}                                          & 75\%                                                  & 10                                                              \\ \cline{2-3} 
\multirow{-4}{*}{\cellcolor[HTML]{DDFDFF}n = 200}                 & 100\%                                                 & 10                                                              \\ \hline
\rowcolor[HTML]{DDFDFF} 
\cellcolor[HTML]{DAE8FC}                                          & 25\%                                                  & 9                                                              \\ \cline{2-3} 
\multirow{-2}{*}{\cellcolor[HTML]{DAE8FC}n = 300}                 & 50\%                                                  & 10                                                              \\ \hline
\end{tabular}
\end{table}

%Introducir algún ejemplo?

\section{Diseño Experimental}

\subsection{Criterio de Parada}

Se han elegido las instancias mencionadas anteriormente ya que también se proporcionaban algunos resultados de otros algoritmos, por lo que resultaba conveniente a la hora de comprobar si los resultados obtenidos por nuestros algoritmos bases eran competitivos. 
Ya que, en el caso de que no lo fuesen, tendríamos que buscar otros algoritmos base sobre los que trabajar.
Sin embargo, nos encontramos con un problema, esto es, el criterio de parada presentado en estos casos es el tiempo.
Adicionalmente, el tiempo de parada también depende del número de elementos, $n$, de forma que se tiene:
\begin{itemize}
\item Para $n = 100$, se tienen \textbf{5 segundos} de ejecución.
\item Para $n > 100$, en nuestro caso, $n = 200$ y $n = 300$, se tienen \textbf{30 segundos} de ejecución.
\end{itemize}

Como se ha indicado antes, utilizar el tiempo de ejecución como criterio de parada resulta un problema. 
Esto se debe a que no es un criterio de parada fiable, ya que depende de la capacidad de computación de cada ordenador. 
No es comparable la velocidad de los ordenadores actuales con la velocidad de los ordenadores de dentro de 10 años, de la misma no podemos comparar el rendimiento de un ordenador de hace 10 años con respecto a uno actual. 
E incluso dentro de los ordenadores de la misma generación, dependerá de las características propias de cada computador. 
Por lo que se llega a la conclusión de que si se quiere que los resultados obtenidos en este trabajo puedan ser usados como referencia, o incluso si se quiere recrear el trabajo, en un futuro se debe cambiar el criterio de parada a algo portable, a algo que sea independiente de cuándo se produzca el experimento. 
Por ello, se ha decidido cambiar el criterio de parada a un número de iteraciones máximo. 
Este criterio sí es portable, ya que independientemente de qué tipo de computador se utilice para obtener los resultados siempre se van a obtener los mismos resultados utilizando los mismos parámetros.

Primero debemos indicar lo que entenderemos por iteraciones. 
Denotaremos como iteraciones al número de veces que se repite el procedimiento completo un determinado algoritmo, en nuestro caso se va a traducir en lo siguiente: 
\begin{itemize}
	\item Para el algoritmo \textit{Random}, el número iteraciones será el número de veces que generemos una solución de forma aleatoria.
	\item Para los algoritmos genéticos, una iteración puede suponer un número distinto de evaluaciones. 
	En los algoritmos genéticos al número de iteraciones se llama también generaciones.
\end{itemize}

Dicho esto, para elegir el número de iteraciones máximo que se iba a utilizar se tomó como referencia el tiempo establecido anteriormente. 
En tanto se tenía elegir un número de iteraciones como criterio de parada, se decidió estudiar a qué equivalía actualmente el criterio de parada por tiempo establecido. 
Para ello, mediante una variable \texttt{contador}, se ejecutaron todos los archivos con el tiempo como criterio de parada y se mostraba por pantalla el número de iteraciones que se había alcanzado. 
Tras realizar una media de todos estos datos y redondearlo, obtenemos que el nuevo criterio de parada es \textbf{90000 iteraciones} para todos los archivos. 
El que el número de iteraciones máximo no dependa del número de elementos como lo hacía el tiempo de ejecución máximo se puede justificar en tanto que se necesita más tiempo para realizar todos los cálculos si aumenta el número de datos. 

Además, para asegurarnos que realmente ambos criterios de parada eran equivalentes, se ejecutaron todos los archivos usando el algoritmo base con criterio de parada por iteraciones y por tiempo. 
Nótese que no solo se almacenan las soluciones finales, si no que también se almacenan las soluciones intermedias llegado a ciertos porcentajes de la ejecución. 
Tras esto, se compararon ambos resultados y se pudo comprobar que, efectivamente, eran equivalentes. 

Por lo tanto, se puede decir con seguridad que los cambios que apliquemos a un criterio de parada en específico se puede traducir a un cambio en el otro criterio de parada. 
Esto cabe la pena destacarlo ya que, recordemos, el objetivo de este trabajo es crear un algoritmo útil y competitivo para tratar con problemas \textit{expensive}, por lo que queremos obtener buenos resultados en un tiempo muy reducido. 

Para simular de forma eficiente esta reducción del tiempo, se propuso el siguiente cambio con respecto al tiempo como criterio de parada:
\begin{itemize}
	\item En vez de ejecutar los archivos con $n=100$ durante 5 segundos, lo reduciremos a 25ms.
	\item En vez de ejecutar los archivos con $n>100$ durante 30 segundos, lo reduciremos a 150ms.
\end{itemize}
Haciendo cálculos, obtenemos que solo utilizaremos el 0.5\% inicial de cada ejecución, lo que podemos traducir en que nuestro nuevo criterio de parada serán \textbf{450 iteraciones}. 

Con el fin de comprobar que esta reducción había sido suficiente y necesaria, se ejecutan de nuevo todos los archivos con el algoritmo base ahora con 450 iteraciones y comparamos los resultados con los obtenidos anteriormente con 90000 iteraciones. 
Podemos ver que, efectivamente, se han obtenido resultados mucho peores. 

Finalmente, ya hemos establecido nuestro criterio de parada escalable y tenemos unos resultados base que utilizar. 
A partir de esto, nuestro objetivo será mejorar estos resultados lo máximo posible.

\subsection{Parámetros}

La ejecución del programa no requiere de ningún tipo de parámetro que se tenga que introducir manualmente. 
Todos los parámetros que se nombren a continuación vienen definido dentro del código del programa \texttt{main} como constantes globales o que dependen del propio problema. 

En primer lugar, hablemos sobre las constantes globales:
\begin{itemize}
	\item \texttt{NEVALUACIONESMAX}: Es el número de iteraciones máximas, mantendremos su valor a 450 por lo explicado en el apartado anterior. 
	\item \texttt{NTRIES}: Para que nuestros resultados no sean muy dependientes de la aleatoriedad, es necesario ejecutar cada archivo cierta cantidad de veces y obtener la media. 
Esta media será verdaderamente el resultado que guardaremos y compararemos con el resto. 
De forma más o menos arbitraria, se ha establecido su valor a 50.
	\item \texttt{INITSEED}: En relación con la constante anterior, es necesario no utilizar siempre la misma semilla de aleatoriedad, ya que eso resultaría en obtener siempre los mismos resultados, por lo que ejecutarlos varias veces sería una pérdida de tiempo. 
Así pues, tendremos que utilizar distintas semillas para cada ejecución. 
Además, por conveniencia, es recomendable que las semillas utilizadas sean comunes a todas las sucesivas ejecuciones de los distintos algoritmos (una vez más, para asegurarnos que la aleatoriedad afecta a todos los algoritmos de forma similar) y que siempre se obtengan los mismos resultados aún si ejecutamos el mismo archivo varias veces. 
Por ello, una solución es establecer un valor de semilla inicial arbitrario y utilizar este para generar el resto de semillas.
\end{itemize}

En definitiva, podemos resumir los parámetros, para lo que se usan y sus valores en la Tabla \ref{Resumen}.

\begin{table}
\caption{Resumen de parámetros utilizados}
\label{Resumen}
\begin{tabular}{|l|l|l|}
\hline
\rowcolor[HTML]{F7EAC7} 
Parámetros                                 & Resumen                                                                                                                                   & Valor                                                                                 \\ \hline
\rowcolor[HTML]{DAE8FC} 
\texttt{NEVALUACIONESMAX} & \begin{tabular}[c]{@{}l@{}}Criterio de parada: Número de\\ Iteraciones Máximas para \\ cada algoritmo\end{tabular}                       & 450                                                                                   \\ \hline
\rowcolor[HTML]{DDFDFF} 
\texttt{NTRIES}           & \begin{tabular}[c]{@{}l@{}}Número de veces que se va a \\ ejecutar el algoritmo por cada\\ archivo\end{tabular}                           & 50                                                                                    \\ \hline
\rowcolor[HTML]{DAE8FC} 
\texttt{INITSEED}         & \begin{tabular}[c]{@{}l@{}}Semilla inicial para poder \\ generar el resto de semillas que\\ utilizaremos para los algoritmos\end{tabular} & 5                                                                                     \\ \hline
\rowcolor[HTML]{DDFDFF} 
\texttt{EvaluacionMAX}    & \begin{tabular}[c]{@{}l@{}}Criterio de parada: Número de\\ Iteraciones Máximas dicho\\ algoritmo\end{tabular}                            & \texttt{NEVALUACIONESMAX}                                           \\ \hline
\rowcolor[HTML]{DAE8FC} 
\texttt{Semilla}          & \begin{tabular}[c]{@{}l@{}}Semilla de aleatoriedad para\\ determinada ejecución del \\ algoritmo\end{tabular}                             & \begin{tabular}[c]{@{}l@{}}Valor generado aleatoriamente \\ por \texttt{INITSEED}\end{tabular} \\ \hline
\rowcolor[HTML]{DDFDFF} 
%\texttt{numcro}             & \begin{tabular}[c]{@{}l@{}}Número de elementos de la\\ población de soluciones\end{tabular}                                               & 10                                                                                    \\ \hline
%\rowcolor[HTML]{DAE8FC} 
%\texttt{pmut}             & \begin{tabular}[c]{@{}l@{}}Probabilidad de mutación\\ de las soluciones de la población\\ ({[}0,1{]})\end{tabular}                        & 0.1                                                                                   \\ \hline
\end{tabular}
\end{table}
