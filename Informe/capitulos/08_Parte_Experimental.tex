\chapter{Parte Experimental (In progress)}
Adicionalmente, como se ha comentado antes, cada algoritmo requiere de sus propios parámetros. 
Aunque explicaremos en más profundidad cada algoritmo en los siguientes capítulos, es necesario que establezcamos ahora cuáles son dichos parámetros y qué valores se han utilizado. 
Para ello, primero vamos a resumir en la siguiente tabla (\ref{table:Parametros}) qué parámetros usa cada algoritmo y después explicaremos qué es cada uno y qué valor tienen asignado.

\begin{table}[H]
\begin{tabular}{|l|l|l|l|l|}
\hline
\rowcolor[HTML]{F7EAC7} 
Algoritmo          & EvaluacionMAX & Semilla     & Nº Elementos & Probabilidad mutación \\ \hline
\rowcolor[HTML]{DAE8FC} 
\textbf{Random}    & \textbf{Sí}   & \textbf{Sí} & No           & No                    \\ \hline
\rowcolor[HTML]{DDFDFF} 
\textbf{AGEU}      & \textbf{Sí}   & \textbf{Sí} & \textbf{Sí}  & \textbf{Sí}           \\ \hline
\rowcolor[HTML]{DAE8FC} 
\textbf{GACEP}     & \textbf{Sí}   & \textbf{Sí} & \textbf{Sí}  & \textbf{Sí}           \\ \hline
\rowcolor[HTML]{DDFDFF} 
\textbf{CHC}       & \textbf{Sí}   & \textbf{Sí} & \textbf{Sí}  & No                    \\ \hline
\rowcolor[HTML]{DAE8FC} 
\textbf{GACEPCHC}  & \textbf{Sí}   & \textbf{Sí} & \textbf{Sí}  & \textbf{Sí}           \\ \hline
\rowcolor[HTML]{DDFDFF} 
\textbf{GACEP3103} & \textbf{Sí}   & \textbf{Sí} & \textbf{Sí}  & \textbf{Sí}           \\ \hline
\end{tabular}
\label{table:Parametros}
\caption{Parámetros utilizados por cada algoritmo}
\end{table}

\begin{itemize}
	\item \texttt{EvaluacionMAX}: Es el número de evaluaciones máximas para dicho algoritmo. 
Su valor será el de \texttt{NEVALUACIONESMAX}.
	\item \texttt{Semilla}: Como se ha indicado anteriormente, es necesario establecer una semilla de aleatoriedad en cada ejecución del algoritmo. 
Por lo tanto, el valor de este parámetro será el de la semilla generada por \texttt{INITSEED}. 
	\item \texttt{Nº Elementos}: Número de soluciones que va a constituir una población. 
Es necesario tener una población pequeña con el fin de utilizar el menor número de evaluaciones posible, pero suficiente como para poder trabajar con cierto margen. 
Por lo tanto, estableceremos el tamaño de población a 10. 
	\item \texttt{Probabilidad mutación}: En algunos algoritmos intentaremos modificar un poco alguna solución en cada iteración. 
Este valor establece el porcentaje de soluciones que la población que mutará. 
Genéricamente este valor suele ser del 0.1, por lo que también lo utilizaremos en nuestro problema. 
Correspondería que solo mutaría una solución de la población por cada generación. 
\end{itemize}
\section{Algoritmos de Referencia experimentación}

\section{Resultados versión expensive}

\section{Incorporación del histórico}

\section{Uso de GRASP}

\section{Operador de Cruce Intensivo}

\section{Estudio de la diversidad}

\section{Incrementando la diversidad (con nuevo reemplazo)}